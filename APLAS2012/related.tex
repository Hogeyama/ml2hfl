\section{Related Work}
\label{sec:related}

\subsection{Container Abstraction}
Dillig et al.~\cite{Dillig2011} have proposed an automatic technique for
static reasoning about containers.  The proposed framework is based on
an abstract interpretation for containers.  Their framework is similar
to our framework in the sense that they model containers as mappings
from location to values.  While they consider the client-side use of the
specific data structures (i.e. containers), we consider a more general
class of data structures include user-defined data structures.


\subsection{Verification Frameworks for Functional Programs}
There are several
frameworks~\cite{Kawaguchi2009,Chin2003,Unno2010,Ong2011} that aimed at
verification of functional programs with data structures.

Kawaguchi et al.'s liquid type inferecne~\cite{Kawaguchi2009} are
semi-automated verification framework based on refinement types.
Their framework requires the shape of predicates, called logical qualifiers, to users.
The limitation of our framework is that it cannot deal with
\cc{recursive dependency} on recursive data structures such that
orderedness.  Their framework can deal with recursive dependent type
such that $\App{\INT}{\LIST_\leq} = \mu t.\NIL +
\Cons{x_1\COL\INT}{x_2\COL\App{\{\nu\COL\INT \mid \nu \leq
x_1\}}{\LIST}}$, i.e., ordered lists of integers.

Chin et al.'s sized type inference~\cite{Chin2003}
\memo{inferred by fix-point computation}
Their framework cannot handle higher-order functions.

\memo{HMTT,PMRS}

\subsection{}
[Comparison with others (Soonho Kong et al. APLAS10, etc.)]
