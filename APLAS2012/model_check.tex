\section{Background: Predicate Abstraction and CEGAR for Higher-order Model Checking}
\label{sec:model-check}
%\memo{Explanation of previous work}
%\cc{
%A source program is translated into a higher-order program with integers
%by encoding recursive data structures. 
%The language for programs with integers is essentially the same as the source
%language in the previous section, except that it has no list constructors and destructors.
%}
We use our previous verification framework for higher-order
programs with integers~\cite{KobayashiPLDI2011}.  This framework is
based on the higher-order model
checking~\cite{Ong2006,KobayashiPOPL2009}, i.e. model checking of
higher-order recursion schemes, with techniques of predicate abstraction
and counter-example guided abstraction refinement for higher-order
functional programs.

Figure~\ref{fig:cegar} shows the overall structure of our previous
method.  We first translate the source program into boolean program by
predicate abstraction.  Here, predicates are given by the notion of
abstraction types.  Abstraction types are simple types with predicates
that is used as how programs are abstracted to boolean programs.  For
example, abstraction type
$\TFun{x\COL\INT[P_1,\dots,P_n]}{r\COL\INT[Q_1,\dots,Q_m]}$ denotes the
return values are abstracted by predicates $Q_1,\dots,Q_m$ assuming
argument are abstracted by predicates $P_1,\dots, P_n$.  We can verify
the abstracted program by using higher-order model checker.
Higher-order model checker, however, cannot be applied directly to the
abstracted program because of the gap between the semantics of the
abstracted program and recursion scheme that has a call-by-name
semantics.  To resolve the gap, we transform the abstracted program into
a CPS program.  The source program is safe if the abstracted program is
safe.  Otherwise, the model checker generates a counterexample for the
abstracted program.  We can find a corresponding concrete error trace,
and we can check the error path is feasible or not in the source program
by symbolic execution.  The source program is unsafe if the error trace
is feasible.  Otherwise, we can know that the abstraction is too coarse
to verify the source program.  We next refine abstraction types to
refute the counterexample, and continue the CEGAR-loop.

We do not show the details of predicate abstraction and CEGAR for higher-order functional programs here.
See the previous paper~\cite{KobayashiPLDI2011} for the details.


%\section{Model Checking for Higher-order Programs with Integers}
%\label{sec:model-check}
%[Introduction of language (of PLDI2011 + pairs.)]
%
%A source program is translated to a higher-order programs with integers
%by encoding lists as functions. The language of essentially the same as the source
%language in the previous section, except that lists constructor and destructor are not in $\LINT$.
%
%[Description of the framework of PLDI2011.]
%
%We can use the verification framework for higher-order programs with integers [refs].

