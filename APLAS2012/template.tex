%  
%  Template for CERIES-GCOE11
%               Copyright (C) 2011 Koichi Ito, all rights reserved.
%               Last updated: 2011/09/07
%

\documentclass[10pt]{article}
\usepackage{ceries}
\usepackage{latexsym}
\usepackage[dvips]{graphicx}

\title{Guidelines of Camera-Ready Paper Preparation for
  CERIES-GCOE11}

\author{
  F. Adachi$^1$ and N. Kato$^2$
}

\address{
  ${}^1$ Graduate School of Engineering, Tohoku University,
  Sendai 980-8579, Japan\\
  E-mail: xxx@ecei.tohoku.ac.jp, Phone: +81-22-795-0000,
  FAX: +81-22-263-0000\\
  ${}^2$ Graduate School of Information Sciences, Tohoku
  University, Sendai 980-8579, Japan
}

\begin{document}
\maketitle

\begin{abstract}
  The abstract appears in fully-justified text with
  10-point, single-spaced type, and \underline{\bf about 100
    words} in length, at the top of the left-hand column as
  it is here, below the author information. Use the word
  ``ABSTRACT'' as the title, in 10-point Times, boldface
  type and capitalized.
\end{abstract}

\section{Introduction}

This paper includes style, format, number of pages and
language for the camera-ready paper for CERIES-GCOE11. Please
follow this guideline for all authors of oral and poster
sessions.

\section{Format}

\subsection{Page size}

Please use A4 size white paper (210$\times$297 mm). The
printing area is 170mm$\times$247mm. Nothing outside this
area will be printed. The left-side, right-side, top, and
bottom margins are 20mm, 20mm, 25mm and 25mm, respectively.

\subsection{Page style}

The camera-ready paper must be \underline{\bf less than 4
  pages} and all manuscripts must be written in English. At
the center top of the first page, the title of paper, author
name(s), affiliation(s), address(es), corresponding author's
E-mail address, phone number, fax number, and the abstract
of paper should be described. Place a blank line below the
title, author name(s) and affiliation(s).

The abstract and main text should be two columns. Use single
spacing in the body of the text. The section should be
numbered starting with ``1. INTRODUCTION'' after the
abstract. Place a blank line between the sections.
Acknowledgment and References should not be numbered. Page
number should not be added for easily preparing the
proceedings book.


\subsection{Fonts}

Recommended font is Times or Times New Roman. Font size of
paper title, author name(s), affiliation(s), abstract,
normal text, headings, and text in figure caption are 12 pt
bold, 11 pt, 10 pt italic, 10 pt, 10 pt, 10 pt bold, and 10
pt, respectively. The section headings appear in capital
letters (e.g., 1. INTRODUCTION) and the sub-section headings
should be capital and lower letters (e.g., 2.1 Page
size). Acknowledgment and references may be typed with 10
pt.


\section{Numbering}

Figure captions should be below the figures. Table captions
should be above the tables. Arabic numerals should be used
for figures like Fig. \ref{fig:example}, and Arabic numerals
for tables like Table \ref{tbl:example}. Place a blank line
between the text and the figures or tables. We strongly
recommend using black and white images in your paper
because the camera-ready paper will be printed in black and
white.

The references appear in listed and numbered with brackets
at the end of paper as follows. In the case of referring to
the references in the text, the reference number should be
described like \cite{Himes10}.

The equations are numbered with parentheses as follows
\begin{equation}
  C = AB.
\end{equation}

\begin{figure}[t]
  \centering
  \includegraphics[width=.9\linewidth]{./CERIES.eps}
  \caption{Example of a figure.}
  \label{fig:example}
\end{figure}

\begin{table}[t]
  \caption{Example of a table.}
  \label{tbl:example}
  \centering
  \begin{tabular}{ccc}
    \hline
    Method & Accuracy & Speed\\
    \hline
    Conventional & 2 & 2\\
    Proposed     & 1 & 1\\
    \hline
  \end{tabular}
\end{table}


\section{Conclusion}

This paper is an example of the format for the camera-ready
paper of CERIES-GCOE11.


% Acknowledgment
\acknowledgment
We would like to cordially express thanks to all
contributors to CERIES-GCOE11 for their cooperation in the
symposium.


\begin{thebibliography}{99}
\bibitem{Himes10} C. Himes, E. Carlson, R.J. Ricchiuti,
  B.P. Otis, and B.A. Parviz,
  ``Ultralow voltage nano-electronics powered directly and
  solely from a tree,''
  IEEE Trans. Nanotech., Vol. 9, pp. 2--5, 2010.
\end{thebibliography}


\end{document}
