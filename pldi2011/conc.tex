\section{Conclusion}
\label{sec:conc}

We have proposed predicate abstraction and CEGAR techniques for higher-order model checking,
and implemented a prototype verifier. We believe that this is a new promising approach to
automatic verification of higher-order functional programs.
Optimization of the implementation and larger experiments are left for future work.

We conclude the paper with discussions on the scalability of our method.
The current implementation is not scalable for large programs, but
%%but 
we expect that (after several years of efforts) we can eventually obtain 
a much more scalable verifier based on our method, for several reasons. 
First, the complexity of the model checking of higher-order recursion 
schemes is \(n\)-EXPTIME complete~\cite{Ong2006}, but with certain 
parameters being fixed, the complexity is actually polynomial (linear, 
if restricted to safety properties) time in the size of a recursion 
scheme (or, the size of HBP). Furthermore, \(n\)-EXPTIME completeness is 
the \emph{worst-case} complexity, and recent model checking 
algorithms~\cite{Kobayashi2009c,Kobayashi2011a} do not immediately 
suffer from the \(n\)-EXPTIME bottleneck. Secondly, the implementation 
of the underlying higher-order model checker \trecs{} is premature, 
%(note that the first higher-order model checker was built only in 2009; 
%this should be contrasted to the maturity of finite state model checkers)
and there is a good chance for improvement. Thirdly, the current 
implementation of predicate abstraction and CEGAR is also quite naive. 
For example, the current implementation computes abstract programs 
eagerly. We expect that a good speed-up is obtained by computing 
abstract programs lazily.
%% (only when required for higher-order model checking).
