\section{Examples that require nested intersection types}
\label{sec:intersection}

Here we show an example that requires nested intersection types,
supporting the advantage over Jhala et al. and Unno and Kobayashi's approach
to dependent type inference.

Consider the following program:
\begin{verbatim}
let g x y = x in  (* int -> unit -> int *)
let twice f x y = f (f x) y in
let neg x y = -x() in (* (unit->int)->unit->int *)
  if n>=0 then assert(twice neg (g n) >=0) else ()
\end{verbatim}
In order to verify the program above with refinement types, 
we need to assign the following nested intersection type to \texttt{twice}:
\[\begin{array}{l}
 (((\TUNIT\ra \rtbase{\nu}{\INT}{\nu\geq 0})\ra \TUNIT\ra \rtbase{\nu}{\INT}{\nu\leq 0})\\
\land 
 (((\TUNIT\ra \rtbase{\nu}{\INT}{\nu\leq 0})\ra \TUNIT\ra \rtbase{\nu}{\INT}{\nu\geq 0})\\
\quad
   \ra (\TUNIT\ra \rtbase{\nu}{\INT}{\nu\geq 0}) \ra \TUNIT\ra \rtbase{\nu}{\INT}{\nu\geq 0}.
\end{array}
\]
The approaches of  \cite{Rondon2008,Unno2009} fail, as they look for only dependent types of the form:%
%%\footnote{Actually, this form is sufficient for the verification of the above program,
%%if we let \(P_2(\nu,y) \equiv (y\geq 0\imply \nu=-2y)\land (y<0\imply \nu=y)\). But that is difficult to infer.}
\[\begin{array}{l}
( (\TUNIT\ra \rtbase{\nu}{\INT}{P_1(\nu)})\ra (\TUNIT\ra \rtbase{\nu}{\INT}{P_2(\nu)}))\\
\quad
   \ra (\TUNIT\ra \rtbase{\nu}{\INT}{P_3(\nu)}) \ra 
   \TUNIT\ra \rtbase{\nu}{\INT}{P_4(\nu)}.
\end{array}
\]
Our method can verify the above program.
After a CEGAR loop, the following abstraction type is automatically inferred:
\[
\begin{array}{l}
((\TUNIT\ra \TINT[\lambda \nu.\nu\geq 0, \lambda \nu.\nu\leq 0]) 
\ra 
 \TUNIT\ra \TINT[\lambda \nu.\nu\geq 0, \lambda \nu.\nu\leq 0])\\
\quad \ra (\TUNIT\ra \TINT[\lambda \nu.\nu\geq 0]) \ra  \TUNIT\ra \TINT[\lambda \nu.\nu\geq 0].
\end{array}
\]
Then the higher-order model checker essentially infers the following type for the abstract version of \texttt{twice}:
\[
\begin{array}{l}
((\TUNIT\ra \TRUE\times \TBOOL) \ra \TUNIT\ra \TBOOL\times \TRUE)\\
\land ((\TUNIT\ra \TBOOL\times \TRUE) \ra \TUNIT\ra \TRUE\times \TBOOL)\\
\qquad \ra (\TUNIT\ra \TRUE) \ra \TUNIT\ra \TRUE,
\end{array}
\]
from which we can recover the dependent intersection type of \texttt{twice} given above.
