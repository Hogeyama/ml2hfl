\section{Experimental Comparison with depcegar}
\label{sec:depcegar}

Here we give some examples and experimental results
to support advantages of our approach over Terauchi's 
%\iffull 
method~\cite{Terauchi2010}. 
%\else method. \fi

The following is an example that supports the 
advantage (ii) over Terauchi's approach claimed in Section~\ref{sec:related}.
\begin{verbatim}
let rec f g x = if x<=0 then g x else f (f g) (x-1) in
let succ x = x+1 in
 assert(f succ 2 <1)
\end{verbatim}
The program is unsafe, but to find it, 
Terauchi's method needs to unfold \(f\) at least twice, to obtain:
\begin{verbatim}
let rec f0 g x = if x<=0 then g x else f1 (f1 g) (x-1) 
and f1 g x = if x<=0 then g x else f2 (f2 g) (x-1) 
and f2 g x = if x<=0 then g x else f3 (f3 g) (x-1) 
and f3 g x = f3 g x in
let succ x = x+1 in
 assert(f0 succ 2 <1)
\end{verbatim}
The \(\beta\)-normalization of the above program is required to infer 
dependent intersection types, and the number of required reduction steps is
at least exponential in the number of unfoldings.
(Actually, one can easily create an order-\(n\) program for which the number
of \(\beta\)-reduction steps is \(n\)-fold exponential in the number of unfoldings.)
We have tested \texttt{depcegar}~\cite{Terauchi2010} for the above program,
which did not terminate.
Our verifier can find a counterexample in less than a second.

We have also tested \texttt{depcegar} for some of the programs in Section~\ref{sec:experiment}.
The following table shows running times:
\begin{center}
\begin{tabular}{|l|l|}
\hline
program & time (sec.) \\
\hline
\texttt{hrec} & 0.369\\
\hline
\texttt{a-prod} & 0.494\\
\hline
\texttt{a-cppr} & 14.687\\
\hline
\texttt{a-init} & -\\
\hline
\texttt{l-zipunzip} & -\\
\hline
\texttt{l-zipmap} & -\\
\hline
\texttt{hors} & 0.693\\
\hline
\texttt{r-lock} & 0.640\\
\hline
\texttt{r-file} & -\\
\hline
\end{tabular}
\end{center}
\texttt{depcegar} did not terminate for \texttt{a-init}, \texttt{l-zipunzip},
\texttt{l-zipmap}, and \texttt{r-file} in 5 minutes.
We do not know what is going on in \texttt{depcegar} in those
experiments (as \texttt{depecegar} does not generate log messages), 
but we suspect that the results are due to the points (ii) and/or (iii) discussed in Section~\ref{sec:related}.


