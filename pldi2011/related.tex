\section{Related Work}
\label{sec:related}

\subsection{Model Checking of Higher-Order Programs}

%\iffull
The model checking of higher-order recursion schemes has been 
extensively 
%\iffull
%studied~\cite{Knapik2002,Ong2006,Hague2008,Kobayashi2009a}. 
%\else
studied~\cite{Knapik2002,Ong2006,Kobayashi2009a}. 
%\fi
Ong~\cite{Ong2006} proved the decidability of the modal \(\mu\)-calculus 
model checking of recursion schemes. Kobayashi~\cite{Kobayashi2009} then 
proposed a new framework of higher-order program verification based on 
the model checking of recursion schemes, already suggesting a use of 
predicate abstraction and CEGAR to deal with programs manipulating 
infinite data domain. There were two missing pieces in his framework, 
however. One was a practical model checking algorithm for recursion 
schemes (note that the model checking of recursion schemes is in general 
\(n\)-EXPTIME-complete), and the other was a method to apply predicate 
abstraction and CEGAR to higher-order programs. The former piece has 
been supplied later by Kobayashi~\cite{Kobayashi2009c}, and supplying 
the latter piece was the goal of the present paper.
%\else
%\delspan{
%Kobayashi~{\cite{Kobayashi2009}} has shown that verification problems for 
%functional programs with finite data domains can be reduced to the 
%recursion scheme model checking.
%%%and developed a practical model checker for recursion schemes.
%He~{\cite{Kobayashi2009}} also suggested an idea of using predicate 
%abstraction and CEGAR to deal with infinite data domains, but without 
%any concrete method: In particular, he did not discuss at all how 
%abstraction interface like abstraction types can be defined, how HBP can 
%be obtained by using the abstraction interface, or how predicates are 
%systematically discovered from spurious error paths to ensure the 
%progress property.
%}
%\fi

In parallel to the present work, Unno et 
al.~\cite{Kobayashi2010,Unno2010} and Ong and Ramsay~\cite{Ong2011} 
proposed applications of higher-order model checking to verification of 
tree-processing programs. Their approaches are radically different from 
ours. First, they use different abstraction techniques: tree data are 
abstracted using either tree automata~\cite{Kobayashi2010,Unno2010} or 
patterns~\cite{Ong2011}, which cannot abstract values using binary 
predicates (such as \(2\times x\geq y\)). Secondly, The method of 
\cite{Kobayashi2010} applies only to programs that can be expressed in 
the form of (higher-order) tree transducers, and the extension in 
\cite{Unno2010} requires user annotations. Ong and Ramsay's 
method~\cite{Ong2011} applies to general functional programs and 
includes a CEGAR mechanism, but the precision of their method is heavily 
affected by that of a variable binding analysis, and their CEGAR is 
completely different from ours.
%%Ong and Ramsay~\cite{Ong2011} introduces CEGAR, which is quite different from ours.
Their technique does not satisfy relative completeness like 
Theorem~\ref{th:rel-complete}.

%%%In parallel to the present work, Kobayashi et al.~\cite{Kobayashi2010} 
%%%investigated yet another approach to applying higher-order model checking to
%%%verification of higher-order programs.
%%%The technique and the target programs are quite different from those of the present work.
%%%First, they~\cite{Kobayashi2010} use a different abstraction technique;
%%%tree data are abstracted by using states of tree automata. 
%%%Binary predicates such as \(2\times x\geq y\) cannot be expressed in the automata-based approach,
%%%and no CEGAR-like technique is used.
%%%Secondly, their approach applies only to programs that can be expressed in the form
%%%of (higher-order) tree transducers.
%%%\iffull There is a clear distinction between
%%%input tree data and output tree data, and trees cannot be freely constructed
%%%and deconstructed in programs. 
%%%\fi
%%%Unno et al.~\cite{Unno2010} recently extended the method
%%%to handle arbitrary tree-processing programs, but it requires certain annotations of
%%%invariants on intermediate data structures.
%%%Ong and Ramsy~\cite{Ong2011} have also developed a verification method for tree-processing
%%%programs based on higher-order model checking. They introduce a CEGAR method to refine
%%%abstractions, but in a rather different way.



%%\nk{If ICALP submission is accepted, insert a comment on that paper.}

%%%The target programs of their approach are 
%%%by . , and Kobayashi et al. have applied the 
%%%decidability result to verification of higher-order functional 
%%%programs~\cite{Kobayashi2009,Kobayashi2009c}. The main limitation of 
%%%these methods is that they cannot handle programs that manipulate 
%%%infinite data domains such as integers and lists.
%%%%except \cite{Kobayashi2010} 
%%%To relax the limitation, some of the authors have proposed a 
%%%verification method for higher-order programs that manipulate tree data 
%%%structures~\cite{Kobayashi2010}. The method can handle integers by 
%%%encoding them as lists. However, the method cannot handle integer 
%%%operations well since the method does not support intermediate data 
%%%structures. Another limitation of the method is that specifications must 
%%%be regular. Thus, we cannot express the specification like \(x \geq y\) 
%%%on integer variables \(x\) and \(y\).

%complete for a fragment of HMTTs called linear HMTTs, however supports 
%only regular specifications.

\subsection{Dependent Type Inference}

There have been studies on automatic or semi-automatic inference of 
dependent types~\cite{Chin2001,Flanagan2006,Rondon2008,Unno2009,Jhala2009,Terauchi2010}.
%\nk{Add a pointer to Chin's work on size inference.}
There are similarities between the goals of those studies and that of 
our work. First, one of the goals of dependent type inference is to 
prove the lack of assertion failures, as in the present work. Secondly, 
our technique can actually be used for inferring dependent types. 
%%As described in \cite{Kobayashi2009c},
Recursion scheme model checker \trecs{}~\cite{Kobayashi2009c} is 
type-based, and produces type information as a certificate of successful 
verification. For example, for the abstraction of the last example in 
Section~\ref{sec:intro}, it infers the type \(\TUNIT\ra (\TRUE \ra 
\TUNIT)\ra\TUNIT\) for (the abstract version of) \(f\). Combined with 
the abstraction type of \(f\), we can recover the following dependent 
type for \(f:(x\COL \TINT \ra (y\COL\set{\nu\COL\TINT \mid \nu>x}\ra 
\TUNIT)\ra \TUNIT)\).

Though the goals are similar, the techniques are different. Rondon et 
al.'s liquid types~\cite{Rondon2008} requires users to specify 
predicates (or more precisely, shapes of predicates, called 
\emph{qualifiers}) used in dependent types. Jhala et al.~\cite{Jhala2009} 
proposed an automatic method for inferring qualifiers for liquid types. 
Their method extracts qualifiers from a proof that a finite unfolding of 
a source program does not get stuck, and has some similarity to our 
method to infer abstraction types from an error path.
%
Unno and Kobayashi~\cite{Unno2009} proposed an automatic method for 
inferring dependent types.
%%They also use interpolants, but in a different way from ours. 
They first prepare templates of dependent types (that contain predicate 
variables) and generate (possibly recursive) constraints on predicate 
variables. They then solve the constraints by using an interpolating 
theorem prover. Jhala et al.~\cite{Jhala2010} also propose a similar 
method, where they reduce the constraint solving in the last phase to 
model checking of imperative programs.
%%abstract interpretation and imperative program model checking to solve the constraints in the last phase.
These approaches~\cite{Rondon2008,Unno2009,Jhala2009,Jhala2010} do 
support higher-order functions, but in a limited manner, in the sense 
that nested intersection types are not allowed. The difference between 
dependent types with/without intersections is like the one between 
context (or flow) sensitive/insensitive analyses. The former is more 
precise though it can be costly.\footnote{Our method is an extreme case 
of context/flow sensitive analysis, which is sound and complete for 
programs with finite data domains.} In general, nested intersection 
types are necessary to verify a program when function parameters are 
used more than once in different contexts. Indeed, as discussed in 
\iffull
Appendix~\ref{sec:intersection}
\else
\cite{Kobayashi2011}, 
\fi
several of the programs in Section~\ref{sec:experiment} (e.g. 
\texttt{neg}, where the first argument of \texttt{twice} is used in two 
different contexts) require nested intersection types, and almost all 
the examples given by Kobayashi~\cite{Kobayashi2009,Kobayashi2009c} call 
for nested intersection types. 
%%We are not sure how often such situations occur
%%in the reachability verification of ordinary programs. However, 
%%If other verification problems
%%for higher-order programs are encoded into reachability problems, %%\cite{Kobayashi2009,Kobayashi2009c},
%%then nested intersection types are often required; in fact, almost all the examples given in 
%%\cite{Kobayashi2009,Kobayashi2009c} call for nested intersection types. 
%
%Our method supports both nested intersection types and automatic 
%inference of dependent types.

%(especially compared with a more mature one of Liquid 
%types~\cite{Rondon2008})
The limitation of our current prototype implementation is that the 
supported language features are limited. We believe that it is possible 
to extend our implementation to deal with data structures. In fact, the 
predicate abstraction introduced in Section~\ref{sec:pred} applies to 
data structures 
%%\iffull (recall Example~\ref{ex:list}), \else (see an example in \cite{Kobayashi2011}) \fi 
given an appropriate theorem prover. We expect the CEGAR part can also 
be extended, e.g. by restricting the properties on data structures to 
size properties, by treating data constructors as uninterpreted function 
symbols, etc.
%%%\iffull
%%%Another limitation of our approach is that the result of our verification is less predictable.
%%%In fact, as in the experiments for \texttt{xxx} and \texttt{yyy}, a slight change of a 
%%%source program makes the system to fail to discover appropriate predicates.
%%%This is because the CEGAR technique for automatic discovery of predicates is necessarily heuristic.
%%%We plan to improve the heuristic for finding predicates.
%%%\fi

Technically, most closest to ours is Terauchi's work~\cite{Terauchi2010}. 
In his method, candidates for dependent types are inferred from a finite 
unfolding of a source program,
%%by using an interpolating theorem prover, 
and then a fixedpoint computation algorithm is used to filter out 
invalid types. If the source program is not typable with the candidates 
for dependent types, the program is further unfolded and more candidates 
are collected. This cycle (which may diverge) is repeated until the 
source program is found to be well-typed or ill-typed. This is somewhat 
similar to the way our verification method works: abstraction types are 
inferred from an error trace (instead of an unfolding of a program), and 
then higher-order model checking (which also involves a fixedpoint 
computation) is applied to verify the abstract program. If the 
verification fails and an infeasible error path is found, the error path 
is used to infer more predicates, and this cycle is repeated. Thus, 
roughly speaking, our CEGAR phase corresponds to that of Terauchi to 
find candidates for dependent types, and our phases for predicate 
abstraction and higher-order model checking corresponds to Terauchi's 
fixedpoint computation phase.
%%\koba{Stress that our technique is strictly more powerful than Terauchi's type system.}
Advantages of ours are: (i) our method can generate an error path as a 
counterexample; there is no false alarm. On the other hand, a 
counterexample of Terauchi's method is an unfolding of a program, which 
may actually be safe. (ii) We infer predicates from an error trace, 
rather than from an unfolding of a program; From the latter, too many 
constraints are generated, especially for programs containing non-linear 
recursions.
%%\todo{Experiments to support this claim.}
%%(Note that Terauchi's intersection type inference algorithm requi
%%\iffull
%%generated\footnote{In fact,
%%his algorithm first infers the shape of dependent types by intersection type inference, 
%%which requires as many steps as the number of \(\beta\)-reductions.
%%The number of \(\beta\)-reduction steps for simply-typed \(\lambda\)-terms (without recursion) is known to be 
%%non-elementary.}, 
%%\else
%%generated,
%%\fi
%%which can be a burden to
%%the underlying interpolating theorem prover. 
(iii) Our method can find dependent types constructed from arbitrary 
boolean combinations of the inferred predicates, while Terauchi's method 
only looks for dependent types constructed from the formulas directly 
generated by an interpolating theorem prover; thus, the success of the 
latter more heavily relies on the quality or heuristics of the 
underlying interpolating theorem prover.
%
%(Note that as discussed at the end of Section~\ref{sec:experiment}, 
%whether appropriate predicates are obtained through an interpolating 
%theorem prover are somewhat unpredictable.)
%
%%(A price to pay instead in our method is 
%%that the computation of predicate abstraction and higher-order model checking can be more costly.)
(iv) Because of the point (iii) above, our method (predicate abstraction 
+ higher-order model checking) can also be used in a liquid type-like 
setting~\cite{Rondon2008} where atomic predicates are given by a user.
%%(but unlike liquid types, our method can infer nested intersection types),
(v) Because of the point discussed in 
Remark~\ref{rem:more-expressive-than-deptype}, the combination of our 
predicate abstraction and higher-order model checking is strictly more 
powerful than Terauchi's approach (as long as suitable predicates are 
found).
%
(vi) Our method can be extended to verify more general properties 
(expressed by the modal \(\mu\)-calculus), by appealing to the results 
on higher-order model checking~\cite{Ong2006,Kobayashi2009a}.
%%\cite{Kobayashi2011}.
%%Some of the above claims are supported by experimental results: see \iffull Appendix~\ref{sec:depcegar}.
%%\else \cite{Kobayashi2011}. \fi
%%For a fairer comparison, however, larger experiments are required in both approaches.
%%%\iffull
%%%(The programs \texttt{sum} and \texttt{mult} used in our experiments 
%%%are almost the same as those tested by Terauchi~\cite{Terauchi2010},
%%%and the running times are comparable. We cannot, however, draw any conclusion from such small experiments.%
%%%\footnote{Actually, we have tried other examples in the paper with his system on the web, but 
%%%many of them could not be tested because of unknown errors.
%%%For \texttt{mc91}, Terauchi's system failed
%%%because of a failure of the underlying theorem prover, so it is not clear whether this shows a fundamental 
%%%limitation of his method. })
%%%\fi
%%%type constraints 
%%%Their method 
%%%The notion of counterexamples is however different
%%%from that of the present work; a counterexample in their work is an ill-typed 
%%%infer dependent types completely automatically. 
%%%Dependent types have been applied to verification of higher-order 
%%%functional programs with infinite data 
%%%domains~\cite{Flanagan2006,Rondon2008,Unno2009,Terauchi2010}.
%%%%
%%%Rondon et al. have proposed liquid types~\cite{Rondon2008}, which are 
%%%dependent types whose base types can be refined by using user-specified  
%%%predicates.
%%%% as in our abstraction types.
%%%Their system verifies a program by automatically inferring liquid types 
%%%of the program. However, the system does not support counterexample 
%%%finding and automatic abstraction refinement unlike in our system.
%%%%predicate discovery
%%%
%%%Some of the authors have proposed a dependent type inference 
%%%method~\cite{Unno2009}, which supports counterexample finding and 
%%%automatic abstraction refinement using an interpolating theorem prover. 
%%%%counterexample guided 
%%%%the depedent type system used in
%%%However, the method is incomplete for higher-order programs in a sense 
%%%that a counterexample reported by the method may be infeasible.
%%%% with finite data domains
%%%In contrast, our present verification framework is complete for 
%%%higher-order programs.
%%%%sound but not terminating
%%%
%%%Terauchi has recently proposed another dependent type inference 
%%%method~\cite{Terauchi2010}, which is complete for higher-order programs 
%%%thanks to the use of intersection types. 
%
%%%\nk{Rewrite below.}
%%%Given a program and candidate types, which are essentially the same as 
%%%our abstraction types, his system automatically infers intersection 
%%%types of the program that can be constructed from the candidate types. 
%%%If the program cannot be typed by using the current candidate types, his 
%%%system can automatically find new candidate types.
%%%%
%%%Our method can be seen as a similar dependent type inference method: the 
%%%predicate abstraction part (that uses dependent types) in combination 
%%%with the higher-order model checking part~\cite{Kobayashi2009} (that 
%%%uses intersection types) essentially infer intersection types of a 
%%%program that can be constructed from abstraction types.
%%%%which we believe is as expressive as Terauchi's type system.
%%%%\todo{The CEGAR part is essentially different from ...}
%%%
%%%In contrast to the verification methods based on dependent types, our 
%%%method is carefully designed so that we can solve a verification problem 
%%%by clearly dividing it into three sub-problems: predicate abstraction, 
%%%higher-order model checking, and CEGAR. The division of the problem may 
%%%make the verification slower. However, the division enables us to extend 
%%%our method easily. For example, we can support advanced language 
%%%features such as recursive data structures and exceptions, and/or 
%%%support verification of other safety properties than reachability such 
%%%as resource usage safety~\cite{Igarashi2005} by just extending the 
%%%predicate abstraction part.
%%%%
%%%Moreover, we can support modal-\(\mu\) calculus model checking by 
%%%replacing only the model checking part with \cite{Kobayashi2009a}.
%%%%
%%%The CEGAR part alone can be replaced if more effective predicate 
%%%discovery method than the interpolation-based one proposed in this paper 
%%%is proposed.
%%%
%%%% and/or more advanced type systems than the simple type system used in 
%%%%this paper \cite{Tsukada2010}
%%%
%%%%%%we can further optimize the predicate abstraction part and the CEGAR part separately.
%%%%
%%%%The model-checking part~\cite{Kobayashi2009,Kobayashi2009d} is sound and 
%%%%complete, supports more expressive specifications and possibly more 
%%%%time-efficient.
%%%
%%%%the abstraction type \(b[\seq{P}]\) used in our method can be seen as 
%%%%the set of refinement types of the form \(\set{\nu:b \mid \phi_i}\), 
%%%%where \(\phi_i\) is a propositional formula whose propositions are 
%%%%\(\seq{P}(\nu)\)
%%%
%%%%

\subsection{Traditional Model Checking}

Predicate abstraction and CEGAR have been extensively studied in the 
context of finite state or pushdown model 
checking~\cite{Graf1997,Ball2001,Ball2002,Ball2005,Clarke2003a,Henzinger2002,Henzinger2004,Heizmann2010}. 
%
Predicate abstraction has also been applied to the game-semantics-based 
model checking~\cite{Bakewell2009}. We are not, however, aware of 
previous work that applies predicate abstraction and CEGAR to 
higher-order model checking. As discussed in Section~\ref{sec:intro}, 
the extension of predicate abstraction and CEGAR to higher-order model 
checking is non-trivial. One may think that 
defunctionalization~\cite{Reynolds1972} can be used to eliminate 
higher-order functions and apply conventional model checking. The 
defunctionalization however uses recursive data structures to represent 
closures, so that the resulting verification method is too imprecise, 
unless a clever abstraction technique for recursive data structures is 
available.

%
The three components of our verification method, predicate abstraction, 
higher-order model checking (\trecs{}), and CEGAR, may be seen as 
higher-order counterparts of the three components of 
SLAM~\cite{Ball2001,Ball2002,Ball2005}: \textsc{C2bp}, \textsc{Bebop}, 
and \textsc{Newton}.
%
Our use of dependency in abstraction types appears to subsume Ball et 
al.'s polymorphic predicate abstraction~\cite{Ball2005}. For example, 
the \texttt{id} function in \cite{Ball2005} can be abstracted by using 
the abstraction type {\(x:\TINT[\,] \to \TINT[\lambda y.y=x]\)}.

There are a lot of studies to optimize predicate abstraction (especially 
for optimizing or avoiding the costly computation of abstract transition 
functions) in the context of conventional model 
checking~\cite{Ball2001,McMillan2005}. We have already borrowed some of 
the optimization techniques as mentioned in Section~\ref{sec:experiment}, 
and plan to adapt more techniques.

%One may think that traditional first-order model checking can be applied 
%to verification of higher-order programs by using 
%defunctionalization~\cite{Reynolds1972}, a program transformation which 
%eliminates higher-order functions. However, defunctionalization requires 
%recursive types (to express closures), so that the defunctionalized 
%version of a simply-typed program is not simply-typed. Thus, the 
%resulting verification method is no longer complete even for 
%higher-order boolean programs. Without very good abstraction mechanisms 
%for recursive data structures, the resulting method will be very 
%imprecise for higher-order functional programs.

%
%%%\cite{Graf1997} have proposed predicate abstraction, which abstracts an 
%%infinite state system to a finite state one for finite state model 
%%checking. \cite{Ball2001} have applied predicate abstraction for 
%%model-checking C programs. 

%%%CEGAR has been originally proposed by Clarke et al.~\cite{Clarke2003a}, 
%%%and applied to software model checking~\cite{Ball2002,Henzinger2002}. 
%%%\cite{Henzinger2004} have shown that an interpolating theorem 
%%%prover~\cite{McMillan2005,Beyer2008} can be used to systematically 
%%%extract new predicates from a spurious counterexample (i.e., an 
%%%infeasible error path). However, their method is limited to first-order 
%%%programs. In this paper, we have proposed a CEGAR for higher-order 
%%%programs based on an interpolating theorem prover.

%saving costs for computing abstract transition: lazy 
%abstraction~\cite{Henzinger2002}, proofs from counterexamples~\cite{McMillan2006,Heizmann2010}

%FOCI~\cite{McMillan2005}
%CSIsat~\cite{Beyer2008},
%CLPprover~\cite{}
%MathSAT~\cite{}

\subsection{Abstract Interpretation}
The combination of predicate abstraction and higher-order model checking 
may be viewed as a kind of abstract interpretation~\cite{Cousot1978}. 
The abstract domain used for each functional value is defined by 
abstraction types, and predicate abstraction transforms a source program 
into an HBP whose semantics corresponds to the abstract semantics of the 
source program. Higher-order model checking then computes the abstract 
semantics. An advantage here is that thanks to the model checking 
algorithm~\cite{Kobayashi2009c} for higher-order recursion schemes, the 
computation of the abstract semantics is often much faster than a naive 
fixed-point computation (which is extremely costly for higher-order 
function values).
%%We are not aware of (implementation of) abstract interpreter-based
%%verification methods for higher-order functional programs that are directly relevant to ours.
