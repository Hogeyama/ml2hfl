\section{Construction of SHP}
\label{sec:shp}

%%The general construction is given below.

This section gives the general construction of SHP.

Consider a program normalized to the following form:
\begin{eqnarray*}
D&::=& \set{f_1\ \seq{x}_1 = e_{10}\PAROP e_{11},\ldots,f_m\ \seq{x}_m = e_{m0}\PAROP e_{m1}} \\
e&::=& \ASSUME{v}{\CALL} \mid \LETEQIN{x}{\OP(\seq{v})}{\CALL}\\
\CALL &::=& c \mid x\ \seq{v} \mid f\ \seq{v} \mid \FAIL \\
v&::=& c \mid x\ \seq{v} \mid f\ \seq{v}
%%v&::=& c \mid x \mid f\ \seq{v}
\end{eqnarray*}
Here, for the sake of simplicity, we have assumed that every function definition has
at most one (tail) function call, and the return value is \(\tuple{}\); this does not
lose generality as the normal form can be obtained by applying CPS transformation and \(\lambda\)-lifting.
Given a path \(s = b_1\cdots b_\ell\) of \(D\)
(which means that the branch \(b_i\) has been chosen at \(i\)th function call), 
the corresponding SHP \(D' = \mkSHP(D, s)\) is given by:
\[
\begin{array}{l}
D' = \{\Copy{f_i}{j}\ \seq{x}_i = \X{e_{ib_j}}{j+1} \mid i\in\set{1,\ldots,m}, j\in\set{1,\ldots,\ell},\\
\qquad               \mbox{the target of the \(j\)th function call is \(f_i\)}\}\\
\quad \cup 
  \{\Copy{f_i}{j}\ \seq{x}_i = \tuple{} \mid i\in\set{1,\ldots,m}, j\in\set{1,\ldots,\ell},\\
\qquad                  \mbox{the target of the \(j\)th function call is not \(f_i\)}\}\\
\quad \cup \set{\textit{main}\tuple{} = \Copy{\textit{main}}{1}\tuple{}}
%%
%%\Copy{f_i}{j}\ {\seq{x}_i} = 
%% \left\{\begin{array}{ll}
%%           \X{e_{ib_j}}{j} &
%%   \mbox{if 
%%        {\tuple{}}    & \mbox{otherwise}
%%        \end{array}
%%  \right.
\end{array}
\]
where \(\X{e}{j}\) is given by:
\[
\begin{array}{l}
\X{\ASSUME{v}{\CALL}}{j} = \ASSUME{v}{\X{\CALL}{j}}\\
\X{\LETEQIN{x}{\OP(\seq{v})}{\CALL}}{j} = \LETEQIN{x}{\OP(\seq{v})}{\X{\CALL}{j}}\\
\X{c}{j} = c\qquad
\X{\FAIL}{j} = \FAIL\qquad
\X{x}{j} = x\\
\X{x\ v_1\,\cdots\,v_k}{j} = \NTH{j}(x)\ \Dup{v_1}{j+1}\,\cdots\,\Dup{v_k}{j+1} \qquad(k \geq 1)\\
\X{f\ v_1\,\cdots\,v_k}{j} = \Copy{f}{j}\ \Dup{v_1}{j+1}\,\cdots\,\Dup{v_k}{j+1}\\
\Dup{c}{j} = c\qquad
\Dup{x}{j} = x \quad \mbox{(if \(x\) is a base variable)}\\
\Dup{(x\ \seq{v})}{j} =
                   \tuple{\underbrace{\lambda\seq{y}.\tuple{},\ldots,\lambda\seq{y}.\tuple{}}_{j-1},\NTH{j}(x)(\Dup{\seq{v}}{j}),\ldots,\NTH{\ell}(x)(\Dup{\seq{v}}{j})}\\
 \quad \mbox{(if \(x\) is a function variable)} \\
\Dup{(f\ \seq{v})}{j} =
                   \tuple{\underbrace{\lambda\seq{y}.\tuple{},\ldots,\lambda\seq{y}.\tuple{}}_{j-1},\Copy{f}{j}(\Dup{\seq{v}}{j}),\ldots,\Copy{f}{\ell}(\Dup{\seq{v}}{j})}\\
\end{array}
\]
Here, each function parameter has been replaced by a \(\ell\)-tuple of functions.

