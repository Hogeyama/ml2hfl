\section{Counterexample-Guided Abstraction Refinement (CEGAR)}
\label{sec:cegar}

This section describes a CEGAR procedure to discover new predicates used 
for predicate abstraction when the higher-order model checker \trecs{} 
has reported an error path \(s\) of a boolean program. 
%\koba{Revise Figure~1, and refer to it}

\subsection{Feasibility checking}

Given an error path \(s\) of an abstract program, we first check whether 
\(s\) is feasible in the source program \(D\), i.e. whether 
\(\textit{main}\tuple{}\redswith{s}{D}\FAIL\). This can be easily 
checked by actually executing the source program along the path \(s\), 
and checking whether all the branching conditions are true. (Here, we 
assume that the program is closed. If we allow free variables for 
storing base values, we can just symbolically execute the source program 
along the path, and check whether all the conditions are satisfiable.) 
If the source program indeed has the error path (i.e. 
\(\textit{main}\tuple{}\redswith{s}{D}\FAIL\)), then we report the error 
path as a counterexample.

\subsection{Predicate discovery and refinement of abstraction types}
If the error path is infeasible (i.e. \(\textit{main}\tuple{}\not\redswith{s}{D}\FAIL\)), 
we find new predicates to refine predicate abstractions.

In the case of the model checking of first-order programs, this is 
usually performed by, for each program point \(\ell\) in the error path, 
(i) computing the strongest condition \(C_1\) at \(\ell\), (ii) 
computing the weakest condition \(C_2\) for reaching from \(\ell\) to 
the failure node, and (iii) using a theorem prover to find a condition 
\(C\) such that \(C_1\! \Rightarrow\! C\) and \(C\! \Rightarrow\! \neg 
C_2\). Then the predicates in \(C\) can be used for abstracting the 
state at the program point. For example, in the reduction sequence (2) 
of \(M_1\) in Section~\ref{sec:intro}, the condition \(C_1\) on the 
local variable \(x\) is \(n\!>\!0\land x\!=\!n\), and the condition 
\(C_2\) is \(x+1\geq 0\). From them, we obtain \(C \equiv x>0\) as a 
predicate for abstracting \(x\).

It is unclear, however, how to extend it to deal with higher-order 
functions. For example, in the example above, how can we find a suitable 
abstraction type for functions \(f\) and \(g\)? To address this issue, 
as mentioned in Section~\ref{sec:intro}, we use the following type-based 
approach. From an infeasible error path, we first construct a 
\emph{straightline higher-order program} (abbreviated to SHP, which is 
straightline in the sense that it contains neither branches nor 
recursion and that each function is called at most once) that has 
exactly one execution path, corresponding to the path \(s\) of the 
source program. We then infer the dependent types of functions in the 
straightline program, and use the predicates occurring in the dependent 
types for refining abstraction types of the source program. We describe 
each step in more detail below.


\subsubsection{Constructing SHP}
\label{sec:straight}

Given a source program and a path \(s\), the corresponding SHP is 
obtained by (i) making a copy of each function for each call in the 
execution path, and (ii) for each copy, removing the branches not taken 
in \(s\).

\begin{example}
\label{ex:shp}
Recall the program \(M_3\) in Section~\ref{sec:intro}.
\[
\begin{array}{l}
\textit{main}\tuple{} = k\ m \qquad f\ x\ g = g(x+1)\\
h\ z\ y = \PAR{(\ASSUME{y>z}{\tuple{}})}{(\ASSUME{\neg (y>z)}{\FAIL})}\\
k\ n = {(\ASSUME{n\geq 0}{f\ n\ (h\ n)})}
%%\qquad\qquad 
\PAROP {(\ASSUME{\neg (n\geq 0)}{\tuple{}})}\\
\end{array} 
\]
%%(where \(n\) is some constant).
Here, we have represented conditionals and assert expressions in our 
language.\footnote{Here, for the sake of simplicity, we assume that \(m\) 
is some integer constant. As already mentioned, the random number 
generator \(\textit{randi}\) can actually be encoded in our language.} 
Given the spurious error path \(0\cdot 1\), we obtain the following SHP.
%%\(\textit{SP}_1\).
\[
\begin{array}{ll}
\textit{main}\tuple{} = k\ m \qquad
h\ z\ y = {\ASSUME{\neg (y>z)}{\FAIL}}\\
f\ x\ g = g(x+1) \qquad
k\ n = {\ASSUME{n\geq 0}{f\ n\ (h\ n)}} 
\end{array} 
\]
It has been obtained by removing irrelevant non-deterministic branches 
in \(h\) and \(k\).
\end{example}

The construction of an SHP generally requires duplication of function 
definitions and function parameters. For example, consider the following 
program:
\[
\begin{array}{l}
\textit{main}\tuple{} = k\ m \qquad
\textit{twice}\ f\ x = f(f\ x)\\
g\ x = \IFTE{x\leq 0}{1}{2+g(x-1)}\\
k\ n = \LETEQIN{x}{\textit{twice}\ g\ n}
                {\ASSERT(x>0)}\\
\end{array}
\]
(where \(m\) is some integer constant). The program calls the function 
\(g\) twice, and asserts that the result \(x\) is positive.
%%%\begin{verbatim}
%%%let twice f x = f (f x)
%%%let g x = if x=0 then 1 else 2+(g(x-1))
%%%let main = let x = twice g 0 in assert(x>0)
%%%\end{verbatim}
Suppose that an infeasible path \(0101\) has been given, which 
represents the following (infeasible) execution path:
\[
\begin{array}{l}
\textit{main}\tuple{} \red{} k\ m\reds{}  \LETEQIN{x}{g (g\ m)}{\cdots}\\
\qquad \stackrel{0}{\reds{}}    \LETEQIN{x}{g(1)}{\cdots} %%\\
%%\qquad 
\stackrel{1}{\reds{}}    \LETEQIN{x}{2+g(0)}{\cdots}\\
\qquad \stackrel{0}{\reds{}}    \LETEQIN{x}{2+1}{\cdots} %%\\
%%\qquad 
\reds \ASSERT{3>0} %%\\ %%(= (\ASSUME{3>0}{\tuple{}})\PAROP(\ASSUME{3\leq 0}{\FAIL})) %%\\
%%\qquad 
\stackrel{1}{\reds{}} \FAIL
\end{array}
\]
The path is infeasible because the final transition is invalid.

From the source program and the path above, we construct the following 
straightline program:\footnote{For clarity, we have extended our 
language with tuples of functions. If necessary, they can be removed by 
the currying transformation.}
\[
\begin{array}{l}
\textit{main}\tuple{} = k\ m \qquad
\textit{twice}\ \tuple{\Copy{f}{1},\Copy{f}{2}}\ x = \Copy{f}{2}(\Copy{f}{1}\ x)\\
\Copy{g}{1}\ x = \ASSUME{x\leq 0}{1}\qquad
\Copy{g}{3}\ x =  \ASSUME{x\leq 0}{1}\\
\Copy{g}{2}\ x = \ASSUME{\neg(x\leq 0)}{2+\Copy{g}{3}(x-1)}\\
k\ n = \LETEQIN{x}{\textit{twice}\,\tuple{\Copy{g}{1},\Copy{g}{2}}\,n}
 \ASSUME{\neg (x>0)}{\FAIL}\\
\end{array}
\]
As \(g\) is called three times, we have prepared three copies 
\(\Copy{g}{1}\), \(\Copy{g}{2}\), \(\Copy{g}{3}\) of \(g\), and 
eliminated unused non-deterministic branches. Note that the function 
parameter \(f\) of \(\textit{twice}\) has been replaced by a function 
pair \(\tuple{\Copy{f}{1},\Copy{f}{2}}\) accordingly.
%%\begin{verbatim}
%%let twice (f1, f2) x = f2 (f1 x)
%%let g1 x = assume(x=0)
%%let g2 x = assume(x!=0); 2+g3(x-1)
%%let g3 x = assume(x=0)
%%let main = let x = twice (g1,g2) 0 in assert(x>0)
%%\end{verbatim}

The general construction is given below. Consider a program normalized 
to the following form:
\begin{eqnarray*}
D&::=& \set{f_1\ \seq{x}_1 = e_{10}\PAROP e_{11},\ldots,f_m\ \seq{x}_m = e_{m0}\PAROP e_{m1}} \\
e&::=& \ASSUME{v}{\CALL} \mid \LETEQIN{x}{\OP(\seq{v})}{\CALL}\\
\CALL &::=& \tuple{} %c
\mid x\ \seq{v} \mid f\ \seq{v} \mid \FAIL \\
v&::=& c \mid x\ \seq{v} \mid f\ \seq{v}
%v&::=& c \mid x \mid f\ \seq{v}
\end{eqnarray*}
Here, for the sake of simplicity, we have assumed that every function definition has
at most one (tail) function call, and the return value is \(\tuple{}\); this does not
lose generality as the normal form can be obtained by applying CPS transformation and \(\lambda\)-lifting.
Given a path \(s = b_1\cdots b_\ell\) of \(D\)
(which means that the branch \(b_i\) has been chosen at \(i\)th function call), 
the corresponding SHP \(D' = \mkSHP(D, s)\) is given by:
\[
\begin{array}{l}
D' = \{\Copy{f_i}{j}\ \seq{x}_i = \X{e_{ib_j}}{j+1} \mid i\in\set{1,\ldots,m}, j\in\set{1,\ldots,\ell},\\
\qquad               \mbox{the target of the \(j\)th function call is \(f_i\)}\}\\
\quad \cup 
  \{\Copy{f_i}{j}\ \seq{x}_i = \tuple{} \mid i\in\set{1,\ldots,m}, j\in\set{1,\ldots,\ell},\\
\qquad                  \mbox{the target of the \(j\)th function call is not \(f_i\)}\}\\
\quad \cup \set{\textit{main}\tuple{} = \Copy{\textit{main}}{1}\tuple{}}
%%
%%\Copy{f_i}{j}\ {\seq{x}_i} = 
%% \left\{\begin{array}{ll}
%%           \X{e_{ib_j}}{j} &
%%   \mbox{if 
%%        {\tuple{}}    & \mbox{otherwise}
%%        \end{array}
%%  \right.
\end{array}
\]
%where 
Here, \(\X{e}{j}\) is given by:
\[
\begin{array}{l}
\X{\ASSUME{v}{\CALL}}{j} = \ASSUME{v}{\X{\CALL}{j}}\\
\X{\LETEQIN{x}{\OP(\seq{v})}{\CALL}}{j} = \LETEQIN{x}{\OP(\seq{v})}{\X{\CALL}{j}}\\
%\X{c}{j} = c\qquad
\X{\tuple{}}{j} = \tuple{}\qquad
\X{\FAIL}{j} = \FAIL\qquad
\X{x}{j} = x\\
\X{x\ v_1\,\cdots\,v_k}{j} = \NTH{j}(x)\ \Dup{v_1}{j+1}\,\cdots\,\Dup{v_k}{j+1} \qquad(k \geq 1)\\
\X{f\ v_1\,\cdots\,v_k}{j} = \Copy{f}{j}\ \Dup{v_1}{j+1}\,\cdots\,\Dup{v_k}{j+1}\\
\Dup{c}{j} = c\qquad
\Dup{x}{j} = x \quad \mbox{(if \(x\) is a base variable)}\\
\Dup{(x\ \seq{v})}{j} =
                   \tuple{\underbrace{\lambda\seq{y}.\tuple{},\ldots,\lambda\seq{y}.\tuple{}}_{j-1},\NTH{j}(x)(\Dup{\seq{v}}{j}),\ldots,\NTH{\ell}(x)(\Dup{\seq{v}}{j})}\\
 \quad \mbox{(if \(x\) is a function variable)} \\
\Dup{(f\ \seq{v})}{j} =
                   \tuple{\underbrace{\lambda\seq{y}.\tuple{},\ldots,\lambda\seq{y}.\tuple{}}_{j-1},\Copy{f}{j}(\Dup{\seq{v}}{j}),\ldots,\Copy{f}{\ell}(\Dup{\seq{v}}{j})}\\
\end{array}
\]
Here, each function parameter has been replaced by a \(\ell\)-tuple of 
functions.

The SHP \(\mkSHP(D,s)\), constructed from a source program \(D\) and a 
spurious error path \(s\), contains neither recursion nor 
non-deterministic branch, and is reduced to \(\FAIL\) if and only if 
\(\textit{main}\tuple{}\redswith{s}{D} \FAIL\). Furthermore, each 
function in the SHP is called at most once.

%%%Consider a program normalized to the following form:
%%%\begin{eqnarray*}
%%%D&::=& \set{f_1\ \seq{x}_1 = e_{10}\PAROP e_{11},\ldots,f_m\ \seq{x}_m = e_{m0}\PAROP e_{m1}} \\
%%%e&::=& \ASSUME{v}{\CALL} \mid \LETEQIN{x}{\OP(\seq{v})}{\CALL}\\
%%%\CALL &::=& c \mid x\ \seq{v} \mid f\ \seq{v} \mid \FAIL \\
%%%v&::=& c \mid x\ \seq{v} \mid f\ \seq{v}
%%%%%v&::=& c \mid x \mid f\ \seq{v}
%%%\end{eqnarray*}
%%%Here, for the sake of simplicity, we have assumed that every function definition has
%%%at most one (tail) function call, and the return value is \(\tuple{}\); this does not
%%%lose generality as the normal form can be obtained by applying CPS transformation and \(\lambda\)-lifting.
%%%Given a path \(s = b_1\cdots b_\ell\) of \(D\)
%%%(which means that the branch \(b_i\) has been chosen at \(i\)th function call), 
%%%the corresponding SHP \(D' = \mkSHP(D, s)\) is given by:
%%%\[
%%%\begin{array}{l}
%%%D' = \{\Copy{f_i}{j}\ \seq{x}_i = \X{e_{ib_j}}{j+1} \mid i\in\set{1,\ldots,m}, j\in\set{1,\ldots,\ell},\\
%%%\qquad               \mbox{the target of the \(j\)th function call is \(f_i\)}\}\\
%%%\quad \cup 
%%%  \{\Copy{f_i}{j}\ \seq{x}_i = \tuple{} \mid i\in\set{1,\ldots,m}, j\in\set{1,\ldots,\ell},\\
%%%\qquad                  \mbox{the target of the \(j\)th function call is not \(f_i\)}\}\\
%%%\quad \cup \set{\textit{main}\tuple{} = \Copy{\textit{main}}{1}\tuple{}}
%%%%%
%%%%%\Copy{f_i}{j}\ {\seq{x}_i} = 
%%%%% \left\{\begin{array}{ll}
%%%%%           \X{e_{ib_j}}{j} &
%%%%%   \mbox{if 
%%%%%        {\tuple{}}    & \mbox{otherwise}
%%%%%        \end{array}
%%%%%  \right.
%%%\end{array}
%%%\]
%%%where \(\X{e}{j}\) is given by:
%%%\[
%%%\begin{array}{l}
%%%\X{\ASSUME{v}{\CALL}}{j} = \ASSUME{v}{\X{\CALL}{j}}\\
%%%\X{\LETEQIN{x}{\OP(\seq{v})}{\CALL}}{j} = \LETEQIN{x}{\OP(\seq{v})}{\X{\CALL}{j}}\\
%%%\X{c}{j} = c\qquad
%%%\X{\FAIL}{j} = \FAIL\qquad
%%%\X{x}{j} = x\\
%%%\X{x\ v_1\,\cdots\,v_k}{j} = \NTH{j}(x)\ \Dup{v_1}{j+1}\,\cdots\,\Dup{v_k}{j+1} \qquad(k \geq 1)\\
%%%\X{f\ v_1\,\cdots\,v_k}{j} = \Copy{f}{j}\ \Dup{v_1}{j+1}\,\cdots\,\Dup{v_k}{j+1}\\
%%%\Dup{c}{j} = c\qquad
%%%\Dup{x}{j} = x \quad \mbox{(if \(x\) is a base variable)}\\
%%%\Dup{(x\ \seq{v})}{j} =
%%%                   \tuple{\underbrace{\lambda\seq{y}.\tuple{},\ldots,\lambda\seq{y}.\tuple{}}_{j-1},\NTH{j}(x)(\Dup{\seq{v}}{j}),\ldots,\NTH{\ell}(x)(\Dup{\seq{v}}{j})}\\
%%% \quad \mbox{(if \(x\) is a function variable)} \\
%%%\Dup{(f\ \seq{v})}{j} =
%%%                   \tuple{\underbrace{\lambda\seq{y}.\tuple{},\ldots,\lambda\seq{y}.\tuple{}}_{j-1},\Copy{f}{j}(\Dup{\seq{v}}{j}),\ldots,\Copy{f}{\ell}(\Dup{\seq{v}}{j})}\\
%%%\end{array}
%%%\]
%%%Here, each function parameter has been replaced by a \(\ell\)-tuple of functions.

The generated straightline program satisfies the following properties.
%\koba{Is the following property used in the short version?}
\begin{lemma}
Suppose \(D' = \mkSHP(D,s)\). Then:
\begin{enumerate}
\item \(D'\) contains neither recursions nor non-deterministic branches \(e_1\PAROP e_2\).
\item \(\textit{main}\tuple{}\redswith{s}{D} \FAIL\) if and only if
\({\textit{main}}\tuple{}\redswith{}{D'} \FAIL\).
\item Each function \(\Copy{f_i}{j}\) in \(D'\) is called at most once.
\end{enumerate}
\end{lemma}
%\koba{We will need more properties to prove the progress.}


%\begin{figure}
%\typicallabel{S-Assume}
%\infrule[S-Fun]
%  {f\ \seq{x}=e \in D}
%  {(\psi,E[f\ \seq{v}]) \stackrel{\epsilon}{\RED}_D (\psi,E[[\seq{v}/\seq{x}]e])}
%\rulesp
%
%\infax[S-Let]
%  {(\psi,E[\LETEQIN{x}{v}{e}]) \stackrel{\epsilon}{\RED}_D (\psi,E[[v/x]e])}
%\rulesp
%
%\infrule[S-Op]
%  {x:\mbox{fresh}}
%  {(\psi,E[\OP(\seq{v})]) \stackrel{\epsilon}{\RED}_D (\psi \land x=\OP(\seq{v}),E[x])}
%\rulesp
%
%\infax[S-Par]
%  {(\psi,E[\PAR{e_0}{e_1}]) \stackrel{i}{\RED}_D (\psi,E[e_i])}
%\rulesp
%
%\infrule[S-Assume]
%  {\psi \land v\mbox{~is~satisfiable}}
%  {(\psi,E[\ASSUME{v}{e}]) \stackrel{\epsilon}{\RED}_D (\psi \land v,E[e])}
%\rulesp
%
%\infax[S-Fail]
%  {(\psi,E[\FAIL]) \stackrel{\epsilon}{\RED}_D (\psi,\FAIL)}
%
%\caption{Symbolic Evaluation Rules}
%\label{fig:symb}
%\end{figure}

%%%\subsection{Straightline Higher-Order Program Generation}
%%%If the error path is infeasible, we obtain a straightline program for 
%%%the error path from the original program.
%%%
%%%In this section, we assume that a program conforms to the following 
%%%syntax:
%%%\begin{eqnarray*}
%%%D&::=& \set{f_1\ \seq{x}_1 = e_1,\ldots,f_m\ \seq{x}_m = e_m} \\
%%%e&::=& \ASSUME{v}{d} \mid \PAR{e_1}{e_2} \\
%%%d
%%%&::=& c \mid x\ \seq{v} \mid f\ \seq{v} \mid \LETEQIN{x}{\OP(\seq{v})}{d} \mid \FAIL \\
%%%v&::=& c \mid x \mid f\ \seq{v}
%%%\end{eqnarray*}
%%%The above assumption does not lose generality since we can always obtain 
%%%a program that conforms to the syntax by using the CPS transformation.
%%%
%%%Given a program \((D,e)\), an infeasible error path of \((D,e)\) is of the following form:
%%%\begin{eqnarray*}
%%%&&e \underbrace{\redswith{s_0}{}}_{\rn{E-Par}} \ASSUME{v}{d} \redwith{\epsilon}{}^* f_1\ \seq{v} \underbrace{\red}_{\rn{E-App}} \\
%%%&&\theta_1 e_1 \underbrace{\redswith{s_1}{}}_{\rn{E-Par}} \theta_1 (\ASSUME{v_1}{d_1}) \redwith{\epsilon}{}^* \theta_1 (x_1\ \seq{v}_1) \underbrace{\red}_{\rn{E-App}} \cdots \\
%%%%x_1 may be function name
%%%&&\theta_N e_N \underbrace{\redswith{s_N}{}}_{\rn{E-Par}} \theta_N (\ASSUME{v_N}{d_N}) \redwith{\epsilon}{}^* \FAIL \\
%%%\end{eqnarray*}
%%%Here, \(\theta_i=\set{\seq{x}_i \mapsto \seq{v}_{i-1}}\), \(\FV{e_i} 
%%%\cup \FV{\ASSUME{v_i}{d_i}} \cup \FV{x_i\ \seq{v_i}} \subseteq 
%%%\set{\seq{x_i}}\), and \(\theta_i x_i\) is of the form \(f_{i+1}\ 
%%%\seq{v}_i'\). We write \((D,e) \slice{s_0 \cdots s_N} 
%%%(D',\ASSUME{v}{f_1^{(1)}\ \X{\seq{v}}{1}})\), where \(D'\) is defined by:
%%%\begin{eqnarray*}
%%%D'&=&\set{f_i^{(i)}\ \seq{x}_i=\ASSUME{v_i}{\Y{x_i}{i+1}\ \X{\seq{v}_i}{i+1}}}_{i=1}^{N-1} \cup \\
%%%  & &\set{f_i^{(N)}\ \seq{x}_N=\ASSUME{v_N}{\FAIL}} \\
%%%\Y{x}{i}&=&\NTH{i}{x} \\
%%%\Y{f}{i}&=&f^{(i)} \\
%%%\X{v_1\cdots v_n}{i}&=&\X{v_1}{i}\cdots\X{v_n}{i} \\
%%%\X{c}{i}&=&c \\
%%%\X{x}{i}&=&x \\
%%%\X{f\ \seq{v}}{i}&=&\tuple{f^{(1)}\ \X{\seq{v}}{i},\dots,f^{(N)}\ \X{\seq{v}}{i}}
%%%\end{eqnarray*}
%%%
%%%%\(e \not\redswith{}{D} \FAIL\)
%%%\begin{theorem}
%%%\(e_2 \not\redswith{}{D_2} \FAIL\) if and only if \(s\) is infeasible 
%%%and \((D_1,e_1) \slice{s} (D_2,e_2)\).
%%%\end{theorem}

\subsubsection{Typing SHP}

The next step is to infer dependent types for functions in SHP. Thanks 
to the properties that SHP contains neither recursion nor 
non-deterministic branch and that every function is linear, the standard 
dependent type system is sound and complete for the safety of the 
program. Let us write \(\pDT D\) if \(D\) is typable in the fragment of 
the dependent type system presented in Section~\ref{sec:pred} without 
intersection types (but extended with (non-dependent) tuple types). 
\begin{lemma}
Let \(D' = \mkSHP(D,s)\). Then, \(\pDT D':\Delta\) for some \(\Delta\) 
if and only if \(\textit{main}\tuple{} \not\redswith{}{D'} \FAIL\).
\end{lemma}
\begin{pfsketch}
The ``only if'' part follows immediately from the soundness of the 
dependent type system. For the ``if'' part, it suffices to observe that, 
as every function in \(D'\) is linear, each variable \(x\) of base type 
can be assigned a type \(\rtbase{\nu}{b}{\nu=v}\), where \(v\) is the 
value that \(x\) is bound to.
\end{pfsketch}

We can use existing algorithms~\cite{Unno2009,Terauchi2010} to infer 
dependent types: we first prepare a template of a dependent type for 
each function, generate constraints on predicate variables, and solve 
the constraints. We give below an overview of the dependent type 
inference procedure through an example; an interested reader may wish to 
consult \cite{Unno2009,Terauchi2010}.
%
%%\footnote{Although the dependent type inference techniques of \cite{Unno2009,Terauchi2010} are
%%different (especially in the way recursive functions are handled), 
%%they are essentially the same for straightline programs.}

\begin{example}
\label{ex:dt-inference} Recall the straightline program in 
Example~\ref{ex:shp}. We prepare the following templates of the types of 
functions \(f\), \(h\), \(k\):
\[
\begin{array}{l}
f: (\rtfunb{x}{\rtbase{\nu}{\INT}{P_1(\nu)}} %%\\
%%\qquad 
{\rtfun{(\rtfunb{y}{\rtbase{\nu}{\INT}{P_2(\nu,x)}}{\TUNIT})}{\TUNIT}})\\
h: (\rtfunb{z}{\rtbase{\nu}{\INT}{P_3(\nu)}}
    {\rtfunb{y}{\rtbase{\nu}{\INT}{P_4(\nu,z)}}{\TUNIT}})\\
k: (\rtfunb{x}{\rtbase{\nu}{\INT}{P_0(\nu)}}{\TUNIT})
\end{array}
\]
From the program, we obtain the following constraints on 
\(P_0,\ldots,P_4\):
\[
\begin{array}{l}
P_0(m) \qquad \forall x.(P_1(x) \imply P_2(x+1, x))\\
\forall z,y.(P_3(z) \land P_4(y, z) \imply y>z)\\
%%\forall x.n\geq 0 \imply (P_1(n) \land P_3(n) \land (P_2(x, n) \imply P_4(x, n)))\\
\forall n,y.P_0(n) \imply \\
\qquad(n\geq 0 \imply (P_1(n)\land P_3(n)\land (P_2(y,n)\imply P_4(y,n))))\\
\end{array}
\]
Each constraint has been obtained from the definitions of 
\textit{main},\(f,g\), and \(k\).
%% respectively.
They can be normalized to:
\[
\begin{array}{l}
\forall \nu.(\nu=m \imply P_0(\nu))\\
\forall n,\nu.(P_0(n) \land n\geq 0\land \nu=n \imply P_1(\nu))\\
\forall x,\nu.(P_1(x)\land \nu=x+1 \imply P_2(\nu,x))\\
\forall n,\nu.(P_0(n) \land n\geq 0 \land \nu=n\imply P_3(\nu))\\
\forall n,z,\nu.(P_0(n) \land n\geq 0 \land z=n\land P_2(\nu,n) \imply P_4(\nu,z))\\
\forall z,y.(P_3(z) \land P_4(y, z) \imply y>z)
\end{array}
\]
These constraints are ``acyclic'' in the sense that for each constraint 
of the form \(C_i \imply P_i(\seq{x})\), \(C_i\) contains only (positive) 
occurrences of predicates \(P_j\)'s such that \(j<i\) occur. Such 
constraints can be solved by using a sub-procedure of existing methods 
for dependent type inference based on 
interpolants~\cite{Unno2009,Terauchi2010}, and the following predicates 
can be obtained. (The inferred predicates depend on the underlying 
interpolating theorem prover.)
\[
\begin{array}{l}
P_0(\nu) \equiv %%\TRUE \qquad
P_1(\nu) \equiv P_3(\nu) \equiv \TRUE \\%%\nu\geq 0\\
P_2(\nu,x) \equiv P_4(\nu,x)\equiv \nu>x
\end{array}
\]
Thus, we obtain the following types for \(f\) and \(h\):
\[
\begin{array}{l}
%%f: (\rtfunb{x}{\rtbase{\nu}{\INT}{\nu\geq 0}}\\
f: (\rtfunb{x}{\rtbase{\nu}{\INT}{\TRUE}} %%\\
%%\qquad 
{\rtfun{(\rtfunb{y}{\rtbase{\nu}{\INT}{\nu>x}}{\TUNIT})}{\TUNIT}})\\
%%h: (\rtfunb{z}{\rtbase{\nu}{\INT}{\nu\geq 0}}
h: (\rtfunb{z}{\rtbase{\nu}{\INT}{\TRUE}}
    {\rtfunb{y}{\rtbase{\nu}{\INT}{\nu>z}}{\TUNIT}})\\
\end{array}
\]
\end{example}

%\iffull
%In general, we have:
%\begin{lemma}
%Let \(D\) be a straightline, linear higher-order program that uses only 
%linear arithmetics. Then, there exists an algorithm that decides whether 
%\(\pDT D\) holds, and if so, outputs \(\Delta\) such that \(\pDT 
%D:\Delta\).
%\end{lemma}
%\begin{pfsketch}
%As mentioned above, first prepare templates of dependent types and 
%generate constraints on predicate variables. The constraints can be 
%normalized to the form
%\[
%\begin{array}{l}
% C_1 \imply P_1, C_2[P_1]\imply P_2,\ldots, C_n[P_1,\ldots,P_{n-1}]\imply P_n,\\
%   C_{n+1}[P_1,\ldots,P_n] \imply \psi,
%\end{array}\]
%where all the first-order variables are universally quantified, and 
%\(C_i\) contains only positive occurrences of predicate variables. It 
%suffices to obtain the strongest solution \(P_1^s = C_1, P_2^s = 
%C_2[P_1^s], \ldots, P_n^s = C_n[P_1^s,\ldots,P_{n-1}^s]\) and check 
%whether \(C_{n+1}[P_1^s,\ldots,P_n^s]\imply \psi\) is a valid formula. 
%(In the actual algorithm, we use interpolants instead of the strongest 
%solutions.)
%\end{pfsketch}
%\koba{We need not restrict ourselves to linear arithmetics; all we need is
%the property that the validity of a formula is undecidable.}
%%%we do not even need interpolants if we compute the greatest solution.}
%\fi

%%\subsection{Dependent Type Inference}
\subsubsection{Refining abstraction types}
%%Given a straightline program \((D,e)\), we infer dependent types of 
%%\((D,e)\) by using existing dependent type inference methods based on 
%%interpolation~\cite{Unno2009,Terauchi2010}. These methods are complete 
%%for straightline programs in a sense that given a simply-typed and safe 
%%straightline program \((D,e)\), these methods can infer dependent types 
%%\(\Delta,\delta\) such that \(\DPT{}{D}{\Delta}\) and 
%%\(\DPT{\Delta}{e}{\delta}\).

The final step is to refine the abstraction types of the source program, 
based on the dependent types inferred for the straightline program. Let 
\(\delta_{f,j}\) be the inferred dependent type of \(\Copy{f}{j}\). Then, 
we can obtain an abstraction type \(\sigma_{f,j}\) such that 
\(\UNDUP{\delta_{f,j}} \in \depty(\sigma_{f,j})\) (the choice of such 
\(\sigma_{f,j}\) depends on what predicates are considered atomic), 
where \(\UNDUP{\delta}\) is defined by:
\begin{eqnarray*}
\UNDUP{\rt{\nu}{b}{\psi}}&=&\rt{\nu}{b}{\psi} \\
\UNDUP{x:\delta_1 \to \delta_2}&=&x:\UNDUP{\delta_1} \to \UNDUP{\delta_2} \\
\UNDUP{\delta_1 \times \dots \times \delta_n}&=&\bigwedge_{i\in \set{1,\dots,n}} \UNDUP{\delta_i}
\end{eqnarray*}
The new abstraction type \(\sigma'_f\) of \(f\) is given by: 
\[ \sigma'_f = \sigma_f \itlub \sigma_{f,1} \itlub \cdots \itlub \sigma_{f,\ell},\]
where \(\sigma_f\) is the previous abstraction type of \(f\) and 
\(\sigma_1\itlub \sigma_2\) is obtained by just merging the corresponding predicates:
\[
\begin{array}{l}
b[\seq{P}] \itlub b[\seq{Q}] = b[\seq{P},\seq{Q}]\\
(x\COL\sigma_1 \to \sigma_2)\itlub(x\COL\sigma'_1 \to \sigma'_2)
  = x\COL(\sigma_1\itlub\sigma'_1) \to (\sigma_2\itlub\sigma'_2)
\end{array}
\]

We write \(\RefAT(\Gamma,\Delta)\) for the refined abstraction type 
environment \(f_1\COL \sigma'_{f_1},\ldots,f_n\COL \sigma'_{f_n}\). 
(There is a non-determinism coming from the choice of \(\sigma_{f,j}\), 
but that does not matter below.)

\begin{example}
\label{ex:it-inference} Recall Example~\ref{ex:it-inference}. From the 
dependent types of \(f\) and \(g\), we obtain the following abstraction 
types:
\[
\begin{array}{l}
%%f: (x\COL\INT[\lambda \nu.\nu\geq 0] \ra (y\COL \INT[\lambda \nu.\nu>x]\ra \TUNIT)\ra \TUNIT)\\
%%h: (z\COL\INT[\lambda \nu.\nu\geq 0]\ra y\COL \INT[\lambda \nu.\nu>z] \ra \TUNIT)
f: (x\COL\INT[\,] \ra (y\COL \INT[\lambda \nu.\nu>x]\ra \TUNIT)\ra \TUNIT)\\
h: (z\COL\INT[\,]\ra y\COL \INT[\lambda \nu.\nu>z] \ra \TUNIT)
\end{array}
\]
Suppose that the previous abstraction types were
\[
\begin{array}{l}
f: (x\COL\INT[\,] \ra (y\COL \INT[\,]\ra \TUNIT)\ra \TUNIT)\\
h: (z\COL\INT[\lambda \nu.\nu= 0]\ra y\COL \INT[\lambda \nu.\nu>0] \ra \TUNIT)
\end{array}
\]
Then, the refined abstraction types are:
\[
\begin{array}{l}
f: (x\COL\INT[\,] \ra (y\COL \INT[\lambda \nu.\nu>x]\ra \TUNIT)\ra \TUNIT)\\
h: (z\COL\INT[\lambda \nu.\nu=0]\ra y\COL \INT[\lambda \nu.\nu>0, \lambda \nu.\nu>z] \ra \TUNIT)
\end{array}
\]

\end{example}
%%%\koba{Rewrite below}
%%%We can extract new predicates used for abstraction of the original 
%%%program from the dependent types \(\Delta\) of the straightline program 
%%%\((D,e)\). We can show that the new predicates obtained from a 
%%%straightline program are sufficient for our predicate abstraction to 
%%%reject the infeasible error path corresponding to the straightline 
%%%program.


\subsection{Properties of the CEGAR algorithm}
We now discuss properties of the overall CEGAR algorithm.
%
If the refined abstraction type is obtained from an infeasible error 
path \(s\), the new abstract boolean program no longer has the path \(s\). 
This is the so called ``progress property'' known in the literature on 
CEGAR for the usual (i.e. finite state or pushdown) model checking. 
%
%\koba{Below (and also in several places above), for the sake of 
%simplicity, I have fixed the main expression to \textit{main}. If we do 
%not adapt the simplification, it would be better to prepare a shorthand 
%form for \(D:\Delta\land \Delta\p e:\delta\), \(\T{}{D}{\Gamma}{D'}\land 
%\T{\Gamma}{e}{\sigma}{e'}\), etc.}
%
Formally, we can prove the following 
property 
\iffull 
(see Appendix~\ref{sec:progress} for the proof):
\else
(see \cite{Kobayashi2011} for the proof):
\fi
\begin{theorem}[progress]
\label{thm:progress}
Let \(D_1\) be a well-typed program and \(s\) be an infeasible path of \(D_1\). 
Suppose 
 \(D_2 = \mkSHP(D_1,s)\) and 
 \(\DPT{}{D_2}{\Delta}\) with \(\Gamma = \RefAT(\Gamma',\Delta)\) for some \(\Gamma'\).
%%\end{itemize}
%%\begin{itemize}
%%\item \(D_2 = \mkSHP(D_1,s)\),
%%\item \(\DPT{}{D_2}{\Delta}\), and 
%%\item \(\Gamma = \RefAT(\Gamma',\Delta)\) for some \(\Gamma'\).
%%\end{itemize}
Then, there exists \(D_3\) such that
 \(\T{}{D_1}{\Gamma}{D_3}\), and
 \(\textit{main}\tuple{} \not\redswith{s}{D_3} \FAIL\).
%%\begin{itemize}
%%\item \(\T{}{D_1}{\Gamma}{D_3}\), and
%%\item \(\textit{main}\tuple{} \not\redswith{s}{D_3} \FAIL\).
%%\end{itemize}
%%%Suppose that
%%%\begin{itemize}
%%%\item \((D_1,e_1) \slice{s} (D_2,e_2)\),
%%%\item \(\DPT{}{D_2}{\Delta}\), and
%%%\item \(\DPT{\Delta}{e_2}{\delta}\).
%%%\end{itemize}
%%%%For any \(\seq{c}\) such that \(\Gamma \pS \seq{c} \COL \seq{b}\)
%%%If \(\Delta \in \depty(\Gamma)\) and \(\delta \in \depty(\sigma)\), 
%%%there exist \(D_3,e_3\) such that:
%%%\begin{itemize}
%%%\item \(\T{}{D_1}{\Gamma}{D_3}\), 
%%%\item \(\T{\Gamma}{e_1}{\sigma}{e_3}\), and 
%%%\item \(e_3 \not\redswith{s}{D_3} \FAIL\).
%%%\end{itemize}
\end{theorem}
%%%\begin{proof}
%%%First, by Theorem~\ref{th:rel-complete}, 
%%%we have \(D_4\) such that
%%%\(\T{}{D_2}{\Gamma}{D_4}\) and 
%%%\(\textit{main}\tuple{} \not\redswith{}{D_4} \FAIL\).
%%%%%First, show that there exist \(D_4,e_4\) such that:
%%%%%\begin{itemize}
%%%%%\item \(\T{}{D_2}{\Gamma}{D_4}\), 
%%%%%\item \(\T{\Gamma}{e_2}{\sigma}{e_4}\), and 
%%%%%\item \(e_4 \not\redswith{s}{D_4} \FAIL\).
%%%%%\end{itemize}
%%%We can then construct \(D_3\) from both \(D_1\) and \(D_4\).
%%%\end{proof}

The progress property above does not guarantee that the verification 
will eventually terminate: There is a case where the entire CEGAR loop 
does not terminate, finding new spurious error paths forever (see 
Section~\ref{sec:experiment}). Indeed, we cannot expect to obtain a 
sound and complete verification algorithm, as the reachability is 
undecidable in general even if programs are restricted to those using 
only linear arithmetics.

We can however modify our algorithm so that it is \emph{relatively 
complete} with respect to the dependent intersection type system, in the 
sense that all the programs typable in the dependent intersection type 
system can be verified by our method. Let \(\textit{genP}\) be a total 
map from the set of integers to the set of predicates. (Such a total map 
exists, as the set of predicates is recursively enumerable.) Upon the 
\(i\)-th iteration of the CEGAR loop, add the predicate 
\(\textit{genP}(i)\) to each position of abstraction type, in addition 
to the predicates inferred from counterexamples. Then, if a program is 
well-typed under \(\Delta\) in the dependent intersection type system, 
an abstraction type environment \(\Gamma\) such that \(\Delta\in 
\depty(\Gamma)\) is eventually found, so that by 
Theorem~\ref{th:rel-complete}, our verification succeeds. Of course, 
this is impractical, but we may be able to adapt the technique of 
\cite{Jhala2006} to get a practical algorithm.
%%%%%We can, however, show that if a program is well-typed in the dependent intersection type system,
%%%we can modify the algorithm so that it can eventually accept the p
%%%a modification of our verification method can eventually succeed.
%%%For that purpose, for each cegar iteration, ...\koba{Rewrite below}
%%%Furthermore, we believe we can extend our method so that it can 
%%%eventually accept the program if a program is typable under the 
%%%dependent intersection type system.
%%%%
%%%For that purpose, we need to use a dependent type inference method that 
%%%can eventually discover every predicate (note that the set of predicates 
%%%is recursively enumerable). Construction of such a dependent type 
%%%inference method is still unknown, 
%%%however, there exists a pioneering 
%%%work on such a predicate discovery by Jhala and McMillan in the context 
%%%of finite-state model checking, which we believe can be adopted to our 
%%%method.
%%%\koba{The construction of such a method is
%%%trivial, although whether an \emph{efficient} method exists is unclear.
%%%So, remove ``we believe'' and rephrase the succeeding statements.}


%%%\begin{remark}
%%%\label{rem:incompleteness}
%%%Our verification method does not terminate for the program given in Remark~\ref{rem:limitation}.
%%%The program is obviously safe (i.e., the assertion always succeeds)
%%%but the program is not typable in the dependent type system in Section~\ref{sec:pred}.
%%%To type the program, we need to assign to \texttt{apply}
%%%a type like \(\forall n.(({\rtbase{\nu}{\INT}{\nu=n}}\ra{\TUNIT})\ra
%%%           {\rtbase{\nu}{\INT}{\nu=n}}\ra{\TUNIT})\),
%%%which is not allowed in the dependent type system in Section~\ref{sec:pred}.
%%%%%%(Note that the program would be typable if the arguments of \texttt{apply} were swapped.)
%%%Thus, by Theorem~\ref{th:rel-soundness}, the program cannot be verified by our verification method either.
%%%In fact, our verification method would generate 
%%%abstraction types 
%%%\[ (\TINT[\lambda \nu.\nu=n]\ra \TUNIT)\ra \TINT[\lambda \nu.\nu=n] \ra \TUNIT\]
%%%for \(n=0,1,2,\ldots\) and diverge.
%%%As mentioned in Remark~\ref{rem:limitation}, the program can be verified if we swap
%%%the parameter of \textit{apply} or we add an additional parameter to \textit{apply}.
%%%\end{remark}

%function coercing�������
%
%function subtyping
%
%denotation semantics���g���Ă����̂�syntactic�ɂ����@
%
%simulation�֌W���`

