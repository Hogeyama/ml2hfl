\section{Proof of Theorem~\ref{th:rel-complete}}
\label{sec:relcomp}

In this section, we consider the normalized dependent typing rules in Figure~\ref{fig:dep_rules2},
which are equivalent to those in Figure~\ref{fig:dep_rules}.
%%%Below we assume that the rule \rn{A-Fail} has been extended to:
%%%
%%%\infrule[A-Fail']
%%%  {\models \theta_{\Gamma}\psi}
%%%  {\T{\Gamma}{\FAIL}{\sigma}{\ASSUME{\psi}{\FAIL}}}

\begin{figure}[tbp]
%%\koba{It seems we need an empty intersection \(\top = \bigwedge_{\set{i\in \emptyset}} \delta_i \).
%%Consider \texttt{f\ g = fail}, where the simple type of \(f\) is \((\TUNIT\ra \TUNIT)\ra \TUNIT\).
%%Then, \(f\) is only given a type \(\top\).}

\typicallabel{D-Assume}

%%%\infrule[D-\(\land\)-Intro]
%%%  {\DPT{\Delta}{e}{\delta_1} \andalso
%%%   \DPT{\Delta}{e}{\delta_2}}
%%%  {\DPT{\Delta}{e}{\delta_1 \land \delta_2}}
%%%\rulesp
%%%
%%%\infrule[D-\(\land\)-Elim]
%%%  {\DPT{\Delta}{e}{\delta_1 \land \delta_2} \andalso
%%%   i \in \set{1,2}}
%%%  {\DPT{\Delta}{e}{\delta_i}}
%%%\rulesp

\infrule[D-Base]
  {\mbox{$e$ is a constant, a variable or an expression of the form \(\OP(\seq{v})\)}\\
    \alphaB(\Delta) \pS e:b}
  {\DPT{\Delta}{e}{\rt{\nu}{b}{\nu=e}}}
\rulesp


%%%\infrule[D-VarFun]
%%%  {x:\delta \in \Delta\andalso \mbox{$\delta$ is a function type}}
%%%  {\DPT{\Delta}{x}{\delta}}
%%%\rulesp
%%%
%%%\infrule[D-App]
%%%  {\DPT{\Delta}{e}{\dtI_{j\in\set{1,\ldots,m}} {(y\COL\delta_{j,0}\ra \delta_{j,1})}}\\
%%%   \DPT{\Delta}{v}{\dtI_{j\in\set{1,\ldots,m}} \delta_{j,0}}}
%%%  {\DPT{\Delta}{e\,v}{\dtI_{j\in\set{1,\ldots,m}} [v/y]\delta_{j,1}}}

\infrule[D-App]
  {\Delta(x)= \dtI_{j\in\set{1,\ldots,m}}y_1\COL\delta_{j,1} \to \cdots \to y_k\COL\delta_{j,k} \to\delta_j\\
   \mbox{\(\Delta(x)\) is a function type}\\
   \DPT{\Delta} %%,y_1:\delta_{j,1},\dots,y_{i-1}: \delta_{j,i-1}}
  {v_i}{\dtI_{j\in\set{1,\ldots,m}}[v_1/y_1,\ldots,v_{i-1}/y_{i-1}]\delta_{j,i}} \\
%%   \delta'_{j,i} \subT [v_1/y_1,\ldots,v_{i-1}/y_{i-1}]\delta_{j,i}\\
  \mbox{ (for each $i\in\set{1,\ldots,k}$, $j\in\set{1,\ldots,m}$)}}
  {\DPT{\Delta}{x\ \seq{v}}{\dtI_{j\in\set{1,\ldots,m}} [\seq{v}/\seq{y}]\delta_j}}
\rulesp

\infrule[D-Let]
  {\DPT{\Delta}{e_1}{\delta'} \andalso
   \DPT{\Delta,x:\delta'}{e_2}{\delta}}
  {\DPT{\Delta}{\LETEQIN{x}{e_1}{e_2}}{\delta}}
\rulesp


\infrule[D-Fail]
  {\models \SEM{\Delta} \IMPLY \FALSE}
  {\DPT{\Delta}{\FAIL}{\delta}}
\rulesp

\infrule[D-Assume]
  {\alphaB(\Delta)\pS v:\TBOOL\\
   \DPT{\Delta,x\COL\rtbase{\nu}{\TBOOL}{v}}{e}{\delta}}
  {\DPT{\Delta}{\ASSUME{v}{e}}{\delta}}

%%\infrule[D-Assume]
%%  {\DPT{\Delta}{v}{\TBOOL} \andalso
%%   \DPT{\Delta,v}{e}{\delta}}
%%  {\DPT{\Delta}{\ASSUME{v}{e}}{\delta}}
\rulesp

\infrule[D-Par]
  {\DPT{\Delta}{e_1}{\delta} \andalso
   \DPT{\Delta}{e_2}{\delta}}
  {\DPT{\Delta}{\PAR{e_1}{e_2}}{\delta}}
\rulesp

%%%\infrule[D-Coerce]
%%%  {\DPT{\Delta}{e}{\rt{\nu}{b}{\psi'}} \andalso
%%%   \models \SEM{\Delta} \land \psi' \IMPLY \psi}
%%%  {\DPT{\Delta}{e}{\rt{\nu}{b}{\psi}}}
%%%
%%%\koba{How about subtyping on functions?}

\infrule[D-Coerce]
  {\DPT{\Delta}{e}{\delta'}\andalso \Delta \p \delta'\subT \delta}
  {\DPT{\Delta}{e}{\delta}}
\infrule[D-Prog]
  {f_i: (\seq{x}_i:\seq{\delta}_i \to \delta_i) \in \Delta \andalso
   \DPT{\Delta,\seq{x}_i:\seq{\delta}_i}{e_i}{\delta_i} \\
   (\mbox{for~each~}i=1,\dots,n)}
  {\DPT{}{\set{f_1\ \seq{x}_1 = e_1,\ldots,f_m\ \seq{x}_m = e_m}}{\Delta}}
\rulesp

\infrule[Sub-Base]
  {\models \SEM{\Delta} \land \psi \IMPLY \psi'}
  {\DPSUB{\Delta}{\rt{\nu}{b}{\psi}\subT \rt{\nu}{b}{\psi'}}}
\rulesp
\infrule[Sub-Fun]
  {\DPSUB{\Delta}{\delta_{i,1}\subT \delta_{i,1'}} \mbox{ (for each $i\in S$)}\\
   \DPSUB{\Delta, x\COL\delta_{i,1}}{\delta_{i,2}'\subT\delta_{i,2}} \mbox{ (for each $i\in S$)}\\
    S \subseteq S'}
%%   \DPSUB{\Delta,x\COL \delta_1}{\delta_2'\subT \delta_2}}
  {\DPSUB{\Delta}{\dtI_{i\in S'} (x\COL \delta'_{i,1}\ra \delta'_{i,2}) 
                    \subT\dtI_{i\in S} (\delta_{i,1}\ra \delta_{i,2})}}
%%  {\DPSUB{\Delta}{(x\COL\delta'_1\to \delta'_2)\subT (x\COL\delta_1\to \delta_2)}}

%%\infrule[Sub-IT]
%%  {\forall i\in\set{1,\ldots,m}.\exists i'\in\set{1,\ldots,m'}.\DPSUB{\Delta}{\delta'_{i'}\subT \delta_{i}}}
\caption{Normalized Dependent Intersection Type System}
\label{fig:dep_rules2}
\end{figure}

%%\begin{pfsketch}
%%\koba{The following is just a semantic intuition.}

%%%We first define the semantics of dependent types by:
%%%\[
%%%\begin{array}{l}
%%%\dtsem{\rtbase{\dvar}{b}{\psi}} = \set{\dvar \in \sem{b} \mid \psi}\\
%%%\dtsem{x\COL\delta_1\ra \delta_2} =
%%%  \left\{\begin{array}{ll}
%%%     \set{g \mid \forall x\in \dtsem{\delta_1}.g(x)\subseteq \dtsem{\delta_2}}\\
%%%    \qquad \mbox{if $\delta_2$ is a base type}\\
%%%     \set{g \mid \forall x\in \dtsem{\delta_1}.g(x)\in \dtsem{\delta_2}}\\
%%%    \qquad \mbox{otherwise}
%%%  \end{array}
%%%  \right.
%%%\end{array}
%%%\]
%%%Note that a function in \(x\rtbase{\dvar}{\TINT}{\psi_1}\ra \rtbase{\dvar}{\TINT}{\psi_2}\)
%%%returns a \emph{set} of integers, because of the non-determinism in our language.
%%%The non-determinism is however restricted to base types, so that 
%%%if \(\delta_2\) is a function type, the codomain of \(g(x)\) is \(\dtsem{\delta_2}\),
%%%rather than the powerset of \(\dtsem{\delta_2}\).
%%%
%%%The semantics of dependent type environments is given by:
%%%\[
%%%\begin{array}{l}
%%%\dtsem{\Delta} = \set{\emptyset}\\
%%%\dtsem{\Delta,x\COL \delta} =
%%%    \set{\rho\set{x\mapsto v} \mid \rho\in \Delta, v\in \dtsem{\rho\delta}}
%%%\end{array}
%%%\]
%%%\(\rho\delta\) is the dependent type obtained by applying the substituion \(\rho\) to \(\delta\)
%%%(this is necessary because of the dependency of types on values).

%%The semantic domain \(\sem{\sigma}\) is defined by:
%%\[
%%\begin{array}{l}
%%\sem{b[P_1,\ldots,P_n]} = \PSet{(\TBOOL\times \cdots \times \TBOOL)\cup\set{\FAIL}}\\
%%x\COL b[P_1,\ldots,P_n] \ra \sigma_2 = 
%%  \TBOOL\times \cdots \times \TBOOL \ra \sem{\sigma_2}\\
%%\sem{\sigma_1\ra \sigma_2} = 
%%    \sem{\sigma_1} \ra \sem{\sigma_2}\\
%%\end{array}
%%\]

We first define the set of \emph{type templates}, ranged over by \(\xi\), by:
\[
\begin{array}{l}
\xi ::= \set{\nu\COL b \mid C} \mid \dtI_{j\in S} (x\COL\xi_1\ra \xi_2)\\
C ::= \hole_i \mid C_1\land C_2 \mid C_1\lor C_2 \mid \neg C_1 \mid \TRUE \mid \FALSE
\end{array}
\]
We write \(C[\psi_1,\ldots,\psi_k]\) for the formula obtained from \(C\) by replacing 
every occurence of \(\hole_i\)  (where \(i\in\set{1,\ldots,k}\)) with \(\psi_i\).
A \emph{type template environment} is a sequence \(x_1\COL\xi_1,\ldots,x_n\COL\xi_n\)
of bindings of variables to type templates. Type templates and type template environments
are subject to the well-formedness condition analogous to that of dependent types
and dependent type environments. 

For a type template \(\xi\) and an abstraction type \(\sigma\), 
the dependent type \(\comp{\xi}{\sigma}\) is defined by:
\[
\begin{array}{l}
\comp{\set{\nu\COL b \mid C}}{b[P_1,\ldots,P_n]} = \set{\nu\COL b \mid C[P_1(\nu),\ldots,P_n(\nu)]}\\
\comp{\dtI_{i\in S}(x\COL\xi_{1,i}\ra \xi_{2,i})}{x\COL\sigma_1\ra \sigma_2} =\\
\qquad\qquad    \dtI_{i\in S}(x\COL \comp{\xi_{1,i}}{\sigma_1} \ra \comp{\xi_{2,i}}{\sigma_2})
\end{array}
\]
For example, let \(\xi\) and \(\sigma\) be:
\[
\begin{array}{l}
\xi = (x\COL\set{\nu\COL \TINT\mid \neg \hole_2}\ra \set{\nu\COL \TINT\mid \neg\hole_1})\\
\qquad  \land (x\COL\set{\nu\COL \TINT\mid \hole_1\land \hole_2}\ra \set{\nu\COL \TINT\mid \hole_1})\\
\sigma = x\COL\TINT[\lambda \nu.\nu>0, \lambda \nu.\nu<3]\ra \TINT[\lambda \nu.\nu>x]
\end{array}
\]
Then, \(\comp{\xi}{\sigma}\) is the following dependent type:
\[
\begin{array}{l}
(x\COL\set{\nu\COL \TINT\mid \neg \nu<3}\ra \set{\nu\COL \TINT\mid \neg(\nu>x)})\\
  \land (x\COL\set{\nu\COL \TINT\mid \nu>0\land \nu<3}\ra \set{\nu\COL \TINT\mid \nu>x})
\end{array}
\]

Similarly, given a type template environment \(\Xi\) and an abstraction type environment \(\Gamma\),
the dependent type environment \(\comp{\Xi}{\Gamma}\) is defined by:
\[
\begin{array}{l}
\comp{\emptyset}{\emptyset} = \emptyset\\
\comp{(\Xi,x\COL\xi)}{\Gamma,x\COL\sigma} = \comp{\Xi}{\Gamma}, x\COL \comp{\xi}{\sigma}
\end{array}
\]

By the definition of type templates and \(\depty\), 
\(\delta\in \depty(\sigma)\) if and only if \(\delta = \comp{\xi}{\sigma}\) for some \(\xi\).
Similarly, \(\Delta\in \depty(\Gamma)\) if and only if
\(\Delta = \comp{\Xi}{\Gamma}\) for some \(\Xi\).
%%We write \(\tytemp{\sigma}{\delta}\) for \(\kappa\) such that 

For a type template \(\xi\),
we define \(\sem{\xi}\) by:
\[
\begin{array}{l}
\sem{\rtbase{\dvar}{b}{C}}=\\
\qquad  \set{\tuple{c_1,\ldots,c_n} \mid C[c_1,\ldots,c_n]=\TRUE}\\
%%  \set{(d, \tuple{c_1,\ldots,c_n}) \mid [d/\dvar]\psi \land  C[c_1,\ldots,c_n]}\\
%%\qquad \PSet{\set{\tuple{c_1,\ldots,c_n} \mid C[c_1,\ldots,c_n] \equiv \TRUE}}\\
\sem{\dtI_{j\in S}(\rtfun{x\COL\xi_{1,j}}{\xi_{2,j}})} = \\ %%, x\COL\sigma_1 \to \sigma_2} =\\
\qquad \set{g \mid \forall j\in S.\forall h\in \sem{\xi_{1,j}}.g(h) \subseteq \sem{\xi_{2,j}}}
\end{array}
\]
Intuitively, \(\sem{\xi}\) denotes the set of abstract (semantic) values obtained by 
abstracting values of a type \(\delta\) by using an abstraction type \(\sigma\) such that \(\comp{\xi}{\sigma}=\delta\).

The above semantics is extended to type template environments by:
\[
\begin{array}{lcl}
\sem{\emptyset} &=& \emptyset\\
\sem{\Xi, x\COL\xi} &=&
  \set{\rho\set{x\mapsto v} \mid \rho\in \sem{\Xi}, v\in \sem{\xi}}
\end{array}
\]
We call an element of \(\sem{\Xi}\) an \emph{abstract environment} and write \(\env\) for it.

%%The semantics \(\esem{e}{\st}\) of of a term \(e\) of type \(\st\) is given by:
%%\[
%%\begin{array}{l}
%%\esem{e}{\TBOOL\times\cdots\times \TBOOL} = 
%%      \set{ v \mid e \reds{} v} \cup \set{\FAIL \mid e \reds{} E[\FAIL]}\\
%%\esem{e}{\st_1\ra \st_2} = 
%%      \set{(\esem{e_1}{\st_1}, \esem{e e_1}{\st_2}) \mid e_1:\st_1}\\
%%\end{array}
%%\]

Define the set of simple types (for boolean programs) by:
\[ \st ::= \TBOOL\times\cdots \times \TBOOL \mid \st_1 \ra \st_2\]
The semantics \(\set{\st}\) is given by:
\[
\begin{array}{l}
\sem{\underbrace{\TBOOL\times \cdots \times \TBOOL}_n} =
\set{ \tuple{c_1,\ldots,c_n}\mid c_i\in\set{\TRUE,\FALSE}}\cup \set{\FAIL}\\
\sem{\st_1\ra \st_2} =
  \set{g \mid \forall h\in\sem{\st_1}\setminus\set{\FAIL}.g(h)\subseteq \sem{\st_2}}\cup\set{\FAIL}
\end{array}
\]
(Note that the codomain is the powerset of \(\sem{\st_2}\), because of non-determinism.)

Given an abstract environment \(\env\) and an expression \(e\) (of a higher-order boolean program)
of simple type \(\st\), 
the semantics \(\esem{e}{\env} (\subseteq \sem{\st})\) of a term \(e\) is given by:

\[
\begin{array}{l}
\esem{c}{\env} = \set{c}\\
\esem{x}{\env} = \set{\env(x)}\\
\esem{\OP(\seq{v})}{\env} = \set{\sem{\OP}(\seq{w}) \mid \seq{w}\in \esem{\seq{v}}{\env}}\\
\esem{\lambda x:\st.e}{\env} = \\
\quad      \{\set{(v,\esem{e}{\env\set{{x}\mapsto {v}}}) \mid {v}\in \sem{\st}\setminus\set{\FAIL}}
          \cup \set{(\FAIL,\set{\FAIL})}\}\\
\esem{e_1 e_2}{\env} =\\
\quad     \set{g(h) \mid g\in \esem{e_1}{\env}\setminus\set{\FAIL}, h\in \esem{e_2}{\env}}
    \cup (\esem{e_1}{\env}\cap \set{\FAIL})\\
\esem{\LETEQIN{x}{e_1}{e_2}}{\env} =\\
\quad      \set{\esem{e_2}{\env\set{x\mapsto v}} \mid v\in \esem{e_1}{\env}\setminus\set{\FAIL}}
     \cup (\esem{e_1}{\env}\cap \set{\FAIL})\\
\esem{\FAIL}{\env} = \set{\FAIL}\\
\esem{\ASSUME{v}{e}}{\env} = 
  \left\{\begin{array}{ll}
       \emptyset  & \mbox{if \(\TRUE\not\in \esem{v}{\env}\)}\\
      \esem{e}{\env} & \mbox{otherwise}
     \end{array}
  \right.\\
\esem{e_1\PAROP e_2}{\env} =
      \esem{e_1}{\env} \cup \esem{e_2}{\env}\\
\end{array}
\]


The following lemma states that we can adjust abstraction according to a change of abstraction types.
\begin{lemma}
\label{lem:relcomp-coerce}
Suppose:
\begin{enumerate}
\item \(\T{\Gamma}{e}{\sigma'}{e_0'}\); and 
\item \(\forall j\in J.\comp{\Xi_j}{\Gamma} \pDT \comp{\xi'_j}{\sigma'} \subT \comp{\xi_j}{\sigma}\),
\end{enumerate}
where \(J\) is a finite set.
Then, there exists \(e'\) such that
\begin{enumerate}
\item \(\T{\Gamma}{e}{\sigma}{e'}\); and 
\item \(\forall j\in J.\forall \env\in \sem{\Xi_j}.
(\esem{e_0'}{\env}\subseteq \sem{\xi_j'}\imply \esem{e'}{\env}\subseteq \sem{\xi_j})\).
\end{enumerate}
\end{lemma}
\begin{proof}
This follows by induction on the size of \(\alphaB(\sigma)\).
%%(Here, the size of types is defined by:
%%\(\mathit{}(b) =1\) and \(\mathit{order}(\tau_1\ra\tau_2) =
%%  \MAX(\mathit{order}(\tau_1)+1, \mathit{order}
%%the order of a base type is \(0\), and the order of \(\tau_1\ra \tau_2\) is the maximum of

%%\(\comp{\Xi}{\Gamma} \pDT \comp{\xi'}{\sigma'} \subT \comp{\xi}{\sigma}\), with case
%%analysis on the last rule used.
\begin{itemize}
\item Case \rn{Sub-Base}:\\
In this case, we have:
\[
\begin{array}{l}
\xi_j' \equiv \rtbase{\nu}{b}{C_j'}\\
\xi_j \equiv  \rtbase{\nu}{b}{C_j}\\
\sigma' \equiv b[P'_1,\ldots,P'_{k'}]\\
\sigma \equiv b[P_1,\ldots,P_k]\\
\SEM{\comp{\Xi}{\Gamma}}\land C'[P'_1(\nu),\ldots,P'_{k'(\nu)}] \imply C[P_1(\nu),\ldots,P_k(\nu)]
\end{array}
\]
%%Let \(\sigma' = b[Q_1,\ldots,Q_{k'}]\).
%%By the induction hypothesis, we have:
%%\[
%%\begin{array}{l}
%%\T{\Gamma}{e}{\rtbase{\nu}{b}{C_1[P_1(\nu),\ldots,P_k(\nu)]}}{e_1'}\\
%%\esem{e_1'}{\rho}\in\sem{\delta', \sigma'}
%%\end{array}
%%\]
By the last condition, we have:
\[
\begin{array}{l}
z_i \COL \set{\nu:b_i \mid D_{j,i}}\in \Xi_j \mbox{ (for each $i\in \set{1,\ldots,m}$)}\\
z_i \COL  b_i[Q_{i,1},\ldots,Q_{i,k_i}] \in \Gamma \mbox{ (for each $i\in \set{1,\ldots,m}$)}\\
\models D_{j,1}[Q_{1,1}(z_1),\ldots,Q_{1,k_1}(z_1)]\land \cdots\\
\quad\land D_{j,m}[Q_{m,1}(z_m),\ldots,Q_{m,k_m}(z_m)]\\
\quad\land C_j'[P'_1(\nu),\ldots,P'_{k'}(\nu)] \imply  C_j[P_1(\nu),\ldots,P_k(\nu)]\\
\qquad\qquad \hfill\mbox{ (for each $j\in J$)}
\end{array}
\]

Let \(\psi\) and \(\psi'\) be:
\[
\begin{array}{l}
\psi \equiv 
  \bigwedge_{j\in J}(D_{j,1}[z_1]\land \cdots\land D_{j,m}[z_m]\land C_j'[v_0]\imply C_j[v_1])\\
\psi' \equiv 
 \bigwedge_{j\in J}( D_{j,1}[z_1]\land \cdots\land D_{j,m}[z_m] \land C_j'[v_0]\imply C_j[v_2])\\
v_0 = \tuple{\NTH{1}(y),\ldots,\NTH{k'}(y)}\\
v_1 = \tuple{\NTH{k'+1}(y),\ldots,\NTH{k'+k-1}(y),\TRUE}\\
v_2 = \tuple{\NTH{k'+1}(y),\ldots,\NTH{k'+k-1}(y),\FALSE}\\
\end{array}
\]

%%%Let \(D[u_1,\ldots,u_m, x',x]\) be a boolean formula such that
%%%\[
%%%\begin{array}{l}
%%%\exists z_1,\ldots,z_m,z.
%%% u_1 = \tuple{Q_{1,1}(z_1),\ldots,Q_{1,k_1}(z_1)}\land \cdots\\
%%% \land u_m = \tuple{Q_{m,1}(z_m),\ldots,Q_{m,k_m}(z_m)}\\
%%% \land x'=\tuple{P'_1(z),\ldots,P'_{k'}(z)}\\
%%% \land x=\tuple{P_1(z),\ldots,P_{k}(z)}
%%%\end{array}
%%%\]
%%%
%%%Then, we have:
%%%\[
%%%\models D[u_1,\ldots,u_m,x,y] \land D_1[u_1]\land \cdots\land D_m[u_m]\land C'[x]\imply C[y].
%%%\]
%%%
%%%Let \(\psi\) and \(\psi'\) be:
%%%\[
%%%\begin{array}{l}
%%%\psi \equiv 
%%%D[u_1,\ldots,u_m,v_0,v_1] \\
%%%\quad\land (D_1[u_1]\land \cdots\land D_m[u_m]\land C'[v_0]\imply C[v_1])\\
%%%\psi' \equiv 
%%%D[u_1,\ldots,u_m,v_0,v_2] \\
%%%\quad\land (D_1[u_1]\land \cdots\land D_m[u_m]
%%%\land C'[v_0]\imply C[v_2])\\
%%%v_0 = \tuple{\NTH{1}(y),\ldots,\NTH{k'}(y)}\\
%%%v_1 = \tuple{\NTH{k'+1}(y),\ldots,\NTH{k'+k-1}(y),\TRUE}\\
%%%v_2 = \tuple{\NTH{k'+1}(y),\ldots,\NTH{k'+k-1}(y),\FALSE}\\
%%%\end{array}
%%%\]
Then, we have
\[
\begin{array}{l}
\models P_k(y) \imply \THE{\Gamma,y\COL b[P_1',\ldots,P'_{k'},P_1,\ldots,P_{k-1}]}\psi\\
\models \neg P_k(y) \imply \THE{\Gamma,y\COL b[P_1',\ldots,P'_{k'},P_1,\ldots,P_{k-1}]}\psi'\\
\end{array}
\]
Let \(e'\) be:
\[
\begin{array}{l}
\LETEQIN{\tuple{\seq{u}}}{e_0'}\\
\LETEQIN{\tuple{y_1,\ldots,y_{k'+k-1}}}{\tuple{\seq{u},\underbrace{\nondet,\ldots,\nondet}_{k-1}}}\\
\LETEQIN{y_{k'+k}}{\PARB{(\ASSUME{\psi}{\TRUE})}{(\ASSUME{\psi'}{\FALSE})}}\\
  {\tuple{y_{k'+1},\ldots,y_{k'+k-1},y_{k'+k}}}
\end{array}
\]
(where \(\nondet = \PARB{\TRUE}{\FALSE}\)).
Then, we have
\(\T{\Gamma}{e}{\sigma}{e'}\), modulo some simplification of the output of the translation.
(Here, the expression is complex because of the rules \rn{A-CoerceAdd} and \rn{A-CoerceRem},
but the idea is simple:
\(e'\) first assigns non-deterministic booleans for the values of \(P_1,\ldots,P_{k-1}\),
and then filter out invalid assignments when the value of \(P_k\) is computed.)
Suppose \(\rho\in \sem{\Xi_j}\) and \(\esem{e_0'}{\rho}\subseteq \sem{\xi_j'}\).
By the definition of \(e'\),
\(\tuple{c_1,\ldots,c_k}\in \esem{e'}{\rho}\) implies 
\(D_{j,1}[\rho(z_1)]\land \cdots \land D_{j,m}[\rho(z_m)]\land C_j'[c'_1,\ldots,c'_{k'}]\imply C_j[c_1,\ldots,c_k]\)
for some \(\tuple{c_1',\ldots,c'_{k'}}\in \esem{e_0'}{\rho}\).
By the condition \(\env\in\sem{\Xi}\) and \(\esem{e_1'}{\rho}\subseteq \sem{\xi_j'}\),
we have \(D_{j,1}[\rho(z_1)]\land \cdots \land D_{j,m}[\rho(z_m)]\land C_j'[c'_1,\ldots,c'_k]=\TRUE\).
Thus, 
\(C_j[c_1,\ldots,c_k]=\TRUE\).
Thus, we have \(\esem{e'}{\env}\subseteq \sem{\xi_j}\) as required.

\item Case \rn{Sub-Fun}:\\
In this case, for every \(j\in J\), we have:
\[
\begin{array}{l}
\xi_j' = \dtI_{i\in S_j'} (x\COL\xi'_{1,i}\ra \xi'_{2,i})\\
\xi_j = \dtI_{i\in S_j} (x\COL\xi_{1,i}\ra \xi_{2,i})\\
S_j\subseteq S_j'\\
\sigma = x\COL\sigma_1 \ra \sigma_2\\
\sigma' = x\COL\sigma_1'\ra \sigma_2'\\
%%\delta' = \dtI_{i\in\set{1,\ldots,m,\ldots,n}} (x\COL\delta'_{1,i}\ra \delta'_{2,i})\\
%%\delta = \dtI_{i\in\set{1,\ldots,m}} (x\COL\delta_{1,i}\ra \delta_{2,i})\\
\forall i\in S_j.
\comp{\Xi_j}{\Gamma} \pDT \comp{\xi_{1,i}}{\sigma_1} \subT \comp{\xi'_{1,i}}{\sigma'_1}\\
\forall i\in S_j.
\comp{\Xi_j}{\Gamma},x\COL\comp{\xi_{1,i}}{\sigma_1} \p \comp{\xi'_{2,i}}{\sigma'_2}\subT \comp{\xi_{2,i}}{\sigma_2}
%%  \mbox{ for each $i\in \bigcup_{j\in J}S_j$}\\
\end{array}
\]
By the assumption, we have:
\[
\begin{array}{l}
\T{\Gamma}{e}{\delta'}{e_0'}\\
\end{array}
\]
Since \(\T{\Gamma, x\COL\sigma}{x}{\sigma}{x}\) and \(\forall \env\in\sem{\Xi,x\COL\xi}.\esem{x}{\env}\subseteq \sem{\xi}\)
hold in general, by the induction hypothesis, we have:
\[
\begin{array}{l}
\T{\Gamma,x\COL\sigma_1}{x}{\sigma_1'}{e_1'}\\
\forall j\in J.\forall i\in S_j.\forall \env_1\in\sem{\Xi_j,x\COL\xi_{1,i}}.
   \esem{e_1'}{\env_1}\subseteq \sem{\xi'_{1,i}} \\
\hfill \cdots \mbox{(1)}\\
\T{\Gamma,x\COL\sigma_1,x'\COL\sigma_1',y\COL\sigma_2'}{y}{\sigma_2}{e_2'}\\
\forall j\in J.\forall i\in S_j.\forall \env_2\in\sem{\Xi_j,x\COL\xi_{1,i},x\COL\xi'_{1,i},y\COL\xi_{2,i}'}.\\
\qquad   \esem{e_2'}{\env_2}\subseteq \sem{\xi_{2,i}}
\hfill \cdots \mbox{(2)}\\
\end{array}
\]
%%for each \(i\in S\). \koba{Here, how can we guarantee that \(e_1', e_2'\) do not depend on \(i\)?}

Let \(e'\) be
\[\lambda x.\LETEQIN{x'}{e_1'}\LETEQIN{y}{e_0'\,x'}{e_2'}.\]
Then we have
\[\T{\Gamma}{e}{\sigma}{e'}.\]

It remains to show
\[\forall j\in J.\forall\env\in \sem{\Xi_j}.
(\esem{e_0'}{\env}\subseteq \sem{\xi_j'}\imply \esem{e'}{\env}\subseteq \sem{\xi_j}).\]
Suppose \(\env\in \sem{\Xi}\) and \(\esem{e_0'}{\env}\subseteq \sem{\xi_j'}\).
Assume also that \(v_0\in \sem{\xi_{1,i}}\) (where \(i\in S_j\)).
It suffices to show 
\[\esem{\LETEQIN{x'}{e_1'}\LETEQIN{y}{e_0'\,x'}{e_2'}}{\rho_1} \subseteq \sem{\xi_{2,i}}\]
for \(\rho_1 = \rho\set{x\mapsto v_0}\).

By the condition (1), we have \(\esem{e_1'}{\rho_1}\subseteq \sem{\xi_{1,i}'}\).
Let \(v_1\in \esem{e_1'}{\rho_1}\subseteq \sem{\xi_{1,i}'}\).
By the assumption \(\esem{e_0'}{\env}\subseteq \sem{\xi_j'}\),
we also have 
\[\esem{e_0'x'}{\rho_1\set{x'\mapsto v_1}}\subseteq \sem{\xi_j'}.\]
Let \(v_2\in \esem{e_0'x'}{\rho_1\set{x'\mapsto v_1}}\)
and \(\rho_2 = \rho_1\set{x'\mapsto v_1,y\mapsto v_2}\).
Then by the condition (2), we have
\[
\esem{e_2'}{\rho_2}\subseteq \sem{\xi_{2,i}}.
\]
Thus, we have
\[\esem{\LETEQIN{x'}{e_1'}\LETEQIN{y}{e_0'\,x'}{e_2'}}{\rho_1} \subseteq \sem{\xi_{2,i}}\]
as required.
\end{itemize}
\end{proof}

The following is the key lemma, which states that if an expression has type \(\delta\) in
the dependent intersection type system, we can obtain an abstract program that is precise enough
to derive the property described by \(\delta\).
\begin{lemma}
\label{lem:relcomp-main}
Suppose that \(\comp{\Xi}{\Gamma} \pDT e: \comp{\xi}{\sigma}\).
%% with \(\Delta=\comp{\Xi}{\Gamma}\) and \(\dtvar=\comp{\xi}{\sigma}\).
%%Further, assume also that \(\FV{\sigma}=\FV{\dtvar}\).
Then, there exists \(e'\) such that
\begin{enumerate}
\item \(\T{\Gamma}{e}{\sigma}{e'}\)
\item For any \(\rho\in \sem{\Xi}\),
    \(\esem{e'}{\rho}\subseteq \sem{\xi}\).
\end{enumerate}
\end{lemma}

\begin{proof}
This follows by induction on the derivation of \(\comp{\Xi}{\Gamma} \pDT e: \comp{\xi}{\sigma}\),
with case analysis on the last rule used.

\begin{itemize}
\item \textit{{\rn{D-Base}}}:
In this case, \(\xi = \hole_i\).
Let \(\dtvar = \rt{\nu}{b}{\nu=e}\) and \(\sigma=b[\seq{P},\lambda \nu.\nu=e,\seq{Q}]\).
Let \(e'\) be:
\(\tuple{\seq{\nondet},\TRUE,\seq{\nondet}}\) where \(\nondet = \PARB{\TRUE}{\FALSE}\).
Then \(\T{\Gamma}{e}{\tau}{e'}\) is obtained by using the rules
\rn{A-Base}, \rn{A-CoerceAdd}, and \rn{A-CoerceRem} (modulo simplifications of the tarnslated term).
 \(\esem{e'}{\rho}\subseteq\sem{\xi}\)
also holds as required.

%%\koba{Here, it is important that the abstract values of \(\seq{P},\seq{Q}\) can be independent of
%%the actual value of \(e\).}
\item \textit{\rn{D-App}}:
In this case,
\[
\begin{array}{l}
\delta = \dtI_{j\in S} [\seq{v}/\seq{y}]\delta_j\\
\Delta(x) = \dtI_{j\in S}y_1\COL\delta_{j,1}\ra \cdots \ra y_k\COL\delta_{j,k}\ra \delta_j\\
\Delta \pDT v_i: \dtI_{j\in S} [v_1/y_1,\ldots,v_{i-1}/y_{i-1}]\delta_{j,i}
\end{array}
\]
By the condition \(\Delta = \comp{\Xi}{\Gamma}\) and \(\delta = \comp{\xi}{\sigma}\), we have:
\[
\begin{array}{l}
\Xi(x) = \dtI_{j\in S}y_1\COL\xi_{j,1}\ra \cdots \ra y_k\COL\xi_{j,k}\ra \xi_j\\
\Gamma(x) = y_1\COL \sigma_1\ra \cdots \ra y_k\COL\sigma_k\ra \sigma'\\
\delta_{j,i} = \comp{\xi_{j,i}}{\sigma_i}\\
\delta_j = \comp{\xi_j}{\sigma'}
\comp{\xi}{\sigma} = \comp{\dtI_{j\in S} \xi_j}{[\seq{v}/\seq{y}]\sigma'}
\end{array}
\]
Thus, we have:
\[
\begin{array}{l}
\dtI_{j\in S}
  [v_1/y_1,\ldots,v_{i-1}/y_{i-1}]\delta_{j,i} =\\
\qquad \comp{(\dtI_{j\in S}\xi_{j,i})}{[v_1/y_1,\ldots,v_{i-1}/y_{i-1}]\sigma_{i}}.
\end{array}\]
By the induction hypothesis, we have:
\[
\begin{array}{l}
\T{\Delta}{v_i}{[v_1/y_1,\ldots,v_{i-1}/y_{i-1}]\sigma_{i}}{e'_{i}}\\
\esem{e_i'}{\rho} \subseteq \sem{\dtI_{j\in S}\xi_{j,i}} 
  \subseteq \sem{\xi_{j,i}}.
\end{array}
\]

By the assumption \(\rho\in \sem{\Xi}\) and 
\(\Xi(x) = \dtI_{j\in S}y_1\COL\xi_{j,1}\ra \cdots \ra y_k\COL\xi_{j,k}\ra \xi_j\),
we have \(\esem{x\, e_1\,\cdots\,e_k}{\rho} \subseteq
                 \sem{\xi_j}\).
Thus, we have
\[\esem{x\, e_1\,\cdots\,e_k}{\rho} \subseteq
                \sem{\dtI_{j\in S}\xi_j}.\]

By the condition \(\comp{\xi}{\sigma} = \comp{\dtI_{j\in S} \xi_j}{[\seq{v}/\seq{y}]\sigma'}\)
and Lemma~\ref{lem:relcomp-coerce}, we have \(e'\) such that
\[
\begin{array}{l}
\T{\Gamma}{e}{\sigma}{e'}\\
\forall \env\in\set{\Xi}.\esem{e'}{\rho}\subseteq \sem{\xi}
\end{array}
\]
\item \textit{\rn{D-Let}}:
In this case, we have:
\[
\begin{array}{l}
e = \LETEQIN{x}{e_1}{e_2}\\
\comp{\Xi}{\Gamma} \pDT e_1: \comp{\xi_1}{\sigma_1}\\
\comp{\Xi}{\Gamma}, x\COL\comp{\xi_1}{\sigma_1}\pDT e_2: \comp{\xi_2}{\sigma_2}
\end{array}
\]
By the induction hypothesis, we have:
\[
\begin{array}{l}
\T{\Gamma}{e_1}{\sigma_1}{e_1'}\\
\T{\Gamma, x\COL\sigma_1}{e_2}{\sigma}{e_2'}\\
\forall \rho\in \sem{\Xi}.\esem{e_1'}{\rho} \subseteq\sem{\xi_1}\\
\forall \rho\in \sem{\Xi}.\forall v\in \sem{xi_1}.
 \esem{e_2'}{\rho\set{x\mapsto v}}\in \sem{\xi}
\end{array}
\]
%%\koba{What if \(\sem{\rho\dtvar',\rho\sigma'}\) is empty??-- In that case,
%%\(\esem{e_1'}{\rho'}\) is empty, so that the condition below holds.}
Let \(e' \equiv \LETEQIN{x}{e_1'}{e_2'}\). Then
From the first two conditions above and \rn{A-Let}, we obtain
\[ \T{\Gamma}{e}{\sigma}{e'}.\]
From the last two conditions, we obtain 
\(\FAIL\not\in \esem{e_1'}{\rho}\) and 
\[\forall \rho\in \sem{\Xi}.\forall v\in \esem{e_1'}{\rho}. \esem{e_2'}{\rho\set{x\mapsto v}}\in \sem{\xi}\]
which implies 
\[\forall \rho\in\sem{\Xi}.\esem{e'}{\env}\subseteq \sem{\xi}.\]


\item \textit{\rn{D-Fail}}:
In this case, we have:
\[
\begin{array}{l}
e \equiv \FAIL\qquad\qquad
\SEM{\comp{\Xi}{\Gamma}} \imply \FALSE
\end{array}
\]
By the second condition, we have:
%% \(z_i\COL \set{\nu:b_i \mid C_i[P_{i,1}(\nu),\ldots,P_{i,k_i}]} \in \comp{\Xi}{\Gamma}
%%(i\in \set{1,\ldots,m})\) with
\[
\begin{array}{l}
z_i \COL \set{\nu\COL b_i \mid C_i} \in \Xi \mbox{ (for $i\in\set{1,\ldots,m}$)}\\
z_i \COL b_i[P_{i,1},\ldots,P_{i,k_i}] \in \Gamma\mbox{ (for $i\in\set{1,\ldots,m}$)}\\
C_1[P_{1,1}(z_1),\ldots,P_{1,k_1}(z_1)]\land \cdots\\
\land C_m[P_{m,1}(z_m),\ldots,P_{m,k_m}(z_m)]\imply \FALSE.
\end{array}
\]
%%%Let \(D[u_1,\ldots,u_m]\) be a boolean formula equivalent to:
%%%\[
%%%\begin{array}{l}
%%%\exists z_1,\ldots,z_m.
%%%  u_1 = \tuple{P_{1,1}(z_1),\ldots,P_{1,k_1}(z_1)}\\
%%% \land \cdots
%%% \land u_m = \tuple{P_{m,1}(z_m),\ldots,P_{m,k_m}(z_m)}
%%%\end{array}
%%%\]
%%%(Note that such a formula always exists, although it may not be computable. It is indeed computable
%%%in Presburger arithmetic.)

Let us define \(e'\) by:
\[
\begin{array}{l}
e' \equiv \ASSUME{\psi}{\FAIL}\\
%%e_3' \equiv \LETEQIN{y}{\TRUE}{\PARB{(\ASSUME{\psi}{\TRUE})}{(\ASSUME{\TRUE}{\FALSE})}}\\
\psi \equiv C_1[z_1]\land \cdots \land C_m[z_m] \imply \FALSE\\  %%%) \land D[z_1,\ldots,z_m]\\
%%\psi' \equiv \TRUE
\end{array}
\]
Then, since \(\models \THE{\Gamma}\psi\) holds,
we have \(\T{\Gamma}{\FAIL}{\sigma}{e'}\).
Suppose \(\env\in \sem{\Xi}\). Then, \(\esem{C_1[z_1]\land \cdots \land C_m[z_m]}{\env} = \set{\FALSE}\),
so that we have \(\esem{e'}{\env} = \emptyset \subseteq \sem{\xi}\) as required.
%%and \(\forall \env\in \sem{\Xi}.\esem{e'}{\env} = \emptyset \subseteq \sem{\xi}\).
%% (note that the value of \(\psi\) is \(\FALSE\)).
%%Thus, \(e'\equiv\LETEQIN{x}{e_1'}{\ASSUME{x}{e_2'}}\) satisfies the required conditions.

\item \textit{\rn{D-Assume}}:
In this case, we have:
\[
\begin{array}{l}
e \equiv \ASSUME{v}{e_1}\\
\comp{\Xi}{\Gamma}, x\COL \set{\nu:\TUNIT\mid v} \pDT e:\comp{\xi}{\sigma}
\end{array}
\]
Let \(\xi = \set{\nu:\TUNIT\mid \hole_1}\) and \(\sigma_1 = \TUNIT[\lambda x.v]\).
By the induction hypothesis, we have \(e'_1\) such that
\[
\begin{array}{l}
 \T{\Gamma, x\COL \sigma_1}{e_1}{\sigma}{e_1'}\\
\forall \env\in\sem{\Xi}.\forall w\in\sem{\xi}.
  \esem{e_1'}{\rho\set{x\mapsto w}}\subseteq \sem{\xi}
\end{array}
\]

Let us define \(e_2'\) by:
\[
\begin{array}{l}
e_2' \equiv \LETEQIN{\tuple{u,w}}{e_3'}{w}\\
e_3' \equiv \LETEQIN{y}{\TRUE}{\PARB{(\ASSUME{\psi}{\TRUE})}{(\ASSUME{\psi'}{\FALSE})}}\\
\psi \equiv \psi' \equiv \TRUE\\
\end{array}
\]
Let \(e'\) be \(\LETEQIN{x}{e_1'}{\ASSUME{x}{e_2'}}\).
Then, by using \rn{A-Base}, \rn{A-CoerceAdd}, and \rn{A-CoerceRem}, we obtain
\[\T{\comp{\Xi}{\Gamma}}{e}{\comp{\xi}{\sigma}}{e'}.\]
Suppose \(\env\in \sem{\Xi}\). 
Then, we have
\[ \esem{e'}{\env} = \esem{e_1'}{\env}=\esem{e_1'}{\env\set{x\mapsto \TRUE}} \subseteq \sem{\xi}\]
as required.

\item Case \rn{D-Par}:
In this case, we have:
\[
\begin{array}{l}
e \equiv e_1 \PAROP e_2\\
\comp{\Xi}{\Gamma} \pDT e_i: \comp{\xi}{\sigma}
\end{array}
\]
By the induction hypothesis, we have:
\[
\begin{array}{l}
\T{\Gamma}{e_1}{\sigma}{e_1'}\\
\forall \env\in\sem{\Xi}.
\esem{e'_1}{\env} \subseteq \sem{\xi}\\
\T{\Gamma}{e_2}{\sigma}{e_2'}\\
\forall \env\in\sem{\Xi}.
\esem{e'_2}{\env} \subseteq \sem{\xi}\\
\end{array}
\]
The required conditions hold for 
\(e' = e'_1\PAROP e'_2\).

\item Case \rn{D-Coerce}:
In this case, we have
\[
\begin{array}{l}
\comp{\Xi}{\Gamma} \pDT e: \comp{\xi'}{\sigma'}\\
\comp{\xi'}{\sigma'}\subT \comp{\xi}{\sigma}
\end{array}
\]
By the induction hypothesis, we have:
\[
\begin{array}{l}
\T{\Gamma}{e}{\sigma'}{e_1'}\\
\forall \env\in \sem{\Xi}.\esem{e_1'}{\env} \subseteq \sem{\xi'}
\end{array}
\]
By Lemma~\ref{lem:relcomp-coerce}, we have \(e'\) that satisfies the required conditions.
\end{itemize}

\end{proof}

\begin{lemma}[correspondence between the denotational and operational semantics]
\label{lem:denot}
Let \(e\) be a closed term.
If \(e\reds{} \FAIL\), then \(\FAIL \in \esem{e}{\emptyset}\).
\end{lemma}
\begin{proof}
This follows by straightforward induction on the length of \(e\reds{}\FAIL\).
\end{proof}

\begin{pfof}{Theorem~\ref{th:rel-complete}}
Let \(D = \set{f_1(\seq{x}_1) = e_1,\ldots,f_n(\seq{x}_n)=e_n}\).
Suppose \(\DPT{}{D}{\Delta}\) and \(\Delta= \comp{\Xi}{\Gamma}\). 
Then, by Lemma~\ref{lem:relcomp-main}, we have an abstract program
\(D' = \set{f_1(\seq{x}_1)=e_1',\ldots, f_n(\seq{x}_n)=e_n'}\) such that
\[
\begin{array}{l}
\T{}{D}{\Gamma}{D'}\\
%%\Gamma(f_i) = \seq{x}_i\COL\seq{\sigma}_i\ra \sigma_i'\\
\forall \env\in\sem{\Xi}.\esem{\lambda \seq{x}_i.e'_i}{\env} \subseteq \sem{\Xi(f_i)}
\end{array}
\]
Let \(e'^{(m)}\) be the closed expression:
\[
\begin{array}{l}
\LETEQIN{\tuple{f_1^{(0)},\ldots,f_n^{(0)}}}
    {\tuple{\lambda \seq{x}_1.\bot_1,\ldots,\lambda \seq{x}_n.\bot_n}}\\
\LETEQIN{\tuple{f_1^{(1)},\ldots,f_n^{(1)}}}
    {\\\qquad\tuple{\lambda \seq{x}_1.[\seq{f}^{(0)}/\seq{f}]e_1',\ldots,\lambda \seq{x}_1.[\seq{f}^{(0)}/\seq{f}]e_n'}}\\
\cdots \\
\LETEQIN{\tuple{f_1^{(m)},\ldots,f_n^{(m)}}}
    {\\\qquad\tuple{\lambda \seq{x}_1.[\seq{f}^{(m-1)}/\seq{f}]e_1',\ldots,\lambda \seq{x}_1.[\seq{f}^{(m-1)}/\seq{f}]e_n'}}\\
\quad \textit{main}^{(m)}\tuple{}
\end{array}
\]
Here, \(\bot_i= \ASSUME{\FALSE}c_i\) where \(c_i\) is a constant of the result type of \(f_i\).
Intuitively, \(e'^{(m)}\) is the program obtained from \(\textit{main}\) by unfolding the function definitions of \(D'\)
\(m\) times.
By the construction and the condition
\(\forall \env\in\sem{\Xi}.\esem{\lambda \seq{x}_i.e'_i}{\env} \subseteq \sem{\Xi(f_i)}\),
for every \(m\), we have \(\esem{e'^{(m)}}{\emptyset} \subseteq \sem{\TUNIT{C}}\) for some \(C\).
Thus, \(\FAIL\not\in \esem{e'^{(m)}}{\emptyset}\).
By Lemma~\ref{lem:denot}, \(e'\not\reds{}\FAIL\) for every \(m\).
Thus, we have \(\textit{main}\tuple{} \not\redswith{}{D'} \FAIL\).
\end{pfof}
%%\end{pfsketch}
