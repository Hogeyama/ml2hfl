\section{Predicate Abstraction}
\label{sec:pred}

This section formalizes predicate abstraction for higher-order programs.
%
As explained in Section~\ref{sec:intro}, we use \emph{abstraction types} 
to express which predicates should be used for abstracting each 
sub-expression. The syntax of abstraction types is given by:
\begin{eqnarray*}
\sigma \mbox{ (abstraction~types) }&::=& b_1[\seq{P}] \mid \cdots \mid b_n[\seq{P}] \mid x\COL\sigma_1 \to \sigma_2 \\
P,Q \mbox{ (predicates) }&::=& \lambda x.\psi \\
\Gamma \mbox{ (type~environments)} &::=& \emptyset \mid \Gamma,f\COL\sigma \mid \Gamma,x\COL\sigma
%\psi&::=& \NTH{i}{x} \mid \NOT \psi \mid \psi_1 \AND \psi_2
\end{eqnarray*}
Here, the meta-variable \(\psi\) represents an expression of type 
\(\TBOOL\) (which is called \emph{a formula}) that is constructed only 
from variables of base types, constants, and primitive operations; we do 
not allow formulas that contain function applications, like \(y>f(x)\).
%%We also use the meta-variable \(\psi\) to denote Boolean formulas (i.e, formulas that 
%%can be represented as Boolean expressions).
% we call the expressions of the Boolean progrms, Boolean expressions
%
% whose sub-expression can only have the type of the form \(\TBOOL\times 
%\cdots \times \TBOOL\). 
The base type \(b_i[\seq{P}]\) describes values \(v\) of type \(b_i\) 
that should be abstracted to a tuple of booleans
%that over-approximates 
\(\tuple{P_1(v),\dots,P_n(v)}\). For example, the integer \(3\) with the 
abstraction type \(\TINT[\lambda \nu.\nu>0,\lambda \nu.\nu<2]\) is 
abstracted to \(\tuple{\TRUE,\FALSE}\). We often abbreviate \(b[\,]\) 
to \(b\).
%(Recall again that it does not mean the value should satisfy \(\seq{P}\); an integer \(3\) can have type \(\TINT[\lambda x.x<0]\).)
The dependent function type \(x\COL\sigma_1 \to \sigma_2\) describes 
functions that take a value \(v\) of type \(\sigma_1\), and return a 
value of type \([v/x]\sigma_2\).
% if applied to the actual argument \(v\) of type \(\sigma_1\).
The scope of \(x\) in the type \(x\COL\sigma_1 \to \sigma_2\) is 
\(\sigma_2\). When \(x\) does not occur in \(\sigma_2\), we often write 
\(\sigma_1\to\sigma_2\) for \(x\COL\sigma_1 \to \sigma_2\).
%
%In the function type, the variable \(x\) denotes the actual argument, 
%and can be used in the predicates in \(\sigma_2\).
As mentioned already, abstraction types only describe how each 
expression should be abstracted, not the actual value. For example, \(3\) 
can have type \(\INT[\lambda \nu.\nu<0]\), and \(\lambda x.x\) can have 
type \(x\COL\INT[\,] \ra \INT[\lambda \nu.\nu=x+1]\) (and abstracted to
%%a function 
\(\lambda x:\TUNIT.\FALSE\)).

We do not consider types whose predicates are ill-typed or violate a 
variable's scope, such as \(x\COL\TBOOL[\,]\to \TINT[\lambda y.x+1=y]\) 
and \(x\COL\TINT[\lambda x.y>x]\to y\COL\TINT[\,]\to \TBOOL[\,]\). (The 
former uses a boolean variable as an integer, and the latter refers to 
the variable \(y\) outside its scope.) Figure~\ref{fig:wf} defines the 
well-formedness conditions for types and type environments. In the 
figure, \(\Delta\pS e:\tau\) denotes the type judgment of the standard 
simple type system. We write \(\alphaB(\sigma)\) and \(\alphaB(\Gamma)\) 
respectively for the simple type and the simple type environment 
obtained by removing predicates. For example, 
%%\begin{eqnarray*}
%%&&
\(\alphaB(f\COL (x\COL\TINT[\,]\to \TINT[\lambda y.y>x]), z\COL\TINT[\lambda z.z>0]) %%\\
  = f\COL \TINT \to \TINT,z\COL \TINT\).
%%  &=& f\COL \TINT \to \TINT,z\COL \TINT.
%%\end{eqnarray*}
\begin{figure}
\infrule{
    \alphaB(\Gamma),x\COL b \pS \psi_i: \TBOOL \mbox{ for each $i\in\set{1,\ldots,n}$}}
   {\Gamma \pW b[\lambda x.\psi_1,\ldots, \lambda x.\psi_n]}
\rulesp
\InfruleW{0.45}{\Gamma \pW \sigma_1 \andalso \Gamma, x\COL\sigma_1 \pW \sigma_2}
   {\Gamma \pW x\COL\sigma_1 \to \sigma_2}
\InfruleW{0.15}{\ }{\pW \emptyset}
\InfruleW{0.25}{\pW \Gamma \andalso \Gamma \pW \sigma}
      {\pW \Gamma, x\COL\sigma}
\caption{Well-formed types and type environments}
\label{fig:wf}
\end{figure}


Figure~\ref{fig:pred_abst} defines the predicate abstraction relation 
\(\T{\Gamma}{e_1}{\sigma}{e_2}\), which reads that an expression \(e_1\) 
(of the source language in Section~\ref{sec:lang}) can be abstracted to 
an expression \(e_2\) (of the HBP language given in 
Section~\ref{sec:boolean}) by using the abstraction type \(\sigma\), 
under the assumption that each free variable \(x\) of \(e_1\) has been 
abstracted using the abstraction type \(\Gamma(x)\). In the rules, it is 
implicitly assumed that all the type environments and types are 
well-formed. We do not distinguish between function variables and other 
variables (hence, \rn{A-App} applies also to a function variable \(f\)).
%%The predicate abstraction relation \(\T{\Gamma}{e_1}{\sigma}{e_2}\) 
%%reads that the Boolean expression \(e_2\) is an abstraction of the 
%%expression \(e_1\) with the abstraction type \(\sigma\) under 
%%the 
%%abstraction type environment \(\Gamma\).
%
%%The derivation rules for the relation are presented in 
%
%The rules are .. the previous dependent type 
%systems~\cite{Flanagan2006,Rondon2008,Unno2009,Terauchi2010} except that 
%...
%
%In \rn{A-Var}, we write \(x\ \seq{e}\) for \(\LETEQIN{\seq{x}}{\seq{e}}{x\ \seq{x}}\).
%In \rn{A-Fun}, we write \(f\ \seq{e}\) for \(\LETEQIN{\seq{x}}{\seq{e}}{f\ \seq{x}}\).
%\rn{A-Assume} and 
In \rn{A-CAdd}, \(\ASSUME{e_1}{e_2}\) is a syntax sugar for 
\(\LETEQIN{x_1}{e_1}{\ASSUME{x_1}{e_2}}\).

\begin{figure}[tbp]
\typicallabel{A-Assume}

%%%\infrule[A-Const]
%%%  {c\mbox{~has~the~base~type~}b}
%%%  {\T{\Gamma}{c}{b[\lambda x.x=c]}{\TRUE}}
%%%\rulesp
%%%
%%%\infrule[A-Op]
%%%  {\alphaB(\Gamma) \pS \OP(\seq{v}):b}
%%%  {\T{\Gamma}{\OP(\seq{v})}{b[\lambda x.x=\OP(\seq{v})]}{\TRUE}}
%%%\rulesp
%%%
%%%
%%%\infrule[A-Var]
%%%  {x:b[\seq{P}] \in \Gamma}
%%%  {\T{\Gamma}{x}{b[\lambda y.y=x]}{\TRUE}}
%%%\rulesp


\infrule[A-Base]
  {\mbox{$e$ is a constant, a variable or an expression of the form \(\OP(\seq{v})\)}\\
    \alphaB(\Gamma) \pS e \COL b}
  {\T{\Gamma}{e}{b[\lambda \nu.\nu=e]}{\TRUE}}
\rulesp

\infrule[A-CAdd]
  {\T{\Gamma}{e}{b[\seq{Q}]}{e'} \\
   \beta(\Gamma,x:b[\seq{Q}]) \pS \psi \COL \TBOOL \andalso
   \beta(\Gamma,x:b[\seq{Q}]) \pS \psi' \COL \TBOOL \\
   \models  P(x) \IMPLY \THE{\Gamma,x:b[\seq{Q}]}(\psi) \andalso
   \models \neg P(x) \IMPLY \THE{\Gamma,x:b[\seq{Q}]}(\psi') \\
   e''=\PARB{(\ASSUME{\psi}{\TRUE})}{(\ASSUME{\psi'}{\FALSE})}}
  {\T{\Gamma}{e}{b[\seq{Q},P]}{\LETEQIN{x}{e'}{\tuple{x,e''}}}}
\rulesp

\infrule[A-CRem]
  {\T{\Gamma}{e}{b[\seq{Q},\seq{P}]}{e'}}
  {\T{\Gamma}{e}{b[\seq{P}]}{\LETEQIN{\tuple{\seq{x},\seq{y}}}}{e'}{\tuple{\seq{y}}}}
\rulesp

\infrule[A-App]
  {\Gamma(x)=(y_1\COL\sigma_1 \to \cdots \to y_k\COL\sigma_k \to\sigma) \\
%% \andalso   \mbox{\(\Gamma(x)\) is a function type} \\
    \T{\Gamma,y_1:\sigma_1,\dots,y_{i-1}:\sigma_{i-1}}{v_i}{[v_1/y_1,\ldots,v_{i-1}/y_{i-1}]\sigma_i}{e_i}\\
  \mbox{ (for each $i\in\set{1,\ldots,k}$)}}
  {\T{\Gamma}{x\ \seq{v}}{[\seq{v}/\seq{y}]\sigma}
     {\begin{array}{l}
      \LETEQIN{y_1}{e_1}\cdots \\
      \LETEQIN{y_k}{e_k}x\ \seq{y}
      \end{array}}}
\rulesp
%%%
%%%\infrule[A-Fun]
%%%  {f:(\seq{x}:\seq{\sigma} \to \sigma) \in \Gamma \andalso
%%%   \T{\Gamma}{\seq{v}}{\seq{\sigma}}{\seq{e}}}
%%%  {\T{\Gamma}{f\ \seq{v}}{[\seq{v}/\seq{x}]\sigma}{f\ \seq{e}}}
%%%\rulesp

\infrule[A-Let]
  {\T{\Gamma}{e_1}{\sigma'}{e_1'} \andalso
   \T{\Gamma,x:\sigma'}{e_2}{\sigma}{e_2'}}% \andalso
%   x \not\in \FV{\sigma}}
  {\T{\Gamma}{\LETEQIN{x}{e_1}{e_2}}{\sigma}{\LETEQIN{x}{e_1'}{e_2'}}}
\rulesp


\infrule[A-Fail]
  {\beta(\Gamma) \pS \psi \COL \TBOOL \andalso
   \models \THE{\Gamma}(\psi)}
  {\T{\Gamma}{\FAIL}{\sigma}{\ASSUME{\psi}{\FAIL}}}
\rulesp

\infrule[A-Asm]
  {\T{\Gamma}{v}{\TBOOL[\lambda x.x]}{e_1} \\
   \T{\Gamma,x\COL\TBOOL[\lambda x.v]}{e}{\sigma}{e_2}}
  {\T{\Gamma}{\ASSUME{v}{e}}{\sigma}{\LETEQIN{x}{e_1}\ASSUME{x}{e_2}}}
\rulesp

\infrule[A-Par]
  {\T{\Gamma}{e_1}{\sigma}{e_1'} \andalso
   \T{\Gamma}{e_2}{\sigma}{e_2'}}
  {\T{\Gamma}{\PAR{e_1}{e_2}}{\sigma}{\PAR{e_1'}{e_2'}}}
\rulesp

%\infrule[A-Par2]
%  {\T{\Gamma}{e_1}{\sigma}{e_1'} \andalso
%   \T{\Gamma}{e_2}{\sigma}{e_2'}}
%  {\T{\Gamma}{\PARB{e_1}{e_2}}{\sigma}{\PARB{e_1'}{e_2'}}}
%\rulesp

%\infrule[A-Tuple]
%  {\T{\Gamma}{v_i}{\sigma_i}{e_i} \andalso
%   (\mbox{for~each~}i=1,\dots,m)}
%  {\T{\Gamma}{(v_1,\dots,v_m)}{\sigma_1 \times \dots \times \sigma_m}{(e_1,\dots,e_m)}}
%\rulesp
%
%\infrule[A-Nth]
%  {\T{\Gamma}{v}{\sigma_1 \times \dots \times \sigma_m}{e}}
%  {\T{\Gamma}{\NTH{i}{v}}{\sigma_i}{\NTH{i}{e}}}
%\rulesp


\infrule[A-CFun]
  {\T{\Gamma}{e}{(x\COL\sigma'_1\to\sigma'_2)}{e'}
  \andalso \T{\Gamma,x\COL\sigma_1}{x}{\sigma'_1}{e_1'}
  \\ \T{\Gamma,x\COL\sigma_1,x'\COL\sigma'_1,y\COL\sigma'_2}{y}{\sigma_2}{e_2'}}
  {\T{\Gamma}{e}{(x\COL\sigma_1\to\sigma_2)}
   {\begin{array}{l}
    \LETEQIN{f}{e'} \\
    \lambda x.\LETEQIN{x'}{e_1'} \\
    \ \ \ \ \ \ \LETEQIN{y}{f\ x'}e_2'\end{array}}}

\rulesp

\infrule[A-Prog]
  {f_i: (\seq{x}_i:\seq{\sigma}_i \to \sigma_i) \in \Gamma \andalso
   \T{\Gamma,\seq{x}_i:\seq{\sigma}_i}{e_i}{\sigma_i}{e_i'} \\
   \mbox{(for each \(i \in \set{1,\dots,m}\))} \andalso
   \Gamma(\textit{main})=\TUNIT[\,] \to \TUNIT[\,]}
  {\T{}{\set{f_1\ \seq{x}_1 = e_1,\ldots,f_m\ \seq{x}_m = e_m}}{\Gamma}{\\\set{f_1\ \seq{x}_1 = e_1',\ldots,f_m\ \seq{x}_m = e_m'}}}
%%\koba{Do we need somewhere the condition that the abstraction types conform to the simple types?}
\caption{Predicate Abstraction Rules}
\label{fig:pred_abst}
\end{figure}


In \rn{A-CAdd} and \rn{A-Fail}, \(\beta(\sigma)\) (\(\beta(\Gamma)\), 
resp.) represent the simple type (simple type environment, resp.) 
obtained by replacing each occurrence of a base abstraction type 
\(b[P_1,\dots,P_m]\) with \(\underbrace{\TBOOL \times \cdots \times 
\TBOOL}_m\). Intuitively, \(\beta(\Gamma)\) represents the type 
environment for the output program of the transformation. 

%%Similar syntax sugars are also used in \rn{A-Assume} and \rn{A-Coerce}.
%%%In \rn{A-Assume} and \rn{A-Coerce}, we write \(\ASSUME{e_1}{e_2}\) for:
%%%\[
%%% \LETEQIN{x}{e_1}{\ASSUME{x}{e_2}}.
%%%\]
%%%In \rn{A-Coerce}, we write \(\tuple{e_1,\dots,e_n}\) for:
%%%\[
%%% \LETEQIN{x_1}{e_1}{\dots \IN\ \LETEQIN{x_n}{e_n}{\tuple{x_1,\dots,x_n}}}.
%%%\]
%




We explain the main rules. 
Base values are abstracted by using three rules \rn{A-Base}, \rn{A-CAdd}, and \rn{A-CRem}.
Before explaining those rules, 
let us discuss the following simplified version, specialized
for a single predicate:
\infrule[A-BSimp]
  {
%%\mbox{$e$ is a constant, a variable or an expression of the form \(\OP(\seq{v})\)}\\
%%    \alphaB(\Gamma) \pS e \COL b\\
%%   \beta(\Gamma,x:b[\seq{Q}]) \pS \psi \COL \TBOOL \andalso
%%   \beta(\Gamma,x:b[\seq{Q}]) \pS \psi' \COL \TBOOL \\
   \models  P(e) \IMPLY \THE{\Gamma}(\psi) \andalso
   \models \neg P(e) \IMPLY \THE{\Gamma}(\psi')}
  {\T{\Gamma}{e}{b[P]}{\PARB{(\ASSUME{\psi}{\TRUE})}{(\ASSUME{\psi'}{\FALSE})}}}

\noindent
Here, we assume that \(e\) is a constant, a variable, or an expression of the form \(\OP(\seq{v})\)
and has a base type \(b\). \(\psi\) and \(\psi'\) are boolean formulas that may contain variables in \(\Gamma\).
As \(e\) may contain variables, we need to take into account information about the values of the variables,
which is obtained by using the substitution \(\THE{\Gamma}\), defined as:
%%\begin{eqnarray*}
\(\set{x \mapsto \tuple{P_1(x),\dots,P_m(x)} \mid \Gamma(x)=b[P_1,\dots,P_m]}\).
%%\end{eqnarray*}
For example, let \(\Gamma\) be \(x\COL\TINT[\lambda x.x\mathbin{>}0, \lambda x.x<0]\)
and \(\psi\) be \(\NTH{1}(x) \land \NTH{2}(x)\). Then,
\(\THE{\Gamma}(\psi) = x>0\land x<0\).
As in this example, the substitution \(\THE{\Gamma}\) maps a boolean expression of an abstract program
to the corresponding condition in the source program.
In rule \rn{A-BSimp} above, \(\models  P(e) \IMPLY \THE{\Gamma}(\psi)\) means that
\(P(e)\) is true only if \(\THE{\Gamma}(\psi)\) is true, i.e. the value of \(\psi\) in the abstract program is true.
Thus, the abstract value of \(e\) may be \(\TRUE\) only if the value of \(\psi\) is \(\TRUE\),
hence the part \(\ASSUME{\psi}{\TRUE}\) in the abstract program. Similarly,
the abstract value of \(e\) may be \(\FALSE\) only if the value of \(\psi'\) is \(\TRUE\),
hence the part \(\ASSUME{\psi'}{\FALSE}\).

For example, let \(e \equiv x+1\),  \(P \equiv \lambda x\geq 0\), and \(\Gamma \equiv x\COL \TINT[P]\).
Then, \(\models P(x+1) \imply \TRUE\) and \(\models \neg P(x+1) \imply \neg P(x)\), so that 
\(e\) is abstracted to \(\PARB{(\ASSUME{\TRUE}{\TRUE})}{(\ASSUME{\neg x}{\FALSE})}\).
Note that \(\THE{\Gamma}(\neg x) = [P(x) /x] \neg x = \neg P(x)\).

We need to generalize the above rule to the case for multiple predicates.
The following is a naive rule.
\infrule[A-BCartesian]
  {\T{\Gamma}{e}{b[P_i]}{e_i}}
  {\T{\Gamma}{e}{b[P_1,\ldots,P_n]}{\tuple{e_1,\ldots,e_n}}}
This produces a well-known cartesian abstraction, which is often too imprecise.
The problem is that each boolean value of the abstraction is computed separately,
ignoring the correlation. 
For example, let \(P_1 \equiv \lambda x.x>0\) and
\(P_2 \equiv \lambda x.x \leq 0\) with \(n=2\). Then, a possible abstraction of an unknown integer
should be \(\tuple{\TRUE,\FALSE}\) and \(\tuple{\FALSE,\TRUE}\), but the above
rule would generate 
\(\tuple{(\PARB{\TRUE}{\FALSE}),(\PARB{\TRUE}{\FALSE})}\), which also contains
\(\tuple{\TRUE,\TRUE}\) and \(\tuple{\FALSE,\FALSE}\).

The discussion above motivated us to introduce the three rules 
\rn{A-Base}, \rn{A-CAdd}, and \rn{A-CRem}. 
In order to abstract an expression \(e\) with \(b[P_1,\ldots,P_n]\),
we first use \rn{A-Base} to abstract \(e\) to \(\TRUE\) by using the abstraction type \(b[\lambda \nu.\nu=e]\);
this is necessary to keep the exact information about \(e\) during the computation of abstractions.
\rn{A-CAdd} is then used to add predicates \(P_1,\ldots,P_n\) one by one, taking into account
the correlation between the predicates. Note that in \rn{A-CAdd}, 
the result of abstraction by the other predicates is taken into account by the substitution \(\THE{\Gamma,x:b[\seq{Q}]}\).
Finally, \rn{A-CRem} is used to remove the unnecessary predicate \(\lambda \nu.\nu=e\).
See Example~\ref{ex:abstraction} for an application of these rules.

Note that rule \rn{A-CAdd} is non-deterministic in the choice of 
conditions \(\psi\) and \(\psi'\), so that how to compute the conditions 
is left unspecified. We have intentionally made so, because depending on 
base data types, the most precise conditions (the strongest conditions 
entailed by \(P(x)\) and \(\neg P(x)\)) may not be computable or are too 
expensive to compute. For linear arithmetics, however, we can use 
off-the-shelf automated theorem provers to obtain such conditions.

%
%%%In the rule \rn{A-Base}, a base value \(e\) 
%%%is abstracted to \(\TRUE\), by using the equality predicate \(\lambda 
%%%x.x=e\). Note that \(e\) is restricted to a constant, a variable, or a 
%%%primitive operation, so that \(x=e\) is a formula. To abstract base 
%%%values with other predicates, the rules \rn{A-CAdd} and \rn{A-CRem} 
%%%explained later should be used.
%%%%
%%%%
%%%%%\nk{We may want to consider merging \rn{A-Const} and \rn{A-Op}. 
%%%%%In fact, a constant \(c\) can be considered a nullary operator.}
%%%%

In rule \rn{A-App}, each argument \(v_i\) is abstracted by using the 
abstraction type \(\sigma_i\) with \(y_1,\ldots,y_{i-1}\) being replaced 
by the actual arguments. 
%%For example, we can derive:
%%\[
%% \T{\Gamma}{\texttt{inc}\ y}{\TINT[\lambda \nu.\nu=y+1]}{\texttt{inc}\ y}.
%%\]
%%for \(\Gamma=\texttt{inc}:(x:\TINT[\,] \to \TINT[\lambda \nu.\nu=x+1]),y:\TINT[\,]\).
%%%
Note that this rule applies also to the case where the sequence 
\(\seq{v}\) is empty (i.e. \(k=0\)). Thus, we can derive 
\(\T{\Gamma}{y}{\sigma}{y}\) if \(\Gamma(y)=\sigma\). Note also that the 
boolean expression \(e_i\) in \rn{A-App} can depend on 
\(y_1,\dots,y_{i-1}\).

%%In \rn{A-Fail}, we write \(\beta(\sigma)\) (\(\beta(\Gamma)\), resp.) 
%%for the simple type (simple type environment, resp.) obtained by 
%%replacing each occurrence of a base abstraction type \(b[P_1,\dots,P_m]\) 
%%with \(\underbrace{\TBOOL \times \cdots \times \TBOOL}_m\). Intuitively, 
%%\(\beta(\Gamma)\) represents the type environment for the output program 
%%of the transformation. 
%
In \rn{A-Fail}, the assume statement is inserted for filtering out an 
invalid combination of abstract values. For example, let \(\Gamma\) be 
\(x\COL\TINT[\lambda x.x\mathbin{>}0, \lambda x.x<0]\). Then, 
\(\ASSUME{(\NTH{1}(x) \land \NTH{2}(x))}\) is inserted since \(x>0\) and 
\(x<0\) cannot be true simultaneously.
%
In \rn{A-Asm}, we can use the fact that \(v\) is \(\TRUE\) in \(e\) for 
abstracting \(e\).

%%%The rules \rn{A-CAdd} and \rn{A-CRem} are used for coercing the 
%%%abstraction type of a base expression. Before explaining them, let us 
%%%first explain the following straightforward coercion rule:
%%%%
%%%\infrule
%%%  {\T{\Gamma}{e}{b[\seq{Q}]}{e'} \\
%%%   \beta(\Gamma,x:b[\seq{Q}]) \pS \psi_i \COL \TBOOL \andalso
%%%   \beta(\Gamma,x:b[\seq{Q}]) \pS \psi_i' \COL \TBOOL \\
%%%   \models P_i(x) \IMPLY \THE{\Gamma,x:b[\seq{Q}]}(\psi_i) \andalso
%%%   \models \neg P_i(x) \IMPLY \THE{\Gamma,x:b[\seq{Q}]}(\psi_i') \\
%%%   e_i=\PARB{(\ASSUME{\psi_i}{\TRUE})}{(\ASSUME{\psi_i'}{\FALSE}) \\
%%%   \mbox{(for each \(i \in \set{1,\dots,n}\))}}}
%%%  {\T{\Gamma}{e}{b[P_1,\dots,P_n]}{\LETEQIN{x}{e'}\tuple{e_1,\dots,e_n}}}
%%%%
%%%In the rule, the first premise means that \(e\) is abstracted to \(e'\) 
%%%by using predicates \(\seq{Q}\). To coerce the abstraction type of \(e\) 
%%%to \(b[P_1,\dots,P_n]\), we over-approximate the possible value of 
%%%\(P_i(e)\) for each \(i \in \set{1,\dots,n}\), by using \(e'\). Note 
%%%that the most imprecise abstraction is \(\PARB{\TRUE}{\FALSE}\), which 
%%%represents a non-deterministic boolean. To get a more precise 
%%%abstraction, we find a condition \(\THE{\Gamma,x:b[\seq{Q}]}(\psi_i)\) 
%%%(\(\THE{\Gamma,x:b[\seq{Q}]}(\psi_i')\), resp.) that is entailed by 
%%%\(P_i(x)\) (\(\neg P_i(x)\), resp.) and expressed by using only 
%%%available predicates, where \(x\) represents the value of \(e\). By 
%%%using them, we obtain a more precise abstraction of \(P_i(x)\):
%%%\[\PARB{(\ASSUME{\psi_i}{\TRUE})}{(\ASSUME{\psi_i'}{\FALSE})}.\]
%%%%
%%%For example, suppose that we have \(\T{\Gamma}{e}{\TINT[\lambda 
%%%x.x=1]}{e'}\). The abstraction \(e'\) represents whether \(e=1\) holds. 
%%%%
%%%To change the abstraction type to \(\TINT[\lambda x.x>0]\), we look for 
%%%boolean formulas \(\psi\) and \(\psi'\) such that \(\THE{\Gamma,x\COL 
%%%b[\lambda x.x=1]}(\psi)\) and \(\THE{\Gamma,x\COL b[\lambda 
%%%x.x=1]}(\psi')\) are entailed by \(x>0\) and \(\neg(x>0)\) respectively, 
%%%and obtain \(\psi = \TRUE\) and \(\psi' = \neg x\) since we have
%%%\begin{eqnarray*}
%%%&&\models x > 0 \implies \TRUE(\equiv\THE{\Gamma,x\COL b[\lambda x.x=1]}(\psi)), \\
%%%&&\models x \leq 0 \implies x \neq 1(\equiv\THE{\Gamma,x\COL b[\lambda x.x=1]}(\psi')).
%%%\end{eqnarray*}
%%%%
%%%As a result, we obtain:
%%%%
%%%\[\T{\Gamma}{e}{\TINT[\lambda x.x>0]}{\LETEQIN{x}{e'}{\tuple{e''}}}.\]
%%%%
%%%where \(e'' = \PARB{(\ASSUME{\TRUE}\TRUE)}{(\ASSUME{\neg x}\FALSE)}\). 
%%%%
%%%%For example, let us consider how the expression \(x+1\) is abstracted by 
%%%%using the abstraction type \(\TINT[\lambda \nu.\nu>0]\) under the type 
%%%%environment \(x:\TINT[\lambda \nu.\nu=0]\). We look for boolean formulas 
%%%%\(\psi_1\) and \(\psi_1'\) such that \(\THE{x:\TINT[\lambda 
%%%%\nu.\nu=0]}(\psi_1)\) and \(\THE{x:\TINT[\lambda \nu.\nu=0]}(\psi_1')\) 
%%%%are entailed by \(x+1>0\) and \(\neg(x+1>0)\) respectively, and obtain 
%%%%\(\psi_1 = \TRUE\) and \(\psi_1' = \neg x\) since we have
%%%%\begin{eqnarray*}
%%%%&&\models x+1 > 0 \implies \TRUE(\equiv\THE{x:\TINT[\lambda \nu.\nu=0]}(\psi_1)), \\
%%%%&&\models x+1 \leq 0 \implies x \neq 0(\equiv\THE{x:\TINT[\lambda \nu.\nu=0]}(\psi_1')).
%%%%\end{eqnarray*}
%%%%As a result, we obtain:
%%%%\[\T{x:\TINT[\lambda x.x=0]}{x+1}{\TINT[\lambda x.x>0]}{e'}.
%%%%\]
%%%%Here, \(e' = \PARB{(\ASSUME{\TRUE}\TRUE)}{(\ASSUME{\neg x}\FALSE)}\).
%%%%
%%%
%%%Note that the above coercion rule may cause imprecise abstraction even 
%%%if we use the most precise boolean formulas for \(\psi_i\) and 
%%%\(\psi_i'\). It is because the boolean expression \(e_i\) for each 
%%%predicate \(P_i\) is obtained without considering \(e_j\ (j \neq i)\) 
%%%for other predicates \(P_j\). For example, we get an imprecise boolean 
%%%expression \(\tuple{\PARB{\TRUE}{\FALSE},\PARB{\TRUE}{\FALSE}}\) if we 
%%%coerce the type \(\TINT[\,]\) of some expression to \(\TINT[\lambda 
%%%\nu.\nu>0,\lambda \nu.\nu<0]\). (Note here that the second predicate 
%%%must be false if the first predicate is true.)
%%%
%%%To avoid the imprecision, we instead use \rn{A-CAdd} to add predicates 
%%%one-by-one considering other ones, and then use \rn{A-CRem} to remove 
%%%unnecessary ones.
%%%%
%%%For example, given \(\T{\Gamma}{e}{\TINT[\,]}{e_0}\), we can use 
%%%\rn{A-CAdd} and \rn{A-CRem} to derive:
%%%\begin{eqnarray*}
%%%\T{\Gamma}{e}{\TINT[\lambda \nu.\nu>0,\lambda \nu.\nu<0]}{e_1}
%%%\end{eqnarray*}
%%%Here, \(e_1 \equiv \LETEQIN{x_1}{\PARB{\TRUE}{\FALSE}}e_2\) and \(e_2\) 
%%%is:
%%%\[
%%% \tuple{x_1,\IFTE{x_1}{\FALSE}{\PARB{\TRUE}{\FALSE}}}.
%%%\]
%%%%
%%%%%As the former is the most tricky rule, we explain it in detail.
%%%%%In the rule, \(\THE{\Gamma}\) represents the following substitution:
%%%%%\begin{eqnarray*}
%%%%%\set{x \mapsto \tuple{P_1(x),\dots,P_m(x)} \mid \Gamma(x)=b[P_1,\dots,P_m]}.
%%%%%\end{eqnarray*}
%%%%%%Given a Boolean formula \(\psi\) whose variables are in \(\DOM{\Gamma}\), 
%%%%%%we can regard the formula \(\THE{\Gamma}\psi\) as a \emph{concretization} 
%%%%%%of \(\psi\) with \(\Gamma\), by interpreting each variable \(x\) in 
%%%%%%\(\psi\) as the tuple \(\tuple{P_1(x),\dots,P_m(x)}\).
%%%%%% of the predicates for \(x\) specified by \(\Gamma\).
%%%%, which has the type \(\underbrace{\TBOOL \times \cdots \times 
%%%%\TBOOL}_n\)
%%%
%%%%
%%%%\addspan{A-Base is NOT subsumed by A-CADD. For example, there is no way 
%%%%to derive \(\T{\Gamma}{1}{\INT[\lambda x.x=1]}{\TRUE}\) without using 
%%%%\rn{A-Base}. (Note also that the first premise of A-CADD rule must have 
%%%%originally been derived from A-Base.)}

Rule \rn{A-CFun} is used for changing the abstraction type of a function 
from \(x\COL\sigma_1\ra \sigma_2\) to \(x\COL\sigma'_1\ra \sigma'_2\), 
which is analogous to the usual rule for subtyping-based coercion. If a 
function \(f\) is used in different contexts which require different 
abstraction types of \(f\), \rn{A-CFun} can be used to adjust the 
abstraction type of \(f\) to that required for each context.

%%%\begin{example}
%%%\label{ex:cfun}
%%%\addspan{Let \(\Gamma=f:(x:\TINT \to \TINT[\lambda \nu.\nu=x+1])\). 
%%%Suppose that in some context, \(f\) is required to have the abstraction 
%%%type \(\TINT[\lambda \nu.\nu \geq 0] \to \TINT[\lambda \nu.\nu \geq 0]\). 
%%%Predicate abstraction of \(f\) in the context proceeds as in Figure~\ref{fig:cfun}.}
%%%\end{example}

%%%\begin{figure*}
%%%\[
%%%\infers[A-CFun]
%%% {\T{\Gamma}{f}{\TINT[\lambda \nu.\nu \geq 0] \to \TINT[\lambda \nu.\nu \geq 0]}{e}}
%%% {\infers%[A-App]
%%%   {\T{\Gamma}{f}{\sigma}{f}}
%%%   {\Gamma(f)=\sigma} \andalso
%%%  \infers%[A-CRem]
%%%    {\T{\Gamma,x:\TINT[\lambda \nu.\nu \geq 0]}{x}{\TINT[\,]}{\tuple{}}}
%%%    {\infers%[A-App]
%%%      {\T{\Gamma,x:\TINT[\lambda \nu.\nu \geq 0]}{x}{\TINT[\lambda \nu.\nu \geq 0]}{x}}
%%%      {(\Gamma,x:\TINT[\lambda \nu.\nu \geq 0])(x)=\TINT[\lambda \nu.\nu \geq 0]}
%%%    } \andalso
%%%  \infers
%%%    {\T{\begin{array}{r}\Gamma,x:\TINT[\lambda \nu.\nu \geq 0],x':\TINT[\,], \\ y:\TINT[\lambda \nu.\nu=x+1]\end{array}}{y}{\TINT[\lambda \nu.\nu \geq 0]}{x}}
%%%    {%\infers[A-App]
%%%     % {\T{\Gamma,x:\TINT[\lambda \nu.\nu \geq 0]}{x}{\TINT[\lambda \nu.\nu \geq 0]}{x}}
%%%     % {(\Gamma,x:\TINT[\lambda \nu.\nu \geq 0])(x)=\TINT[\lambda \nu.\nu \geq 0]}
%%%     \vdots
%%%    }
%%% }
%%%\]
%%%Here, \(\sigma=(x:\TINT[\,] \to \TINT[\lambda \nu.\nu=x+1])\)
%%%\caption{Predicate Abstraction for Example~\ref{ex:cfun}}
%%%\label{fig:cfun}
%%%\end{figure*}

%%%The predicate abstraction rules are straightforward except for 
%%%\rn{A-Coerce}. \rn{A-Coerce} is necessary to coerce the abstraction type 
%%%\(b[\seq{Q}]\) of some expression \(e\) to another abstraction type 
%%%\(b[\seq{P}]\), which is required by the context of \(e\).
%%%%
%%%Given an abstraction \(e'\) of \(e\) with \(b[\seq{Q}]\), \rn{A-Coerce} 
%%%generates another abstraction of \(e\) with \(b[P_1,\dots,P_m]\) from 
%%%\(e'\). For each \(P_i\), \rn{A-Coerce} first finds the necessary 
%%%conditions \(\phi\) and \(\phi'\) for \(P_i(x)\) and \(\NOT P_i(x)\) to 
%%%hold respectively. Here, \(\phi\) and \(\phi'\) are represented as 
%%%\(\THE{\Gamma,x:b[\seq{Q}]}\psi_i\) and 
%%%\(\THE{\Gamma,x:b[\seq{Q}]}\psi_i'\) respectively by using some Boolean 
%%%formulas \(\psi_i\) and \(\psi_i'\), whose variable \(x\) is interpreted 
%%%as \(\tuple{\seq{Q}(x)}\). Then, we can obtain the Boolean formula 
%%%\(e_i=\PARB{(\ASSUME{\psi_i}{\TRUE})}{(\ASSUME{\psi_i'}{\FALSE})}\) for 
%%%the predicate \(P_i\) such that if \(P_i(x)\) is satisfied in the 
%%%original program, then \(e_i\) evaluates to \(\TRUE\) or diverges in the 
%%%Boolean program, and if \(\NOT P_i(x)\) is satisfied, then \(e_i\) evaluates to \(\FALSE\) or diverges.
%never evaluate to \(\TRUE\).
%never evaluate to \(\FALSE\)

%%\todo{add an example derivation}

We can read the predicate abstraction rules for 
\(\T{\Gamma}{e}{\sigma}{e'}\) as
%%a deterministic and syntax-directed 
an algorithm that takes \(\Gamma, e\) and \(\sigma\)  as input, and 
outputs \(e'\) as an abstraction of \(e\), by (1) restricting 
applications of the rules for coercion (of names \rn{A-CXyz}) to the 
actual arguments
%%\(\seq{v}\) 
of function applications,
%���S�̂ɂ�coercion���K�v
%%the function applications \(x\ \seq{v}\) and \(f\ \seq{v}\) in \rn{A-Var} and \rn{A-Fun}, 
and (2) fixing an algorithm to find the boolean formulas \(\psi\) and 
\(\psi'\) in \rn{A-CAdd}.
%%For example, we can use an algorithm to find the strongest ones as in \cite{Ball2002}.
%
(Note that in \rn{A-Let}, the type \(\sigma'\) can be obtained from 
\(\Gamma\) and \(e_1\).)
%%\todo{In \rn{A-Let}, the bound variable \(x\) is annotated with the type 
%%\(\sigma'\), which can be automatically obtained from \(e_1\).}
The rule for \(\T{}{D}{\Gamma}{D'}\) can then be interpreted as an 
algorithm that takes \(D\) and \(\Gamma\) as input, and outputs an HBP 
\(D'\) as an abstraction of \(D\).

\begin{example}
\label{ex:abstraction}
Recall the program \(M_2\) in Section~\ref{sec:intro}.
Let \(\Gamma\) be: %% the following abstraction type:
\[
\begin{array}{l}
x\COL \ITint{\lambda \nu.\nu\geq 0}, g\COL \ITint{\lambda \nu.\nu> 0} \ra \TUNIT
\end{array}
\]
The body of \(f\) is transformed as follows. \(x+1\) is transformed by:
\[
\infers[A-CRem]
  {\T{\Gamma}{x+1}{\ITint{\lambda \nu.\nu>0}}{e_2}}
  {\infers[A-CAdd]
    {\T{\Gamma}{x+1}{\ITint{\lambda \nu.\nu=x+1,\lambda \nu.\nu>0}}{e_1}}
    {\infers[A-Base]
      {\T{\Gamma}{x+1}{\ITint{\lambda \nu.\nu=x+1}}{\TRUE}}
      {}}}
\]
%
Here, \(e_1 \equiv \LETEQIN{y_1}{\TRUE}{\tuple{y_1,e_3}}\) and \(e_2 
\equiv \LETEQIN{\tuple{y_1,y_2}}{e_1}{y_2}\), with \(e_3 \equiv 
\PARB{(\ASSUME{\TRUE}{\TRUE})}{(\ASSUME{\neg (x\land y_1)}{\FALSE})}\).
%%which is equivalent to \(\TRUE\) if \(x\land y_1=\TRUE\) and \(\TRUE\PAROPB \FALSE\) otherwise.
Here, we used \(\TRUE\) and \(\neg (x\land y_1)\) as \(\psi\) and 
\(\psi'\) respectively, in %rule 
\rn{A-CAdd}. (Note that 
\(P(y_1) \imply \THE{\Gamma,y_1:\TINT[\lambda \nu.\nu=x+1]}(\psi_1)\), i.e.,
\(y_1 \leq 0 \imply \neg(x\geq0 \land y_1=x+1)\) holds.)
By simplifying \(e_2\), we get \(\IFTE{x}{\TRUE}{\TRUE\PAROPB\FALSE}\).
%%\(\T{\Gamma}{x+1}{\ITint{\lambda \nu.\nu>x}}{\ifexp{
Thus, the body \(g(x+1)\) of function \(f\) is transformed by using 
\rn{A-App} as follows:
\[
\infers%[A-App]
  {\T{\Gamma}{g (x+1)}{\TUNIT}{\begin{array}{l}\LETEQIN{y}{\IFTE{x}{\TRUE\\\quad\quad\quad\quad\quad\ \ }{(\TRUE\PAROPB\FALSE)}}g(y)\end{array}}}
  {\Gamma(g)=\ITint{\lambda \nu.\nu> 0} \ra \TUNIT \andalso
   \infers%[A-CRem]
    {\T{\Gamma}{x+1}{\ITint{\lambda \nu.\nu>0}}{e_2}}
    {\vdots}}
\]
\end{example}

Our predicate abstraction rules are applicable to programs that use 
infinite data domain other than integers. 
See 
\iffull
Appendix~\ref{sec:examplelist}
\else
\cite{Kobayashi2011}
\fi
for an example of abstracting a list-processing program.

%%%\iffull
%%%\begin{example}
%%%\label{ex:list}
%%%Consider a language having a list type \(\Tlist\) as a data type, with the following constants and
%%%operators:
%%%\[
%%%\begin{array}{l}
%%%\textit{nil}: \Tlist, \qquad 
%%%%%\textit{isnil}: \Tlist \to \TBOOL\\
%%%\textit{cons}: \TINT, \Tlist \to \Tlist\\
%%%\textit{car}: \Tlist \to \TINT, \qquad 
%%%\textit{cdr}: \Tlist \to \Tlist\\
%%%\textit{length}: \Tlist \to \TINT
%%%\end{array}
%%%\]
%%%Let us consider the following program.
%%%\begin{verbatim}
%%%let map f x = if x=nil then nil 
%%%           else let y=f(car(x)) in 
%%%                let r=map f (cdr(x)) in cons(y,r) in
%%%let g x = x+1 in
%%%  if length(map g l) = length(l) then () else fail
%%%\end{verbatim}
%%%Consider the following abstraction type environment:
%%%\[
%%%\begin{array}{l}
%%%\textit{map}\COL(\ITint{\,}\ra \ITint{\,}) \ra x\COL\Tlist[\,]\\
%%%\qquad\qquad \ra \Tlist[\lambda r.\textit{length}(r)=\textit{length}(x)] \\%%\sigma \\
%%%g\COL \ITint{\,}\ra \ITint{\,}
%%%\end{array}
%%%\]
%%%
%%%By applying the predicate abstraction (and some simplification), 
%%%we get the following abstract version of \texttt{map}: see Appendix~\ref{sec:examplelist} for 
%%%details.
%%%\begin{verbatim}
%%%let map f x = if randb() then tt
%%%              else let y=f() in
%%%               let r=map f() in r in
%%%let g x = () in
%%%  if map g () then () else fail
%%%\end{verbatim}
%%%By model-checking the abstract program, we know that the program never fails.
%%%\qed
%%%\end{example}
%%%\fi

%%%\begin{example}
%%%Recall the program \(M_3\) in Section~\ref{sec:intro}.
%%%Let \(\Gamma\) be the following abstraction type:
%%%\[
%%%\begin{array}{l}
%%%x\COL \ITint{\,}, g\COL \ITint{\lambda \nu.\nu> x} \ra \TUNIT
%%%\end{array}
%%%\]
%%%The body of the function \(f\) is transformed as follows.
%%%First, \(n+1\) is transformed by:
%%%\[
%%%\infers[A-CRem]{\T{\Gamma}{x+1}{\ITint{\lambda \nu.\nu>x}}{e_2}}
%%% {\infers[A-CAdd]{\T{\Gamma}{x+1}{\ITint{\lambda \nu.\nu=x+1,\lambda \nu.\nu>x}}{e_1}}
%%% {\infers[A-Base]{\T{\Gamma}{x+1}{\ITint{\lambda \nu.\nu=x+1}}{\TRUE}}{}}}
%%%\]
%%%Here, \(e_1 \equiv \LETEQIN{y}{\TRUE}{\tuple{y,e_3}}\)
%%%and \(e_2 \equiv \LETEQIN{\tuple{x,y}}{e_1}{y}\), where
%%%\[ e_3 \equiv  (\ASSUME{\TRUE}{\TRUE})\PAROP (\ASSUME{\neg y}{\FALSE}).\]
%%%Here, we used \(\TRUE\) and \(\neg y\) as \(\psi\) and \(\psi'\) respectively,
%%%in rule \rn{A-CAdd}. (Note that \(\neg (y>x) \imply [(y=x+1)/y](\neg y)\) holds.)
%%%By simplifying \(e_2\), we get 
%%%\(\T{\Gamma}{x+1}{\ITint{\lambda \nu.\nu>x}}{\TRUE}\).
%%%Thus, the body \(g(x+1)\) of function \(f\) is transformed into \(g(\TRUE)\).
%%%\qed
%%%\end{example}

%%%\begin{example}
%%%Recall the program \(M_3\) in Section~\ref{sec:intro}.
%%%Let \(\Gamma\) be the following abstraction type:
%%%\[
%%%\begin{array}{l}
%%%f\COL (x\COL\ITint{\,}) \ra (\ITint{\lambda w.w> x} \ra \TUNIT) \ra \TUNIT,\\
%%%h \COL (z\COL\ITint{\,}) \ra (y\COL\ITint{\lambda y.y> z}) \ra \TUNIT,\\
%%%g n\COL \ITint{\lambda x.x\geq 0}
%%%\end{array}
%%%\]
%%%The main expression \verb|if n>=0 then f n (h n) else ()| is expressed as
%%%\(\PAR{e_1}{e_2}\) where
%%%\[
%%%\begin{array}{l}
%%%e_1 \equiv \ASSUME{n\geq 0}{f\,n\,(h\,n)}\\
%%%e_2 \equiv \ASSUME{\neg n\geq 0}{()}.
%%%\end{array}
%%%\]
%%%The condition \(n\geq 0\) is transformed as follows.
%%%\[
%%%\infers[A-C]{\T{\Gamma}{n\geq 0}{\TBOOL[\lambda x.x=\TRUE]}{\LETEQIN{x}{\TRUE}{e_3}}}
%%%   {\infers[A-Op]{\T{\Gamma}{n\geq 0}{\TBOOL[\lambda x.x=(n\geq 0)]}{\TRUE}}{}}
%%%\]
%%%Here, \(e_3\) is \(\PARB{(\ASSUME{(x\IMPLY n)}{\TRUE})}{(\ASSUME{(x\IMPLY\neg n)}{\FALSE})}\).
%%%(Thus, \(\LETEQIN{x}{\TRUE}{e_3}\) is equivalent to:\\
%%%\(\PARB{(\ASSUME{n}{\TRUE})}{(\ASSUME{\neg n}{\FALSE})}\),
%%%which is further simplified to \(n\).)
%%%The expression \(f\,n\,(h\,n)\) is transformed as follows (by using \rn{A-Var}).
%%%\[
%%%\infers[A-Var]
%%%   {\T{\Gamma}{f\,n\,(h\,n)}{\TUNIT}{f\,e_4\,e_5}}
%%%   {\T{\Gamma}{n}{\ITint{\,}}{e_4} &
%%%    \T{\Gamma}{h\,n}{\sigma}{e_5}}
%%%\]
%%%Here, \(\sigma = \ITint{\lambda w.w>n}\ra\TUNIT\), 
%%%\(e_4 \equiv \LETEQIN{x}{\TRUE}{()}\) and \(e_5 \equiv \LETEQIN{y}{(\LETEQIN{x}{\TRUE}{()})}h\,y\).
%%%\(\T{\Gamma}{n}{\ITint{\,}}{e_4}\) is obtained by.
%%%\[
%%%\infers[A-C]{\T{\Gamma}{n}{\ITint{\,}}{\LETEQIN{x}{\TRUE}{()}}}
%%%   {\infers[A-Op]{\T{\Gamma}{n}{\ITint{\lambda x.x=n}}{\TRUE}}{}}
%%%\]
%%%\(\T{\Gamma}{h\,n}{\sigma}{e_5}\) is obtained by:
%%%\[
%%%\infers[A-Var]
%%%  {\T{\Gamma}{h\,n}{\sigma}{e_5}}
%%%  {\infers[A-C]{\T{\Gamma}{n}{\ITint{\,}}{\LETEQIN{x}{\TRUE}{()}}}
%%%   {\infers[A-Const]{\T{\Gamma}{n}{\ITint{\lambda x.x=n}}{\TRUE}}
%%%    {}}}
%%%\]
%%%Thus, with some trivial simplification of expressions, we obtain
%%%\(\ASSUME{n}{f()(h())}\) from \(e_1\).
%%%Similarly, we obtain \(\ASSUME{\neg n}{()}\) from \(e_2\).
%%%Therefore, the abstract version of the main expression is equivalent to:
%%%\(
%%%\IFTE{n}{f()(h())}{()}
%%%\).
%%%\qed
%%%\end{example}

%%%\begin{remark}
%%%It is possible to add the following rule for
%%%changing abstraction types of functions.
%%%%%coercing functions. 
%%%%%\todo{coercion~rule~for~functions}
%%%%\infrule
%%%%  {\SUB{\Gamma}{e_1}{\sigma_1'}{\sigma_1} \andalso
%%%%   \SUB{\Gamma,y:\sigma_1'}{e_2}{\sigma_2}{\sigma_2'}}
%%%%  {\SUB{\Gamma}{\lambda x.\lambda y.e_2\ (x\ (e_1\ y))}{(y:\sigma_1 \to \sigma_2)}{(y:\sigma_1' \to \sigma_2')}}
%%%
%%%\infrule
%%%  {\T{\Gamma}{e}{x\COL\sigma_1\to\sigma_2}{e'}
%%%  \\ \T{\Gamma,x'\COL\sigma_1'}{x'}{\sigma_1}{e_1'}
%%%  \\ \T{\Gamma,x'\COL\sigma_1',x\COL\sigma_1,y\COL\sigma_2}{y}{\sigma_2'}{e_2'}}
%%%  {\T{\Gamma}{e}{x'\COL\sigma_1'\to\sigma_2'}
%%%   {\lambda x'.\LETEQIN{x}{e_1'}\LETEQIN{y}{e'\ x}e_2'}}
%%%%%%\unno{Can we use \(\lambda x.e\)}
%%%%%%\unno{I have replaced the following rule with the above one:
%%%%%%\infrule
%%%%%%  {\T{\Gamma}{e}{x\COL\sigma_1\to\sigma_2}{e'}
%%%%%%  \\ \T{\Gamma,x\COL\sigma_1'}{x'}{\sigma_1}{e_1'}
%%%%%%  \\ \T{\Gamma,y\COL\sigma_2}{y}{\sigma_2'}{e_2'}}
%%%%%%  {\T{\Gamma}{e}{x\COL\sigma_1'\to\sigma_2'}
%%%%%%   {\lambda x'.\LETEQIN{x}{e_1'}\LETEQIN{y}{e'}e_2'}}
%%%%%%}
%%%%\rulesp
%%%This introduces more flexibility in the choice of predicates used at each program point.
%%%For example, there are two calls of a higher-order function: \(f(g)\) and \(f(h)\).
%%%With the extension above, we need not use the same abstraction type for \(g\) and \(h\).
%%%\qed
%%%\end{remark}

We discuss properties of the predicate abstraction relation below. First, 
we show that if abstraction types are consistent, there is always a 
valid transformation. 
%%Here, 
%%we write \(\alphaB(\sigma)\) (\(\alphaB(\Gamma)\), resp.)
%%for the simple type (type environment, resp.) obtained by
%%each occurrence of a base abstraction type \(b[P_1,\dots,P_m]\) 
%%with \(b\). 
We write \(\Gamma\pD e:\sigma\) for the type judgment relation obtained 
from the predicate abstraction rules by removing all the conditions on 
outputs:
%%\iffull
see 
\iffull 
Appendix~\ref{sec:abst-types}. 
\else
\cite{Kobayashi2011}.
\fi
%% for the type derivation rules.
%%\else
%%see the full paper~\cite{KSU10} for the type derivation rules.
%%\fi
%\nk{Write typing rules in appendix or supplemenatary material.}

\begin{theorem}
Suppose \(\Gamma\pD e:\sigma\). Then, \(\alphaB(\Gamma)\pS 
e:\alphaB(\sigma)\). Furthermore, there exists \(e'\) such that 
\(\T{\Gamma}{e}{\sigma}{e'}\).
\end{theorem}
\begin{proof}
Straightforward induction on the derivation of \(\Gamma\pD e:\sigma\). 
Note that in the rule for \rn{A-CAdd}, we can choose \(\TRUE\) as 
\(\psi\) and \(\psi'\).
\end{proof}

The following lemma guarantees that the output of the transformation is 
well-typed.
\begin{lemma}
If \(\T{\Gamma}{e_1}{\sigma}{e_2}\), then \(\beta(\Gamma)\pS e_2: 
\beta(\sigma)\).
\end{lemma}
\begin{proof}
Straightforward induction on the derivation of 
\(\T{\Gamma}{e_1}{\sigma}{e_2}\).
\end{proof}

%% is derivable if and only if \(e_1\) and 
%%\(e_2\) respectively have the types \(\alpha(\sigma)\) and 
%%\(\beta(\sigma)\) in the standard simple type system under 
%%\(\alpha(\Gamma)\) and \(\beta(\Gamma)\). Here, \(\alpha(\bullet)\) and 
%%\(\beta(\bullet)\) respectively replace each occurrence of a base 
%%abstraction type \(b[P_1,\dots,P_m]\) in the argument with \(b\) and 
%%\(\underbrace{\TBOOL \times \cdots \times \TBOOL}_m\).
%%\end{theorem}
%\begin{theorem}
%If \(\T{\Gamma}{e_1}{\sigma}{e_2}\), then \(\alpha(\Gamma) \vdash 
%e_1:\alpha(\sigma)\) and \(\beta(\Gamma) \vdash e_2:\beta(\sigma)\) are 
%derivable, where \(\alpha\) is defined by:
%\begin{eqnarray*}
%\alpha(\emptyset)&=&\emptyset \\
%\alpha(\Gamma,x:\sigma)&=&\alpha(\Gamma),x:\alpha(\sigma) \\
%\alpha(\Gamma,f:\sigma)&=&\alpha(\Gamma),f:\alpha(\sigma) \\
%\alpha(x:\sigma_1 \to \sigma_2)&=&\alpha(\sigma_1) \to \alpha(\sigma_2) \\
%\alpha(b[\seq{P}])&=&b
%\end{eqnarray*}
%\(\beta\) is defined by:
%\begin{eqnarray*}
%\beta(\emptyset)&=&\emptyset \\
%\beta(\Gamma,x:\sigma)&=&\beta(\Gamma),x:\beta(\sigma) \\
%\beta(\Gamma,f:\sigma)&=&\beta(\Gamma),f:\beta(\sigma) \\
%\beta(x:\sigma_1 \to \sigma_2)&=&\beta(\sigma_1) \to \beta(\sigma_2) \\
%\beta(b[P_1,\dots,P_m])&=&\underbrace{\TBOOL \times \cdots \times \TBOOL}_m
%\end{eqnarray*}
%\end{theorem}

The theorem below states that our predicate abstraction is sound in the 
sense that if a source program fails, so does its abstraction
\iffull 
(see Appendix~\ref{sec:soundness} for the proof). 
\else
(see \cite{Kobayashi2011} for the proof). 
\fi
Thus, the safety of the abstract program (which is decidable by 
Theorem~\ref{th:decidability}) is a sufficient condition for the safety 
of the source program.
\begin{theorem}[soundness]
\label{thm:soundness} If \(\T{}{D_1}{\Gamma}{D_2}\) and 
\(\textit{main}\tuple{} \redswith{s}{D_1} \FAIL\), then 
\(\textit{main}\tuple{} \redswith{s}{D_2} \FAIL\).
%%Suppose that 
%%\begin{itemize}
%%\item \(\T{}{D_1}{\Gamma}{D_2}\) and 
%%\item \(\T{\Gamma}{e_1}{\sigma}{e_2}\).
%%\end{itemize}
%%If \(e_1 \redswith{s}{D_1} \FAIL\), then \(e_2 \redswith{s}{D_2} \FAIL\).
\end{theorem}
%%\koba{Using \(\Rightarrow\) for the translation and \(\Longrightarrow\) for
%%reductions seem confusing. Perhaps use \(\leadsto\) for the former.}

The theorem above says that the abstraction is sound but not how good 
the abstraction is. We compare below the verification power of the 
combination of predicate abstraction and higher-order model checking 
with the dependent intersection type system given in 
\iffull 
Appendix~\ref{sec:dit}, 
\else
\cite{Kobayashi2011},
\fi 
which is essentially equivalent to the one in \cite{Terauchi2010}.

We write \(\B[\psi_1,\ldots,\psi_k]\) for the set of formulas 
constructed from \(\psi_1,\ldots,\psi_k\) and boolean operators (\(\TRUE, 
\FALSE, \land, \lor, \neg\)). For an abstraction type \(\sigma\), the 
set \(\depty(\sigma)\) of dependent types is:
\[
\begin{array}{l}
\depty(b[P_1,\ldots,P_n]) =
    \set{\rtbase{\dvar}{b}{\psi} \mid \psi \in \B[P_1(\dvar),\ldots,P_n(\dvar)]}\\
\depty(x\COL\sigma_1 \to \sigma_2) = %%\\
%%\qquad    
\set{(\rtfunb{x}{\dtvar_{11}}{\dtvar_{21}})\land \cdots \land (\rtfunb{x}{\dtvar_{1m}}{\dtvar_{2m}})\\
\qquad\quad
  \mid \delta_{11},\ldots,\delta_{1m}\in \depty(\sigma_1), 
  \delta_{21},\ldots,\delta_{2m}\in \depty(\sigma_2)}
\end{array}
\]
We extend \(\depty\) to a map from abstraction type environments to the 
powerset of dependent type environments by:
\[
\begin{array}{l}
 \depty(\set{x_1\COL\sigma_1,\ldots,x_n\COL\sigma_n}) =\\
    \set{\set{x_1\COL\dtvar_1,\ldots,x_n\COL\dtvar_n} \mid 
\dtvar_i \in \depty(\sigma_i)\mbox{ for each $i\in\set{1,\ldots,n}$}}
\end{array}
\]
The following theorem says that our predicate abstraction (with 
higher-order model checking) has at least the same verification power as 
the dependent intersection type system.

%%\koba{Below, do we need to restrict \(\delta\) to base types?}
\begin{theorem}[relative completeness]
\label{th:rel-complete}
Suppose \(\DPT{}{D}{\Delta}\). %% and \(. %% and \(\DPT{\Delta}{e}{\delta}\).
If \(\Delta \in \depty(\Gamma)\), then %% and \(\delta\in \depty(\sigma)\), then 
there exists \(D'\) such that %%and \(e'\) such that
\(\T{}{D}{\Gamma}{D'}\) and \(\textit{main}\tuple{}\ \mathbin{{\not\!\!\redswith{}{D'}}} \FAIL\).
%%\begin{enumerate}
%%\item \(\T{}{D}{\Gamma}{D'}\),
%%\item \(\T{\Gamma}{e}{\sigma}{e'}\), and
%%\item \(e' \not\redswith{}{D'} \FAIL\).
%%\end{enumerate}
\end{theorem}
\iffull
The proof is given in Appendix~\ref{sec:relcomp}.
\fi
%%%\begin{pfsketch}
%%%\koba{The following is just a semantic intuition.}
%%%For each \(\delta\in \depty(\sigma)\), associate the following denotation
%%%\(\sem{\delta,\sigma}\):
%%%\[
%%%\begin{array}{l}
%%%\sem{\rtbase{\dvar}{b}{C[P_1(\dvar),\ldots,P_n(\dvar)]}, b[P_1,\ldots,P_n]}
%%% = \\
%%%\qquad\set{(c_1,\ldots,c_n) \mid C[c_1,\ldots,c_n] \equiv \TRUE}\\
%%%\sem{\rtfun{\dtvar_{1}}{\dtvar_{2}}, \sigma_1 \to \sigma_2} =\\
%%%\qquad
%%%  \set{g \mid \forall h\in\sem{\dtvar_1,\sigma_1}.g(h)\in\sem{\dtvar_2,\sigma_2}}\\
%%%\sem{\rtfunb{x}{\rtbase{\nu}{b}{C[P_1(\dvar),\ldots,P_k(\dvar)]}}{\dtvar_{2}}, x:b[\seq{P}] \to \sigma_2} =\\
%%%  \set{g \mid \forall (c_1,\ldots,c_k)\in\sem{\rtbase{\nu}{b}{C[P_1(\dvar),\ldots,P_k(\dvar)]},b[\seq{P}]}.\\
%%%\qquad   \forall d\in\set{d \mid P_1(d)=c_1,\ldots,P_k(d)=c_k}.\\
%%%\qquad g(c_1,\ldots,c_k)\in\sem{[d/x]\dtvar_2,[d/x]\sigma_2}}\\
%%%\sem{\dtvar_1\land \dtvar_2, \sigma} =
%%%  \sem{\dtvar_1,\sigma}\cap \sem{\dtvar_2,\sigma}
%%%\end{array}
%%%\]
%%%Then, a term \(e\) of type \(\delta\) can be transformed into a function \(e'\) such that
%%%\(\sem{e'} \in \sem{\dtvar,\sigma}\).
%%%\end{pfsketch}

\begin{remark}
\label{rem:more-expressive-than-deptype} The converse of the above 
theorem does not hold: see 
\iffull 
Appendix~\ref{sec:comparison-with-dit}. 
\else
\cite{Kobayashi2011}.
\fi 
Together with 
Theorem~\ref{th:rel-complete}, this implies that our combination of 
predicate abstraction and higher-order model checking is strictly more 
powerful than the dependent intersection type system.
\end{remark}
%%%Consider the following program:
%%%\begin{verbatim}
%%%let f x y = if (x()>0)&&(y()<=0) then fail else () in
%%%let h x y = x in let g n = f(h n)(h n) in g(randi())
%%%\end{verbatim}
%%%Given the abstraction type environment: 
%%%\[
%%%\begin{array}{l}
%%%f\COL (\TUNIT\ra \INT[\lambda \nu.\nu>0]) \ra (\TUNIT\ra \INT[\lambda \nu.\nu>0]) \ra \TUNIT,\\
%%%  g\COL \INT[\,] \ra \TUNIT,
%%%\end{array}\]
%%%the above program is abstracted to:
%%%\begin{verbatim}
%%%let f x y = if x() && not(y()) then fail else () in
%%%let h x y = x in
%%%let g() = let b = randb() in f (h b) (h b) in g()
%%%\end{verbatim}
%%%and is successfully verified by a higher-order model checker.
%%%The above program is, however, not typable with the corresponding dependent type environment:
%%%\[
%%%\begin{array}{l}
%%%f\COL ((\TUNIT\ra \rtbase{\nu}{\INT}{\nu>0}) \ra (\TUNIT\ra \rtbase{\nu}{\INT}{\nu>0})\ra \TUNIT)\\
%%%\quad\land ((\TUNIT\ra \rtbase{\nu}{\INT}{\nu\leq 0}) \ra (\TUNIT\ra \rtbase{\nu}{\INT}{\nu\leq 0})\ra \TUNIT)\\
%%%g\COL \INT \ra \TUNIT
%%%\end{array}
%%%\]
%%%%%The above program can be verified with the dependent type system by changing the type of \(f\) to
%%%%%\(f\COL (x\COL\rtbase{\nu}{\INT}{\TRUE} \ra \rtbase{\nu}{\INT}{\nu=x}\ra \TUNIT)\).
%%%%%By slightly modifying the program above, however, we can construct a (closed) program that
%%%%%can be verified with our verification method but cannot be verified with the dependent intersection
%%%%%type system (see Appendix~\ref{sec:expressive-power}). 
%%%Together with Theorem~\ref{th:rel-complete}, this implies that
%%%our combination of predicate abstraction and higher-order model checking is strictly more powerful than 
%%%the dependent type system in Figure~\ref{fig:dep_rules}.
%%%\end{remark}
%%%Conversely, the following theorem asserts that
%%%our predicate abstraction 
%%%has no more verification power than the dependent intersection type system.
%%%\koba{Double check that the following theorem does hold.
%%%It is a bit strange as the dependent type does not seem to take the evaluation order
%%%into account..
%%%The dependent type system is sound only for call-by-value?
%%%For example, \((\lambda x.\FAIL)(\textit{loop()})\) seems typable
%%%(if we extend the syntax of applications), by assigning
%%%the type \(\rtbase{x}{\INT}{\bot}\) to \(\textit{loop}()\).
%%%In fact, the rule for \(\FAIL\) seems to assume that functions are strict.}
%%%\begin{theorem}
%%%\label{th:rel-soundness}
%%%Suppose \(\T{}{D}{\Gamma}{D'}\) and \(\textit{main}\tuple{} \not\redswith{}{D'} \FAIL\).
%%%%%\begin{enumerate}
%%%%%\item \(\T{}{D}{\Gamma}{D'}\), 
%%%%%\item \(\T{\Gamma}{e}{\sigma}{e'}\), and
%%%%%\item \(e' \not\redswith{}{D'} \FAIL\).
%%%%%\end{enumerate}
%%%Then, there exists \(\Delta\in \depty(\Gamma)\) %%,\delta\in \depty(\sigma)\) 
%%%such that \(\DPT{}{D}{\Delta}\). %% and \(\DPT{\Delta}{e}{\delta}\).
%%%\end{theorem}
%%%
%%%\begin{pfsketch}
%%%\koba{There may be a direct proof without going through the denotational semantics. The idea below anyway provides
%%%an intuititon why the theorem should hold.}
%%%For each function \(f\) in \(D'\), let \(\sem{f}\) be the set-theoretic denotational semantics of \(f\).
%%%(Actually, if necessary, we can compute this from all the reduction sequences of \(e'\). We may need to be a bit careful
%%%about the issue of definability, if we talk about the denotational semantics.)
%%%Assign to \(f\) the type \(\RecoverDT(\sem{f}, \Gamma(f))\), where
%%%\[
%%%\begin{array}{l}
%%%\RecoverDT((c_1,\ldots,c_k), b[P_1,\ldots,P_k]) =\\ 
%%%\qquad      \rtbase{\nu}{b}{P_1(\nu)^{c_1} \land \cdots \land P_k(\nu)^{c_k}}\\
%%%        \mbox{ (where $\psi^c = \psi$ if \(c=\TRUE\), $\neg\psi$ if \(c=\FALSE\), and \(\FALSE\) if \(c=\bot\))}\\
%%%\RecoverDT(g, x\COL\sigma_1\ra \sigma_2) =\\
%%%  \bigwedge_{h\in \dom(g)} (x\COL \RecoverDT(h,\sigma_1) \ra \RecoverDT(g(h), \sigma_2))\\
%%%\end{array}
%%%\]
%%%Then, \(\Delta = \set{f\COL\RecoverDT(\sem{f},\Gamma(f))\mid f\in\dom(\Gamma)}\) satisfies the
%%%required conditions.
%%%\end{pfsketch}
%%The above theorem states that in order for \(e_1\) not to fail 
%%for any (type-respecting) bindings, it is sufficient for the abstract program \(e_2\) not to fail.
%%Note that the latter is decidable by Theorem~\ref{th:decidability}.


%%\nk{I have moved the following remark here, which should be rewritten to make
%%the point clearer. Especially, include an example for which
%%\((x:\TINT[\,] \to \TINT[\,]) \to y:\TINT[\lambda 
%%y.y=x] \to \TINT[\,]\) is ``crucial''.}


\begin{remark} 
\label{rem:limitation} The well-formedness condition for abstraction 
types is sometimes too restrictive to express a necessary predicate. For 
example, consider the following program.
\begin{verbatim}
let apply f x = f x in let g y z = assert(y=z) in
let k n = apply (g n) n; k(n+1) in  k(0)
\end{verbatim}
In order to verify that the assertion failure does not occur, we need a 
correlation between the argument of \(f\) and \(x\), which cannot be 
expressed by abstraction types. The problem can be avoided either by 
adding a dummy parameter to \texttt{apply} (as \texttt{let apply n f x = 
...}) and using the abstraction type \(n\COL\TINT[\,]\ra (\TINT[\lambda 
\nu.\nu=n]\ra \TUNIT)\ra \TINT[\lambda \nu.\nu=n]\ra \TUNIT\), or by 
swapping the parameters \(f\) and \(x\). A more fundamental solution 
(which is left for future work) would be to introduce \emph{polymorphic} 
abstraction types, like 
%
%\(\forall m\COL\TINT.(\TINT[\lambda \nu.\nu=m]\ra \TUNIT)\ra 
%\TINT[\lambda \nu.\nu=n]\ra n\COL\TINT[\lambda \nu.\nu=m]\ra \TUNIT)\), 
\(\forall m\COL\TINT.(\TINT[\lambda \nu.\nu=m]\ra \TUNIT)\ra 
\TINT[\lambda \nu.\nu=m]\ra \TUNIT\), 
%
and extend the predicate abstraction rules 
accordingly. 
%%The former is a common technique used in dependent types~\cite{Xi1999}, but it is not obvious
%%how to apply this kind of transformation automatically.
\end{remark}
%%%\begin{verbatim}
%%%let f g x = g(x+1) in 
%%%let h z y = assert(y>z) in
%%%let n = rand() in f (h n) n
%%%\end{verbatim}
%%%In order to verify this program, we would need the above (ill-formed) abstraction type,
%%%to express the co-relation between \texttt{g}'s first argument and \texttt{x}.
%%%This problem can be avoided by adding a dummy parameter to \texttt{f}, and
%%%use the abstraction type:
%%%\(z:\ITint{\,} \to (w:\TINT[\lambda w.w<z] \to \TINT[\,]) \to x:\TINT[\lambda x.x\geq z] \to \TINT[\,]\).
%%%This is a common technique used in dependent types~\cite{Xi1999}, but it is not obvious
%%%how to apply this kind of transformation automatically.
%%%%%%\item Functions and quantifers are not allowed to occur in predicates.
%%%%%%This limits 
%%%%%%\end{enumerate}
%%%\qed
%%%\end{remark}
