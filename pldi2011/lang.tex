\section{Language}
\label{sec:lang}

This section introduces a simply-typed, higher-order functional language, 
which is used as the target of our verification method.

We assume a set \(\BT = \set{b_1,\ldots,b_n}\) of data types, and a set 
\(\vset{b_i}\) of constants, ranged over by \(c\), for each data type 
\(b_i\). We also assume that there are operators 
\(\OP:{b_{i_1},\ldots,b_{i_k}\ra b_j}\); we write \(\SEM{\OP}\) for the 
function denoted by \(\OP\). We assume that the set of data types 
includes \(\TUNIT\) with \(\vset{\TUNIT} = \tuple{}\), and \(\TBOOL\) 
with \(\vset{\TBOOL}=\set{\TRUE, \FALSE}\), and that the set of 
operators includes \(=\ :b,b\ra \TBOOL\) for every \(b\in \BT\), and 
boolean connectives such as \(\land:\TBOOL,\TBOOL\ra \TBOOL\).

The syntax of the language is given by:
\begin{eqnarray*}
D \mbox{ (program) }&::=& \set{f_1\ \seq{x}_1 = e_1,\ldots,f_m\ \seq{x}_m = e_m} \\
e \mbox{ (expressions) }
&::=& c \mid x\ \seq{v} \mid f\ \seq{v} \mid \LETEQIN{x}{e_1}{e_2} \mid \OP(\seq{v}) \\
&\mid& \FAIL \mid \ASSUME{v}{e} \mid \PAR{e_1}{e_2} \\ %%\mid \PARB{e_1}{e_2} \\
%%&\mid& (v_1,\dots,v_m) \mid \NTH{i}{v} \\
v \mbox{ (values) }&::=& c \mid x \mid f\ \seq{v} %\\ \mid (v_1,\dots,v_m) 
%%c&::=&n \mid \TRUE \mid \FALSE \mid \cdots \\
%%\OP&::=& + \mid \times \mid \ =\  \mid \NOT \mid \AND \mid \cdots \\
%%T&::=& B \mid T_1 \to T_2 \mid T_1 \times \dots \times T_n \\
%%B&::=& \TINT \mid \TBOOL \mid \cdots \\
%%E&::=& \HOLE \mid \LETEQIN{x}{E}{e}
\end{eqnarray*}

Here, \(\seq{x}\) abbreviates a (possibly empty) sequence 
\(x_1,\ldots,x_n\). In the definition \(f\ \seq{x} = e\), we call the 
length of \(\seq{x}\) the \emph{arity} of \(f\). In the definition of 
values, the length of \(\seq{v}\) in \(f\,\seq{v}\) must be smaller than 
the arity of \(f\). (In other words, \(f\,\seq{v}\) must be a partial 
application.) We assume that every function in \(D\) has a non-zero 
arity, and that \(D\) contains a distinguished function symbol 
\(\textit{main}\in \set{f_1,\ldots,f_m}\) whose simple type is 
\(\TUNIT\ra \TUNIT\).
%%of arity \(1\) that takes the unit value.

Most of the expressions are standard, except the following ones. The 
expression \(\FAIL\) aborts the program and reports a failure. The 
expression \(\ASSUME{v}{e}\) evaluates \(e\) if \(v\) is true; otherwise 
it stops the program (without a failure). The expression 
\(\PAR{e_1}{e_2}\) evaluates \(e_1\) or \(e_2\) in a non-deterministic 
manner. Note that a standard conditional expression 
\(\IFTE{v}{e_1}{e_2}\) can be expressed as: 
\[
 \PAR{(\ASSUME{v}{e_1})}{(\LETEQIN{x}{\NOT v}\ASSUME{x}{e_2})}.
\]
%
We can express the assertion \(\ASSERT{v}\) as 
\(\IFTE{v}{\tuple{}}{\FAIL}\).
%\[
% \PAR{(\ASSUME{v}{\tuple{}})}{(\ASSUME{\neg v}{\FAIL})}.
%\]
%
The random number generator \(\textit{randi}\) used in 
Section~\ref{sec:intro} is defined by:
\[
\begin{array}{l}
\textit{randi}\,\tuple{} = (\textit{randiFrom}\, 1)\PAROP (\textit{randiTo}\, 0)\\
\textit{randiFrom}\,n = n\PAROP (\textit{randiFrom}\,(n+1))\\
\textit{randiTo}\,n = n\PAROP (\textit{randiTo}\,(n-1))\\
\end{array}
\]

We assume that a program is well-typed in the standard simple type 
system, where the set of types is given by:
\[ \tau ::= b_1 \mid \cdots \mid b_n \mid \tau_1\ra \tau_2.\]
%
Furthermore, we assume that the body of each definition has a data type 
\(b_i\), not a function type. This is not a limitation, as we can always 
use the continuation-passing-style (CPS) transformation to transform a 
higher-order program to an equivalent one that satisfies the restriction.
%%%For example, a function definition
%%% f x = e
%%%(where e returns a function)
%%%can be transformed into
%%% f' x k = k e
%%%where k is the continuation parameter. This is indeed applied in our implementation.

%%%\(x\ \seq{v}\) and \(f\ \seq{v}\) are values if the arities of \(x\) and 
%%%\(f\) are less than the length of \(\seq{v}\). We write 
%%%\(\IFTE{e_1}{e_2}{e_3}\) for
%%%\[
%%% \PAR{(\ASSUME{e_1}{e_2})}{(\ASSUME{\NOT e_1}{e_3})}
%%%\]
%%%
%%%\todo{add explanation}

We define the set of \emph{evaluation contexts} by: \(E ::= \HOLE \mid 
\LETEQIN{x}{E}{e}\). The reduction relation is given in 
Figure~\ref{fig:opsem}. We label the reduction relation with 
\(0,1,\epsilon\) to record which branch has been chosen by a 
non-deterministic expression \(\PAR{e_1}{e_2}\); it will be used to 
relate reductions of a source program and an abstract program later in 
Sections~\ref{sec:pred} and \ref{sec:cegar}. We write \(e_1 
\redswith{l_1\cdots l_n}{D} e_2\) if 
\[e_1 (\redwith{\epsilon}{D})^* \redwith{l_1}{D}(\redwith{\epsilon}{D})^* \cdots (\redwith{\epsilon}{D})^* \redwith{l_n}{D}(\redwith{\epsilon}{D})^* e_2.\]
We often omit the subscript \(D\) when it is clear from the context.
%%\todo{add explanation}
The goal of our verification method is to check whether \\
\(\textit{main}\,\tuple{} \redswith{s}{D} \FAIL\).%
%%where \textit{main} is a function symbol with arity \(0\).
\footnote{Thus, we consider the reachability problem for a closed 
program. Note, however, that we can express unknown values by using 
non-determinism (recall \textit{randi}). It is also easy to extend our 
method to deal with more general verification problems, such as resource 
usage verification~\cite{Kobayashi2009}.}

\begin{figure}[t]
\typicallabel{E-Assume}
\infrule[E-App]
  {f\ \seq{x}=e \in D}
  {E[f\ \seq{v}] \stackrel{\epsilon}{\RED}_D E[[\seq{v}/\seq{x}]e]}
\rulesp

\infax[E-Let]
  {E[\LETEQIN{x}{v}{e}] \stackrel{\epsilon}{\RED}_D E[[v/x]e]}
\rulesp

\infax[E-Op]
  {E[\OP(\seq{c})] \stackrel{\epsilon}{\RED}_D E[\SEM{\OP}(\seq{c})]}
\rulesp

\infax[E-Par]
  {E[\PAR{e_0}{e_1}] \stackrel{i}{\RED}_D E[e_i]}
\rulesp

%%%\infax[E-Par-Right]
%%%  {\PAR{e_1}{e_2} \stackrel{1}{\RED}_D e_2}
%%%\rulesp
%%%
%%%\infax[E-Par2-Left]
%%%  {\PARB{e_1}{e_2} \stackrel{\epsilon}{\RED}_D e_1}
%%%\rulesp
%%%
%%%\infax[E-Par2-Right]
%%%  {\PARB{e_1}{e_2} \stackrel{\epsilon}{\RED}_D e_2}
%%%\rulesp

\infax[E-Assume]
  {E[\ASSUME{\TRUE}{e}] \stackrel{\epsilon}{\RED}_D E[e]}
\rulesp

%%\infax[E-Nth]
%%  {\NTH{i}{(v_1,\dots,v_m)} \stackrel{\epsilon}{\RED}_D v_i}
%%\rulesp

%\infrule[E-Ctx]
%  {e_1 \stackrel{l}{\RED}_D e_2}
%  {E[e_1] \stackrel{l}{\RED}_D E[e_2]}
%\rulesp

\infax[E-Fail]
  {E[\FAIL] \stackrel{\epsilon}{\RED}_D \FAIL}
%\rulesp

%%%\infax
%%%  {e \stackrel{\epsilon}{\RED^*}_D e}
%%%\rulesp
%%%
%%%\infrule
%%%  {e \stackrel{\pi_1}{\RED^*}_D e'' \andalso
%%%   e'' \stackrel{\pi_2}{\RED}_D e'}
%%%  {e \stackrel{\pi_1 \cdot \pi_2}{\RED^*}_D e'}
%%%\rulesp

\caption{Call-by-Value Operational Semantics}
\label{fig:opsem}
\end{figure}

