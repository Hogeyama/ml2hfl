%\newcommand\unno[1]{\textbf{[#1 -unno]}}
%\newcommand{\todo}[1]{***ToDo: #1***}
%\newcommand\finish[1]{\textbf{[#1]}}
%\newcommand\koba[1]{\textcolor{red}{[#1 -koba]}}

\newcommand\rulesp{\vspace*{2ex}}

\newcommand\END{\mbox{$\square$}}
\newenvironment{pfof}[1]{\paragraph{Proof of {#1}}}{\END}

\newtheorem{theorem}{Theorem}[section]
\newtheorem{lemma}[theorem]{Lemma}
\newtheorem{prop}[theorem]{Proposition}
\newtheorem{conj}[theorem]{Conjecture}
\newtheorem*{cor}{Corollary}

\theoremstyle{definition}
\newtheorem{definition}{Definition}[section]

\theoremstyle{definition}
\newtheorem{example}{Example}[section]

\theoremstyle{remark}
\newtheorem{remark}{Remark}


\newcommand{\COERCE}[1]{\mathit{coerce}(#1)}
\newcommand{\NTH}[2]{\sharp_{#1}#2}

\newcommand{\seq}[1]{\widetilde{#1}}
\newcommand{\set}[1]{\{#1\}}

\newcommand{\rt}[3]{\{#1:#2 \mid #3\}}

\newcommand{\RED}{\longrightarrow}
%\newcommand{\EQ}{\approx}

\newcommand{\BASE}[1]{|#1|}

\newcommand{\TRUE}{\mathtt{true}}
\newcommand{\FALSE}{\mathtt{false}}

\newcommand{\INC}{\mathit{inc}}

\newcommand{\FUN}[3]{\mathtt{fun}(#1,#2,#3)}
\newcommand{\IFTE}[3]{\mathtt{if~}#1\mathtt{~then~}#2\mathtt{~else~}#3}
\newcommand{\IFZ}[3]{\mathtt{if0~}#1\mathtt{~then~}#2\mathtt{~else~}#3}
\newcommand{\FAIL}{\mathtt{fail}}
\newcommand{\LET}{\mathtt{let}}
\newcommand{\IN}{\mathtt{in}}
\newcommand{\LETRECEQ}[2]{\mathtt{let~rec~}#1=#2}
\newcommand{\LETEQIN}[2]{\mathtt{let~}#1=#2\mathtt{~in~}}
\newcommand{\LETRECEQIN}[3]{\mathtt{let~rec~}#1=#2\mathtt{~in~}#3}
\newcommand{\T}[4]{#1 \vdash #2 : #3 \leadsto #4}
\newcommand{\TT}[4]{#1 \vdash #2 : #3 \hookrightarrow #4}
\newcommand{\DT}[3]{#1 \pD #2 : #3}
\newcommand{\DPT}[3]{#1 \vdash_{\texttt{DIT}} #2 : #3}
\newcommand{\R}[3]{#1 \vdash #2 : #3}
\newcommand{\SUB}[4]{#1 \vdash #2 : #3 \leq #4}
\newcommand{\IMPLY}{\Rightarrow}
\newcommand{\AND}{\land}
\newcommand{\OR}{\lor}
\newcommand{\NOT}{\neg}
\newcommand{\OP}{\mathrm{op}}
\newcommand{\SEM}[1]{[\![#1]\!]}
\newcommand{\TINT}{\mathtt{int}}
%%\newcommand{\TUNIT}{\mathtt{unit}}
\newcommand{\TUNIT}{\star}
\newcommand{\TBOOL}{\mathtt{bool}}
\newcommand{\FV}[1]{\mathrm{FV}(#1)}
\newcommand{\FL}[1]{\mathrm{FL}(#1)}
\newcommand{\DOM}[1]{\mathrm{dom}(#1)}
%\newcommand{\ASSERT}[1]{\mathtt{assert~}#1}
\newcommand{\ASSERT}{\mathtt{assert~}}
%%\newcommand{\ASSUME}[2]{\mathtt{assume~}#1\mathtt{~in~}#2}
\newcommand{\ASSUME}[2]{\mathtt{assume~}#1; #2}
%\newcommand{\ASSUME}[2]{[#1] \Rightarrow #2}
\newcommand{\PAR}[2]{#1\ \square\ #2}
\newcommand{\PARB}[2]{#1\ \blacksquare\ #2}
\newcommand{\HOLE}{[\ ]}
\newcommand{\DUP}[1]{\mathit{dup}(#1)}
\newcommand{\DEF}{\equiv}
\newcommand{\THE}[1]{\theta_{#1}}
%\newcommand{\CHOOSE}[2]{\mathrm{choose(#1,#2)}}
%\newcommand{\RET}[2]{\mathtt{ret}_{#1}(#2)}

%\newcommand{\TYPE}[3]{\mathit{type}(#1,#2,#3)}
%\newcommand{\TYPE}[2]{\mathit{type}(#1,#2)}

%\newcommand{\m}[1]{\stackrel{#1}{\mapsto}}
%\newcommand{\m}[1]{\stackrel{#1}{=}}

%\newcommand{\V}[3]{\langle #1 \rangle^{#2}_{#3}}

%\newcommand{\NAIVE}[1]{\mathcal{C}_{\mathit{naive}}(#1)}
%\newcommand{\COMPLEX}[1]{\mathcal{C}_{\mathit{complex}}(#1)}
%\newcommand{\INTERP}[2]{\mathit{interp}(#1,#2)}
%\newcommand{\REFINE}[2]{\mathit{refine}(#1,#2)}
%\newcommand{\DV}[1]{\mathit{dvs}(#1)}

%\newcommand{\COND}[2]{\mathcal{C}(#1,#2)}

\newcommand{\FRESH}{\mathrm{fresh}}
\newcommand{\X}[2]{[#1]_{#2}}
\newcommand{\Y}[2]{\langle #1 \rangle_{#2}}

\newcommand{\UNDUP}[1]{\mathtt{undup}(#1)}
\newcommand{\MERGE}[1]{\mathrm{merge}(#1)}


\newcommand\MAX{\mathit{max}}
\newcommand\comp[2]{#1[#2]}
\newcommand\hole{[\,]}
\newcommand\DtoS{\texttt{D2S}}
\newcommand\IFF{\Leftrightarrow}
\newcommand\snd{\mathtt{snd}}
\newcommand\LIFT{\mathtt{lift}}
\newcommand\stsem[1]{\sem{\texttt{st}(#1)}}
\newcommand\absval[1]{\alpha_{#1}}
\newcommand\dtsem[1]{\sem{#1}}
\newcommand\nondet{\texttt{*}}
\newcommand\restrict[2]{#1\mathbin{\downarrow}_{#2}}
\newcommand\st{\gamma}
\newcommand\env{\rho}
\newcommand\esem[2]{\sem{#1}_{#2}}
\newcommand\ST{\texttt{ST}}
\newcommand\BOOLF{\TBOOL_{\texttt{fail}}}
\newcommand\PSet[1]{2^{#1}}
\newcommand\dtI{\bigwedge}
\newcommand\subT{\leq}
\newcommand\DPSUB[1]{#1 \p_{\texttt{DIT}}}
\newcommand\rtbase[3]{\set{#1\COL #2\mid #3}}
\newcommand\rtfunb[2]{#1\COL{#2}\rightarrow}
\newcommand\rtfun[1]{#1\rightarrow}
\newcommand\INT{\texttt{int}}
\newcommand\itlub{\sqcup}
\newcommand\imply{\Rightarrow}
\newcommand\PAROP{\square}
\newcommand\PAROPB{\blacksquare}
\newcommand\reds{\redswith{}{}}
\newcommand\mkSHP{\textbf{SHP}}
\newcommand\Copy[2]{#1^{(#2)}}
\newcommand\Dup[2]{{#1}^{\flat_{#2}}}
\newcommand\CALL{a}
\newcommand\pDT{\p_{\texttt{DT}}}

\newcommand{\IF}[2]{\mathtt{if~}#1\mathtt{~then~}#2\mathtt{~else~}}

\newcommand\rname{\rn}
\newenvironment{pfsketch}{\paragraph{Proof sketch}}{\END}
\newcommand\BT{\mathbf{B}}
\newcommand\nk[1]{{\footnotesize \color{red}[#1 - \textbf{Naoki}]}}
%%\newcommand\nk[1]{}
\newcommand\alphaB{\mathtt{A2S}}
\newcommand\pS{\vdash_{\mathtt{ST}}}
\newcommand\pD{\vdash_{\mathtt{AT}}}
\newcommand\pW{\vdash_{\mathtt{wf}}}
\newcommand\trecs[1]{\textsc{TRecS}}
\newcommand\tuple[1]{\langle{#1}\rangle}
\newcommand\vset{\SEM}
\newcommand\ITint[1]{\TINT[#1]}
\newcommand\red{\longrightarrow}
\newcommand\redswith[2]{\stackrel{#1}{\Longrightarrow}_{#2}}
\newcommand\redwith[2]{\stackrel{#1}{\longrightarrow}_{#2}}
\newcommand\ra{\rightarrow}
\newcommand\COL{\mathbin{:}}
\newcommand\p{\vdash}

\newcommand\depty{\textit{DepTy}}
\newcommand\RecoverDT{\textit{SynDT}}
\newcommand\B{\mathcal{B}}
\newcommand\dvar{\nu}
\newcommand\dtvar{\delta}
\newcommand\sem[1]{\mathbin{[\![}#1\mathbin{]\!]}}
\newcommand\dom{\textit{dom}}

\newcommand\slice[1]{\stackrel{#1}{\triangleright}}
%\newcommand\REF[1]{\mathrm{ExtractPred}(#1)}

%%% inference rules
\newcommand\Infrule[2]{\begin{minipage}{0.4\columnwidth}\infrule{#1}{#2}\end{minipage}}
\newcommand\InfruleW[3]{\begin{minipage}{#1\columnwidth}\infrule{#2}{#3}\end{minipage}}
\newcommand\InfruleS[3]{\begin{minipage}{#1\textwidth}\infrule{#2}{#3}\end{minipage}}
\newcommand\InfruleSR[4]{\begin{minipage}{#1\textwidth}\infrule[#2]{#3}{#4}\end{minipage}}

\newcommand\Tlist{\texttt{ilist}}
\newcommand\RefAT{\textit{Refine}}


\section{Predicate Discovery}

In this section, we propose a new predicate discovery method for 
higher-order programs.  The method is based on our previous one used by 
MoCHi~\cite{KobayashiPLDI2011}.  We briefly overview it below.

In MoCHi, predicates for abstracting each term of a given program are 
specified as a kind of dependent types called abstraction types.  MoCHi 
infers abstraction types automatically in a counterexample-guided manner: 
 In a CEGAR iteration of MoCHi, if the predicate abstraction at that 
point is not precise enough to show the safety of the original program, 
an error path of the abstracted program is returned as a result of 
higher-order model checking (see Fig.~\ref{fig:cegar}).  If the abstract 
error path is infeasible (i.e., not a genuine path of the original 
program), MoCHi generates a straightline higher-order program (SHP) 
which is safe if and only if the abstract error path is infeasible .  
MoCHi then uses an existing method~\cite{Unno2009} to infer refinement 
types that witness the safety of the SHP.  Finally, MoCHi extracts 
abstraction types from the refinement types, which contain precise 
enough predicates to refute the infeasible abstract error path.

Our new predicate discovery method is also based on the above framework 
for abstraction type inference but we extend the component for 
refinement type inference~\cite{Unno2009} so that it can consider 
information from multiple calling contexts in multiple abstract error 
paths.  This enables MoCHi to infer a general refinement type that 
type-checks the multiple calling contexts, while preserving path- and 
context- sensitivity.  Our new refinement type inference method consists 
of two steps: constraint generation and solving, which are respectively 
explained in Sections~\ref{sec:cg} and \ref{sec:cs}.

\subsection{Constraint Generation}
\label{sec:cg}

%%%\begin{figure*}[tbh]
%%%\begin{eqnarray*}
%%%\CG{\Gamma}{x}
%%%&=&(\reftype{u}{u = x},\top) \quad (\mbox{if}~\sty{x}=\inttype) \\
%%%\CG{\Gamma}{\kappa}
%%%&=&(\Gamma(\kappa),\top) \quad (\mbox{if}~\sty{\kappa} \in \rightarrow) \\
%%%\CG{\Gamma}{c}
%%%&=&(\cty{c},\top) \\
%%%\CG{\Gamma}{\ttlet{x}{e_1}{e_2}}
%%%&=&\mbox{let~}(\sigma,\theta_1)=\CG{\Gamma}{e_1} \\
%%%& &\mbox{let~}(\cpstype,\theta_2)=\CG{\Gamma,x\smallcolon\sigma}{e_2} \\
%%%& &(\cpstype,\theta_1 \land \theta_2) \\
%%%\CG{\Gamma}{\ttapp{e}{x}}
%%%&=&\mbox{let~}(\funtype{y}{\sigma}{\tau},\theta_1)=\CG{\Gamma}{e} \\
%%%& &\mbox{let~}(\sigma',\theta_2)=\CG{\Gamma}{x} \\
%%%& &(\tau[x/y],\theta_1 \land \theta_2 \land \CS{\Gamma}{\sigma'}{\sigma}) \\
%%%\CG{\Gamma}{\ttifndet{e_1}{e_2}}
%%%&=&\mbox{let~}(\cpstype,\theta_1)=\CG{\Gamma}{e_1} \\
%%%& &\mbox{let~}(\cpstype,\theta_2)=\CG{\Gamma}{e_2} \\
%%%& &(\cpstype,\theta_1 \land \theta_2) \\
%%%\CS{\Gamma}{\cpstype}{\cpstype}
%%%&=&\top \\
%%%\CS{\Gamma}{\funtype{x}{\sigma_1}{\tau_1}}{\funtype{x}{\sigma_2}{\tau_2}}
%%%&=&\CS{\Gamma}{\sigma_2}{\sigma_1} \land \CS{\Gamma,x:\sigma_2}{\tau_1}{\tau_2} \\
%%%\CS{\Gamma}{\reftype{u}{\theta_1}}{\reftype{u}{\theta_2}}
%%%&=&\forall u.(\sembrack{\Gamma} \wedge \theta_1) \Rightarrow \theta_2  \quad (\mbox{if}~u \notin \free{\sembrack{\Gamma}})
%%%\end{eqnarray*}
%%%\caption{Constraint generation algorithm.}
%%%\label{fig:cgen}
%%%\end{figure*}

In this section, we extend a constraint generation algorithm used in 
previous work~\cite{Unno2009,Terauchi2010,KobayashiPLDI2011}.  Thus, we 
first briefly overview the previous algorithm.  Let us assume that we 
are given a SHP \(d\) which is typable if and only if the abstract error 
trace is infeasible.  From \(d\), we generate Horn-clause-like 
constraints which are satisfiable if and only if \(d\) is typable.  To 
this end, for each function, we prepare a refinement type template with 
predicate variables, which act as placeholders of refinement predicates 
to be inferred.  We then generate a typing derivation of \(d\) under the 
type environment that associates each function with its type template.  
Horn-clause-like constraints on the predicate variables are then 
extracted from the derivation.  Since the SHP $d$ is recursion-free and 
linear (i.e., each function is called exactly once), generated 
constraints are non-recursive.  \todo{examples}

We extend the above algorithm as follows.
\begin{itemize}
\item Instead of a single infeasible error path, we consider multiple 
paths for predicate discovery:  We keep the set 
\(\set{\pi_1,\dot,\pi_n}\) of infeasible error paths found so far, 
generate the set \(C_i\) of Horn clauses for each path \(\pi_i\), and 
use \(C_1 \cup \dots \cup C_n\) for predicate discovery.
\item In addition to generating Horn-clause-like constraints, we obtain 
an equivalence relation \(E\) on the predicate variables such that \(P\ 
E\ Q\) if and only if the predicate variables \(P\) and \(Q\) represent 
(possibly different) refinement predicates for the same argument of the 
same function in the original program.  \todo{an example}  The 
equivalence relation \(E\) is used to find general solutions for the 
constraints.
\end{itemize}

%%%
%%%We then use the constraint generation algorithm for terms defined in 
%%%Figure~\ref{fig:cgen} to obtain constraints.  In the figure, 
%%%$\CG{\Gamma}{e}$ returns a pair of a refinement type $\tau$ and a 
%%%constraint $\theta$ such that $\Gamma \vdash e : \tau$ is derivable if 
%%%and only if $\theta$ is valid. Similarly, $\CS{\Gamma}{\tau_1}{\tau_2}$ 
%%%returns a constraint $\theta$ such that $\Gamma \vdash \tau_1 \leq 
%%%\tau_2$ is derivable if and only if $\theta$ is valid.
%%%%
%%%The algorithm is almost a straightforward modification of the typing and 
%%%subtyping rules in Section~\ref{sec:reftypesystem} except that the 
%%%application of the subsumption rule is restricted to the acutal argument 
%%%of function applications.




\subsection{Constraint Solving}
\label{sec:cs}

\todo{explain a high-level idea of the algorithm.} The constraint 
solving proceeds as follows.  First, we compute the lower \(\lambda 
\seq{x}.\phi_P\) and the upper \(\lambda \seq{x}.\phi_P'\) bounds for 
each predicate variable \(P\) in the constraints (where \(\FV{\phi_P} 
\subseteq \set{\seq{x}}\) and \(\FV{\phi_P'} \subseteq \set{\seq{x}}\)). 
 The computation of the upper bound for some predicate variable may 
possibly fail because \todo{}.  Let \(S\) be the set of predicate 
variables whose upper bounds were successfully computed.  We then pick 
an equivalence class \(S_0 \in S / E\) (e.g., the largest one), and find 
\(S_1\dots,S_n\) such that:
\begin{itemize}
\item \(S_0 = S_1 \cup \dots \cup S_n\),
\item for each \(i \in \set{1,\dots,n}\), if \(S_i = 
\set{Q_1,\dots,Q_{\ell}}\), then \(\phi_{Q_1} \lor \dots \lor 
\phi_{Q_{\ell}}\) implies \(\phi_{Q_1}' \land \dots \land 
\phi_{Q_{\ell}}'\), and
\item for any \(i,j \in \set{1,\dots,n}\), if \(i \neq j\) and 
\(S_i \cup S_j = \set{Q_1,\dots,Q_{\ell}}\), then \(\phi_{Q_1} \lor 
\dots \lor \phi_{Q_{\ell}}\) does not imply \(\phi_{Q_1}' \land \dots 
\land \phi_{Q_{\ell}}'\).
\end{itemize}
We then pick some \(S_i = \set{Q_1,\dots,Q_{\ell}}\) (e.g., the largest 
one) and find a formula \(\phi\) such that:
\begin{itemize}
\item \(\phi_{Q_1} \lor \dots \lor \phi_{Q_{\ell}}\) implies \(\phi\),
\item \(\phi\) implies \(\phi_{Q_1}' \land \dots \land \phi_{Q_{\ell}}'\), and
\item \(\FV{\phi} \subseteq \set{\seq{x}}\).
\end{itemize}
%Terauchi2010
The problem of finding such a formula \(\phi\) is a well-known logical 
interpolation problem~\cite{}.  In other words, such a formula \(\phi\) 
is obtained as an interpolant \(\mathcal{I}(\phi_{Q_1} \lor \dots \lor 
\phi_{Q_{\ell}},\phi_{Q_1}' \land \dots \land \phi_{Q_{\ell}}')\), where 
an interpolant \(\mathcal{I}(\phi_1,\phi_2)\) of \(\phi_1\) and 
\(\phi_2\) (such that \(\phi_1\) implies \(\phi_2\)) is a formula 
\(\phi\) that satisfies the following conditions:
\begin{itemize}
\item \(\phi_1\) implies \(\phi\),
\item \(\phi\) implies \(\phi_2\), and
\item \(\FV{\phi} \subseteq \FV{\phi_1} \cap \FV{\phi_2}\).
\end{itemize}
However, the conditions for interpolants are not sufficient for our 
purpose: We want to obtain as simple interpolant as possible with 
respect to the number of disjunctions and conjunctions.  To this end, we 
propose a new operator \(\mathcal{J}\) that combines the convex 
hull~\cite{} and the logical interpolation~\cite{} \(\mathcal{I}\) 
operators~\footnote{\todo{explain interpolant is not unique}}.  Let us 
write \(\mathcal{H}(\phi)\) to denote the convex hull of \(\phi\).  
Given \(\phi_1\) and \(\phi_2\) such that \(\phi_1\) implies \(\phi_2\), 
the new operator \(\mathcal{J}(\phi_1,\phi_2)\) is defined as follows:
\begin{eqnarray*}
\mathcal{J}(\phi_1,\phi_2) =
\left\{
\begin{array}{ll}
\mathcal{I}(\mathcal{H}(\phi_1),\neg \mathcal{H}(\neg \phi_2)) & (\mbox{if~}\mathcal{H}(\phi_1) \Rightarrow \neg \mathcal{H}(\neg \phi_2)) \\
\mathcal{I}(\phi_1,\phi_2) & (\mbox{otherwise})
\end{array}
\right.
\end{eqnarray*}
\todo{explain why it works better in most cases}

%%%The refinement type inference method~\cite{Unno2009} relies on a 
%%%technique called interpolation~\cite{Henzinger2004,McMillan2005} from 
%%%automated theorem proving to infer a refinement type of each 
%%%subexpression \(e\) of an ordinary ML type \(\tau\):  The method first 
%%%computes the strongest condition \(\phi_{post}\) on the value \(\nu\) of 
%%%\(e\) and the weakest condition \(\phi_{pre}\) on the value \(\nu\) 
%%%which is required by the context of \(e\).  The method then computes an 
%%%interpolant \(\phi\) of \(\phi_{post}\) and \(\phi_{pre}\), and returns 
%%%\(\set{\nu:\tau \mid \phi}\) as a refinement type of \(e\).
%
%%%%Thus, the method considers both forward and backward information of \(e\) 
%%%%respectively obtained from \(e\) and the context of \(e\).
