%\newcommand\unno[1]{\textbf{[#1 -unno]}}
%\newcommand{\todo}[1]{***ToDo: #1***}
%\newcommand\finish[1]{\textbf{[#1]}}
%\newcommand\koba[1]{\textcolor{red}{[#1 -koba]}}

\newcommand\rulesp{\vspace*{2ex}}

\newcommand\END{\mbox{$\square$}}
\newenvironment{pfof}[1]{\paragraph{Proof of {#1}}}{\END}

\newtheorem{theorem}{Theorem}[section]
\newtheorem{lemma}[theorem]{Lemma}
\newtheorem{prop}[theorem]{Proposition}
\newtheorem{conj}[theorem]{Conjecture}
\newtheorem*{cor}{Corollary}

\theoremstyle{definition}
\newtheorem{definition}{Definition}[section]

\theoremstyle{definition}
\newtheorem{example}{Example}[section]

\theoremstyle{remark}
\newtheorem{remark}{Remark}


\newcommand{\COERCE}[1]{\mathit{coerce}(#1)}
\newcommand{\NTH}[2]{\sharp_{#1}#2}

\newcommand{\seq}[1]{\widetilde{#1}}
\newcommand{\set}[1]{\{#1\}}

\newcommand{\rt}[3]{\{#1:#2 \mid #3\}}

\newcommand{\RED}{\longrightarrow}
%\newcommand{\EQ}{\approx}

\newcommand{\BASE}[1]{|#1|}

\newcommand{\TRUE}{\mathtt{true}}
\newcommand{\FALSE}{\mathtt{false}}

\newcommand{\INC}{\mathit{inc}}

\newcommand{\FUN}[3]{\mathtt{fun}(#1,#2,#3)}
\newcommand{\IFTE}[3]{\mathtt{if~}#1\mathtt{~then~}#2\mathtt{~else~}#3}
\newcommand{\IFZ}[3]{\mathtt{if0~}#1\mathtt{~then~}#2\mathtt{~else~}#3}
\newcommand{\FAIL}{\mathtt{fail}}
\newcommand{\LET}{\mathtt{let}}
\newcommand{\IN}{\mathtt{in}}
\newcommand{\LETRECEQ}[2]{\mathtt{let~rec~}#1=#2}
\newcommand{\LETEQIN}[2]{\mathtt{let~}#1=#2\mathtt{~in~}}
\newcommand{\LETRECEQIN}[3]{\mathtt{let~rec~}#1=#2\mathtt{~in~}#3}
\newcommand{\T}[4]{#1 \vdash #2 : #3 \leadsto #4}
\newcommand{\TT}[4]{#1 \vdash #2 : #3 \hookrightarrow #4}
\newcommand{\DT}[3]{#1 \pD #2 : #3}
\newcommand{\DPT}[3]{#1 \vdash_{\texttt{DIT}} #2 : #3}
\newcommand{\R}[3]{#1 \vdash #2 : #3}
\newcommand{\SUB}[4]{#1 \vdash #2 : #3 \leq #4}
\newcommand{\IMPLY}{\Rightarrow}
\newcommand{\AND}{\land}
\newcommand{\OR}{\lor}
\newcommand{\NOT}{\neg}
\newcommand{\OP}{\mathrm{op}}
\newcommand{\SEM}[1]{[\![#1]\!]}
\newcommand{\TINT}{\mathtt{int}}
%%\newcommand{\TUNIT}{\mathtt{unit}}
\newcommand{\TUNIT}{\star}
\newcommand{\TBOOL}{\mathtt{bool}}
\newcommand{\FV}[1]{\mathrm{FV}(#1)}
\newcommand{\FL}[1]{\mathrm{FL}(#1)}
\newcommand{\DOM}[1]{\mathrm{dom}(#1)}
%\newcommand{\ASSERT}[1]{\mathtt{assert~}#1}
\newcommand{\ASSERT}{\mathtt{assert~}}
%%\newcommand{\ASSUME}[2]{\mathtt{assume~}#1\mathtt{~in~}#2}
\newcommand{\ASSUME}[2]{\mathtt{assume~}#1; #2}
%\newcommand{\ASSUME}[2]{[#1] \Rightarrow #2}
\newcommand{\PAR}[2]{#1\ \square\ #2}
\newcommand{\PARB}[2]{#1\ \blacksquare\ #2}
\newcommand{\HOLE}{[\ ]}
\newcommand{\DUP}[1]{\mathit{dup}(#1)}
\newcommand{\DEF}{\equiv}
\newcommand{\THE}[1]{\theta_{#1}}
%\newcommand{\CHOOSE}[2]{\mathrm{choose(#1,#2)}}
%\newcommand{\RET}[2]{\mathtt{ret}_{#1}(#2)}

%\newcommand{\TYPE}[3]{\mathit{type}(#1,#2,#3)}
%\newcommand{\TYPE}[2]{\mathit{type}(#1,#2)}

%\newcommand{\m}[1]{\stackrel{#1}{\mapsto}}
%\newcommand{\m}[1]{\stackrel{#1}{=}}

%\newcommand{\V}[3]{\langle #1 \rangle^{#2}_{#3}}

%\newcommand{\NAIVE}[1]{\mathcal{C}_{\mathit{naive}}(#1)}
%\newcommand{\COMPLEX}[1]{\mathcal{C}_{\mathit{complex}}(#1)}
%\newcommand{\INTERP}[2]{\mathit{interp}(#1,#2)}
%\newcommand{\REFINE}[2]{\mathit{refine}(#1,#2)}
%\newcommand{\DV}[1]{\mathit{dvs}(#1)}

%\newcommand{\COND}[2]{\mathcal{C}(#1,#2)}

\newcommand{\FRESH}{\mathrm{fresh}}
\newcommand{\X}[2]{[#1]_{#2}}
\newcommand{\Y}[2]{\langle #1 \rangle_{#2}}

\newcommand{\UNDUP}[1]{\mathtt{undup}(#1)}
\newcommand{\MERGE}[1]{\mathrm{merge}(#1)}


\newcommand\MAX{\mathit{max}}
\newcommand\comp[2]{#1[#2]}
\newcommand\hole{[\,]}
\newcommand\DtoS{\texttt{D2S}}
\newcommand\IFF{\Leftrightarrow}
\newcommand\snd{\mathtt{snd}}
\newcommand\LIFT{\mathtt{lift}}
\newcommand\stsem[1]{\sem{\texttt{st}(#1)}}
\newcommand\absval[1]{\alpha_{#1}}
\newcommand\dtsem[1]{\sem{#1}}
\newcommand\nondet{\texttt{*}}
\newcommand\restrict[2]{#1\mathbin{\downarrow}_{#2}}
\newcommand\st{\gamma}
\newcommand\env{\rho}
\newcommand\esem[2]{\sem{#1}_{#2}}
\newcommand\ST{\texttt{ST}}
\newcommand\BOOLF{\TBOOL_{\texttt{fail}}}
\newcommand\PSet[1]{2^{#1}}
\newcommand\dtI{\bigwedge}
\newcommand\subT{\leq}
\newcommand\DPSUB[1]{#1 \p_{\texttt{DIT}}}
\newcommand\rtbase[3]{\set{#1\COL #2\mid #3}}
\newcommand\rtfunb[2]{#1\COL{#2}\rightarrow}
\newcommand\rtfun[1]{#1\rightarrow}
\newcommand\INT{\texttt{int}}
\newcommand\itlub{\sqcup}
\newcommand\imply{\Rightarrow}
\newcommand\PAROP{\square}
\newcommand\PAROPB{\blacksquare}
\newcommand\reds{\redswith{}{}}
\newcommand\mkSHP{\textbf{SHP}}
\newcommand\Copy[2]{#1^{(#2)}}
\newcommand\Dup[2]{{#1}^{\flat_{#2}}}
\newcommand\CALL{a}
\newcommand\pDT{\p_{\texttt{DT}}}

\newcommand{\IF}[2]{\mathtt{if~}#1\mathtt{~then~}#2\mathtt{~else~}}

\newcommand\rname{\rn}
\newenvironment{pfsketch}{\paragraph{Proof sketch}}{\END}
\newcommand\BT{\mathbf{B}}
\newcommand\nk[1]{{\footnotesize \color{red}[#1 - \textbf{Naoki}]}}
%%\newcommand\nk[1]{}
\newcommand\alphaB{\mathtt{A2S}}
\newcommand\pS{\vdash_{\mathtt{ST}}}
\newcommand\pD{\vdash_{\mathtt{AT}}}
\newcommand\pW{\vdash_{\mathtt{wf}}}
\newcommand\trecs[1]{\textsc{TRecS}}
\newcommand\tuple[1]{\langle{#1}\rangle}
\newcommand\vset{\SEM}
\newcommand\ITint[1]{\TINT[#1]}
\newcommand\red{\longrightarrow}
\newcommand\redswith[2]{\stackrel{#1}{\Longrightarrow}_{#2}}
\newcommand\redwith[2]{\stackrel{#1}{\longrightarrow}_{#2}}
\newcommand\ra{\rightarrow}
\newcommand\COL{\mathbin{:}}
\newcommand\p{\vdash}

\newcommand\depty{\textit{DepTy}}
\newcommand\RecoverDT{\textit{SynDT}}
\newcommand\B{\mathcal{B}}
\newcommand\dvar{\nu}
\newcommand\dtvar{\delta}
\newcommand\sem[1]{\mathbin{[\![}#1\mathbin{]\!]}}
\newcommand\dom{\textit{dom}}

\newcommand\slice[1]{\stackrel{#1}{\triangleright}}
%\newcommand\REF[1]{\mathrm{ExtractPred}(#1)}

%%% inference rules
\newcommand\Infrule[2]{\begin{minipage}{0.4\columnwidth}\infrule{#1}{#2}\end{minipage}}
\newcommand\InfruleW[3]{\begin{minipage}{#1\columnwidth}\infrule{#2}{#3}\end{minipage}}
\newcommand\InfruleS[3]{\begin{minipage}{#1\textwidth}\infrule{#2}{#3}\end{minipage}}
\newcommand\InfruleSR[4]{\begin{minipage}{#1\textwidth}\infrule[#2]{#3}{#4}\end{minipage}}

\newcommand\Tlist{\texttt{ilist}}
\newcommand\RefAT{\textit{Refine}}


\vspace{-5pt}
\section{Predicate Discovery}
\label{sec:refine}

In this section, we propose an extension of our previous predicate
discovery method for higher-order programs used in
MoCHi~\cite{KobayashiPLDI2011}.  First, we briefly overview the previous
method in Section~\ref{sec:prev} and then discuss its limitation in
Section~\ref{sec:limit}. Section~\ref{sec:ext} explains the extension of
the method, which remedies the limitation.

\vspace{-2pt}
\subsection{Previous Method}
\label{sec:prev}

In MoCHi, predicates for abstracting each term of a given program are 
specified as a kind of dependent types called abstraction types.  MoCHi 
infers abstraction types automatically in a counterexample-guided manner 
(see Figure~\ref{fig:cegar}): In a CEGAR iteration of MoCHi, if the 
predicate abstraction at that point is not precise enough to show the 
safety of the original program, an error path of the abstracted program 
is returned as a result of higher-order model checking.  If the abstract 
error path is infeasible (i.e., not a genuine path of the original 
program), MoCHi generates a straightline higher-order program (SHP) 
which is safe if and only if the abstract error path is infeasible.  
MoCHi then uses an existing method~\cite{Unno2009} to infer refinement 
types that witness the safety of the SHP.  Here, to make the inference 
context-sensitive and complete as discussed in Section~\ref{sec:intro}, 
MoCHi ensures that the generated SHP is linear (i.e., each function is 
called exactly once) and recursion-free by duplicating and renaming the 
functions called multiple times in the infeasible error path (see 
\cite{KobayashiPLDI2011} for more details).  Finally, MoCHi extracts 
abstraction types from the refinement types, which contain precise 
enough predicates to refute the infeasible error path.

The key ingredient of the above predicate discovery procedure is the 
refinement type inference method~\cite{Unno2009}, which consists of two 
steps: constraint generation and solving.  We review the two steps 
respectively in Sections~\ref{sec:cg} and \ref{sec:cs}.

\vspace{-2pt}
\subsubsection{Constraint Generation}
\label{sec:cg}

\begin{figure}[t]
\begin{alltt}
letrec copy x = if x=0 then 0 else 1 + copy (x-1)
let main n = assert (copy n = n)
\end{alltt}
\caption{A Simplified Version of the Program in Figure~\ref{fig:copy}}
\label{fig:copy2}
\end{figure}

Let us assume that we are given a SHP \(D\) which is typable under a 
refinement type system (see, for example, \cite{Unno2009} for the 
definition of the system) if and only if the abstract error path is 
infeasible.  For example, let us consider the program in 
Figure~\ref{fig:copy2}, which is a simplified version of the one in 
Figure~\ref{fig:copy}.  In the course of its verification, we may obtain 
the following SHP \(D_{\texttt{copy}}\):
\begin{alltt}
 let copy1 x = assume (x<>0); 1 + copy2 (x-1)
 let copy2 x = assume (x=0); 0
 let main n = assume (copy1 n <> n); fail
\end{alltt}
The SHP corresponds to an infeasible error path where the else- and the 
then-branches of \(\texttt{copy}\) are respectively taken in the first 
and the second function call of \(\texttt{copy}\), and the assertion in 
the \(\texttt{main}\) function fails.  Note here that 
\(D_{\texttt{copy}}\) is safe (i.e., \texttt{fail} is not reachable), 
and hence is typable under the refinement type system.

From a SHP \(D\), we generate Horn-clause-like constraints which are 
satisfiable if and only if \(D\) is typable.  To this end, for each 
function in \(D\), we prepare a refinement type template with predicate 
variables, which act as placeholders of refinement predicates to be 
inferred.  We then generate a typing derivation for \(D\) under the type 
environment that associates each function with its type template.  
Horn-clause-like constraints on the predicate variables are then 
extracted from the derivation.  Since the SHP \(D\) is linear and 
recursion-free, generated constraints are non-recursive.  This is 
desirable since constraint solving of non-recursive Horn clauses over 
linear arithmetic is decidable.  For example, for the SHP 
\(D_{\texttt{copy}}\), we prepare the following type 
templates:\footnote{For the sake of simplicity, we here omit the type 
template of \texttt{main} as well as the refinement predicates for the 
argument of \texttt{copy1} and \texttt{copy2}.}
\vspace{-2pt}
\[
\begin{array}{rcl}
\texttt{copy1}&\COL&(x\COL\TFun{\INT}{\set{\nu\COL\INT \mid P_1(x,\nu)}}) \\
\texttt{copy2}&\COL&(x\COL\TFun{\INT}{\set{\nu\COL\INT \mid P_2(x,\nu)}})
\end{array}
\]
By using them, we obtain the following set \(C_{\texttt{copy}}\) of
constraints:
\[
\begin{array}{rcl}
P_2(x-1,y) \land x\neq0 \land z=1+y &\imply& P_1(x,z) \\
x=0 \land y=0 &\imply& P_2(x,y) \\
P_1(n,x) &\imply& x=n
\end{array}
\]

%%%We then use the constraint generation algorithm for terms defined in
%%%Figure~\ref{fig:cgen} to obtain constraints.  In the figure,
%%%$\CG{\Gamma}{e}$ returns a pair of a refinement type $\tau$ and a
%%%constraint $\theta$ such that $\Gamma \vdash e : \tau$ is derivable if
%%%and only if $\theta$ is valid. Similarly, $\CS{\Gamma}{\tau_1}{\tau_2}$
%%%returns a constraint $\theta$ such that $\Gamma \vdash \tau_1 \leq
%%%\tau_2$ is derivable if and only if $\theta$ is valid.
%%%%
%%%The algorithm is almost a straightforward modification of the typing and
%%%subtyping rules in Section~\ref{sec:reftypesystem} except that the
%%%application of the subsumption rule is restricted to the acutal argument
%%%of function applications.

%%%\begin{figure*}[tbh]
%%%\begin{eqnarray*}
%%%\CG{\Gamma}{x}
%%%&=&(\reftype{u}{u = x},\top) \quad (\mbox{if}~\sty{x}=\inttype) \\
%%%\CG{\Gamma}{\kappa}
%%%&=&(\Gamma(\kappa),\top) \quad (\mbox{if}~\sty{\kappa} \in \rightarrow) \\
%%%\CG{\Gamma}{c}
%%%&=&(\cty{c},\top) \\
%%%\CG{\Gamma}{\ttlet{x}{e_1}{e_2}}
%%%&=&\mbox{let~}(\sigma,\theta_1)=\CG{\Gamma}{e_1} \\
%%%& &\mbox{let~}(\cpstype,\theta_2)=\CG{\Gamma,x\smallcolon\sigma}{e_2} \\
%%%& &(\cpstype,\theta_1 \land \theta_2) \\
%%%\CG{\Gamma}{\ttapp{e}{x}}
%%%&=&\mbox{let~}(\funtype{y}{\sigma}{\tau},\theta_1)=\CG{\Gamma}{e} \\
%%%& &\mbox{let~}(\sigma',\theta_2)=\CG{\Gamma}{x} \\
%%%& &(\tau[x/y],\theta_1 \land \theta_2 \land \CS{\Gamma}{\sigma'}{\sigma}) \\
%%%\CG{\Gamma}{\ttifndet{e_1}{e_2}}
%%%&=&\mbox{let~}(\cpstype,\theta_1)=\CG{\Gamma}{e_1} \\
%%%& &\mbox{let~}(\cpstype,\theta_2)=\CG{\Gamma}{e_2} \\
%%%& &(\cpstype,\theta_1 \land \theta_2) \\
%%%\CS{\Gamma}{\cpstype}{\cpstype}
%%%&=&\top \\
%%%\CS{\Gamma}{\funtype{x}{\sigma_1}{\tau_1}}{\funtype{x}{\sigma_2}{\tau_2}}
%%%&=&\CS{\Gamma}{\sigma_2}{\sigma_1} \land \CS{\Gamma,x:\sigma_2}{\tau_1}{\tau_2} \\
%%%\CS{\Gamma}{\reftype{u}{\theta_1}}{\reftype{u}{\theta_2}}
%%%&=&\forall u.(\sembrack{\Gamma} \wedge \theta_1) \imply \theta_2  \quad (\mbox{if}~u \notin \free{\sembrack{\Gamma}})
%%%\end{eqnarray*}
%%%\caption{Constraint generation algorithm.}
%%%\label{fig:cgen}
%%%\end{figure*}

\vspace{-2pt}
\subsubsection{Constraint Solving}
\label{sec:cs}

Given a set \(C\) of non-recursive Horn clauses, our previous constraint
solving algorithm returns a substitution \(\theta\) for predicate
variables in \(C\) such that \(\theta C\) is valid.
%in a backward manner
The algorithm iteratively finds a solution for each predicate variable
\(P\) in \(C\) by using its greatest lower \(\lambda \seq{x}.\phi_{P}\)
and least upper \(\lambda \seq{x}.\phi_{P}'\) bounds with respect to the
partial order \(\sqsubseteq\) of predicates defined by: \(\lambda
\seq{x}.\phi_1 \sqsubseteq \lambda \seq{x}.\phi_2\) if and only if
\(\phi_1 \imply \phi_2\).
%
Intuitively, the predicate \(P(\seq{x})\) represents an invariant of 
some subexpression \(e\) in the SHP where some variable \(\nu 
\in\set{\seq{x}}\) represents the value of \(e\) and each variable in 
\(\set{\seq{x}} \setminus \set{\nu}\) represents a free variable in \(e\). 
 The greatest lower \(\lambda\seq{x}.\phi_P\) and the least upper 
\(\lambda \seq{x}.\phi_{P}'\) bounds for \(P\) respectively represent 
the strongest condition satisfied by the value \(\nu\) and the weakest 
condition on \(\nu\) required by the context of \(e\).
%
By using the greatest lower \(\lambda \seq{x}.\phi_{P}\) and the least 
upper \(\lambda \seq{x}.\phi_{P}'\) bounds, the algorithm 
computes\(\lambda \seq{x}.\mathcal{I}(\phi_P,\neg \phi_P')\) as a 
solution for\(P\) with the help of a technique called 
interpolation~\cite{Henzinger2004,McMillan2005} from automated theorem 
proving.  Here, an interpolant \(\mathcal{I}(\phi_1,\phi_2)\) 
of\(\phi_1\) and \(\phi_2\) (such that \(\phi_1\) and \(\phi_2\) are 
inconsistent) is a formula \(\phi\) that satisfies the following 
conditions:\footnote{Note that interpolants of \(\phi_1\) and \(\phi_2\) 
are not unique.  Actually, existing theorem 
provers~\cite{Henzinger2004,McMillan2005,Beyer2008} return one of them, 
which is denoted by \(\mathcal{I}(\phi_1,\phi_2)\).}
\vspace{-4pt}
\begin{itemize}
\item \(\phi_1\) implies \(\phi\),
\item \(\phi\) and \(\phi_2\) are inconsistent, and
\item \(\FV{\phi} \subseteq \FV{\phi_1} \cap \FV{\phi_2}\).
\end{itemize}
\vspace{-4pt}
%
For example, we obtain the following bounds for \(P_1\) from 
\(C_{\texttt{copy}}\):
\vspace{-4pt}
\begin{itemize}
\item \(\lambda x.\lambda \nu.\phi_{P_1}\equiv \lambda x.\lambda \nu.x=1 \land \nu=1\) and
\item \(\lambda x.\lambda \nu.\phi_{P_1}'\equiv \lambda x.\lambda \nu.\nu=x\).
\end{itemize}
\vspace{-4pt}
We then obtain, for example, the following solution for \(P_1\):
\begin{eqnarray*}
\lambda x.\lambda \nu.\mathcal{I}(\nu.x=1 \land \nu=1,\neg \nu=x) \equiv \lambda x.\lambda \nu.\nu=x.
\end{eqnarray*}
By substituting this for \(P_1\) in \(C_{\texttt{copy}}\), we get
\(C_{\texttt{copy}}'\):
\begin{eqnarray*}
P_2(x-1,y) \land x\neq0 \land z=1+y &\imply& z=x \\
x=0 \land y=0 &\imply& P_2(x,y)
\end{eqnarray*}
We then obtain the following bounds for \(P_2\) from 
\(C_{\texttt{copy}}'\):
\vspace{-4pt}
\begin{itemize}
\item \(\lambda x.\lambda \nu. \phi_{P_2}' \equiv \lambda x.\lambda \nu. x+1\neq0 \imply \nu=x\) and
\item \(\lambda x.\lambda \nu. \phi_{P_2} \equiv \lambda x.\lambda \nu. x=0 \land \nu=0\).
\end{itemize}
\vspace{-4pt}
By using them, we get, for example, the following solution for \(P_2\):
\begin{eqnarray*}
\lambda x.\lambda \nu.\mathcal{I}(\nu.x=0 \land \nu=0,\neg (x+1\neq0 \imply \nu=x)) \equiv \lambda x.\lambda \nu.\nu=x.
\end{eqnarray*}
We thus obtain the following refinement types for \(D_{\texttt{copy}}\):
\begin{eqnarray*}
\texttt{copy1}&\COL&(x\COL\TFun{\INT}{\set{\nu\COL\INT \mid x=\nu}}) \\
\texttt{copy2}&\COL&(x\COL\TFun{\INT}{\set{\nu\COL\INT \mid x=\nu}})
\end{eqnarray*}

\vspace{-2pt}
\subsection{Limitation of Previous Method}
\label{sec:limit}

We now explain the limitation of the previous method by using the 
program in Figure~\ref{fig:copy}.  Let us consider the following SHP 
\(D_{\texttt{cc}}\):
\begin{alltt}
let copy1 x = assume (x<>0); 1 + copy2 (x-1)
let copy2 x = assume (x=0); 0
let copy3 x = assume (x<>0); 1 + copy4 (x-1)
let copy4 x = assume (x=0); 0
let main n = assume (copy3 (copy1 n) <> n); fail
\end{alltt}
The SHP corresponds to an infeasible error path where the else-branch of
\(\texttt{copy}\) is taken in the first and the third calls of
\(\texttt{copy}\), the then-branch is taken in the second and the fourth
calls of \(\texttt{copy}\), and the assertion in the \(\texttt{main}\)
function fails.

For the SHP \(D_{\texttt{cc}}\), we use the following type templates:
\[
\begin{array}{rcl}
\texttt{copy1}&\COL&(x\COL\TFun{\INT}{\set{\nu\COL\INT \mid P_1(x,\nu)}}) \\
%\texttt{copy2}&\COL&(x\COL\TFun{\INT}{\set{\nu\COL\INT \mid P_2(x,\nu)}}) \\
%\texttt{copy3}&\COL&(x\COL\TFun{\INT}{\set{\nu\COL\INT \mid P_3(x,\nu)}}) \\
\vdots\quad &&\qquad\qquad \vdots \\
\texttt{copy4}&\COL&(x\COL\TFun{\INT}{\set{\nu\COL\INT \mid P_4(x,\nu)}})
\end{array}
\]
And, we get the following set \(C_{\texttt{cc}}\) of constraints:
\[
\begin{array}{rcl}
P_2(x-1,y) \land x\neq0 \land z=1+y &\imply& P_1(x,z) \\
x=0 \land y=0 &\imply& P_2(x,y) \\
P_4(x-1,y) \land x\neq0 \land z=1+y &\imply& P_3(x,z) \\
x=0 \land y=0 &\imply& P_4(x,y) \\
P_1(n,x) \land P_3(x,y) &\imply& y=n
\end{array}
\]

We then find a solution for \(P_3\): Since the greatest lower and the 
least upper bounds of \(P_3\) are respectively \(\lambda 
x.\lambda\nu.x=1 \land \nu=1\) and \(\lambda x.\lambda \nu.x=1 \imply 
\nu=1\),
%\(\lambda x,n=1 \land x=1 \imply \nu=n\).
%
existing interpolating proves such as \cite{Beyer2008} returns the 
following solution for \(P_3\):
\[
\begin{array}{rcl}
%&&\lambda x.\lambda \nu.\mathcal{I}(\nu.x=1 \land \nu=1,\neg (n=1 \land x=1 \imply \nu=n)) \\
&&\lambda x.\lambda \nu.\mathcal{I}(\nu.x=1 \land \nu=1,\neg (x=1 \imply \nu=1)) \\
&\equiv& \lambda x.\lambda \nu.x=1 \land \nu=1.
\end{array}
\]
Note here that the solution is specific to the calling context of the 
particular function \texttt{copy3}, and cannot be used as a solution for 
\(P_2\) and \(P_4\).  We here want to get more general predicates like 
\(\lambda x.\lambda \nu.\nu=x\) which are more likely to constitute an 
invariant of the function \texttt{copy} in the original program.  For 
this purpose, we believe it is desirable to find the same solution (if 
possible) for ``related'' predicate variables which represent (possibly 
different) refinement predicates for the same argument or return value 
of the same function in the original program.  For the above example, we 
want to obtain the same solution for \(P_1,\dots,P_4\), and \(\lambda 
x.\lambda \nu.\nu=x\) in fact satisfies this extra constraint.

\vspace{-2pt}
\subsection{Extended Method}
\label{sec:ext}

We now explain our extension of the previous method to remedy the
limitation discussed in Section~\ref{sec:limit}.  The extended predicate
discovery method is based on the framework of the previous method
overviewed in Section~\ref{sec:prev}, but the component for refinement
type inference is extended so that it can merge and generalize
information from multiple calling contexts of a function in multiple
infeasible error paths.  This enables MoCHi to infer a general
refinement type of the function that type-checks the multiple calling
contexts, while preserving the path- and context-sensitivity.  In other
words, the extended method generates constraints from multiple
infeasible error paths (see Section~\ref{sec:extcg}), and tries to find
the same solution (if possible) for related predicate variables (see
Section~\ref{sec:extcs}).

\vspace{-2pt}
\subsubsection{Extensions of Constraint Generation}
\label{sec:extcg}

We extend the previous constraint generation algorithm overviewed in
Section~\ref{sec:cg} as follows.
\vspace{-4pt}
\begin{itemize}
\item For each CEGAR iteration, we generate constraints from multiple
infeasible error paths instead of a single path:  We keep the set
\(\set{\pi_1,\cdots,\pi_n}\) of the infeasible error paths found so far,
generate the set \(C_i\) of Horn clauses for each path \(\pi_i\), and
pass \(C=C_1 \cup \dots \cup C_n\) to the extended constraint solving
algorithm described in Section~\ref{sec:extcs} as an input.
%For example, \todo{}
\item We also construct and pass an equivalence relation \(E\) on the 
predicate variables in \(C\) such that \(P\ E\ Q\) if and only if the 
predicate variables \(P\) and \(Q\) represent (possibly different) 
refinement predicates for the same argument or return value of the same 
function in the original program.  For example, we obtain the trivial 
equivalence relation \(E_{\texttt{cc}}=\set{P_1,\dots,P_4} 
\times\set{P_1,\dots,P_4}\) for \(C_{\texttt{cc}}\).  The constraint 
solving algorithm in Section~\ref{sec:extcs} exploits \(E\) to find 
general solutions for \(C\).
\end{itemize}
\vspace{-4pt}
Thus, the extended algorithm generates a pair \((C,E)\) of Horn clauses 
\(C\) for multiple paths and an equivalence relation \(E\) on the 
predicate variables in \(C\) unlike the previous algorithm which 
generates only Horn clauses for a single path.  Here, the pair \((C,E)\) 
can be viewed as hierarchical constraints where \(C\) must be always 
satisfied and \(E\) should be satisfied if possible.

\vspace{-2pt}
\subsubsection{Extensions of Constraint Solving}
\label{sec:extcs}

In this section, we extend the previous constraint solving algorithm 
overviewed in Section~\ref{sec:cs}.  Given a pair \((C,E)\) of Horn 
clauses \(C\) and an equivalence relation \(E\) on the predicate 
variables in \(C\), the algorithm returns a substitution \(\theta\) for 
the predicate variables in \(C\) such that \(\theta C\) is valid.  A 
distinguishing feature of the algorithm is that it tries to find the 
same solution for predicate variables related by \(E\) if possible.  
This enables the algorithm to obtain general predicates, which are more 
likely to constitute invariants. % of the original program.

%%%\begin{figure}
%%%\todo{}
%%%\caption{The Overall Structure of Our Extended Constraint Solving Algorithm}
%%%\label{fig:extcs}
%%%\end{figure}

%%%The overall structure of the algorithm is shown in Figure~\ref{fig:extcs}.
The extended constraint solving algorithm proceeds as follows:
\begin{enumerate}
\item Find a set \(S\) of predicate variables which are related by \(E\) 
and may have the same solution in \(C\).
\item Find a candidate solution \(\lambda \seq{x}.\phi\) for the
predicate variables in \(S\).
\item Substitute \(\lambda \seq{x}.\phi\) for \(S\) in \(C\) to get
\(C'\) and repeat the entire procedure if \(C'\) still contains a
predicate variable.
\end{enumerate}

\paragraph{Finding a set \(S\) of predicate variables:}
We first compute the greatest lower \(\lambda \seq{x}.\phi_P\) and the 
least upper \(\lambda \seq{x}.\phi_P'\) bounds for each \(P\) in \(C\), 
which are used to check whether given predicate variables may have the 
same solution in \(C\).  Note here that \(\FV{\phi_P} \cap \FV{\phi_P'} 
\subseteq \set{\seq{x}}\) always holds.
%
For example, the greatest lower bounds for \(P_1,\dots,P_4\) in
\(C_{\texttt{cc}}\) are:
\vspace{-4pt}
\begin{itemize}
\item \(\lambda x.\lambda \nu.\phi_{P_1} \equiv \lambda x.\lambda \nu.\phi_{P_3} \equiv \lambda x.\lambda \nu.x=1 \land \nu=1\) and
\item \(\lambda x.\lambda \nu.\phi_{P_2} \equiv \lambda x.\lambda \nu.\phi_{P_4} \equiv \lambda x.\lambda \nu.x=0 \land \nu=0\).
\end{itemize}
\vspace{-4pt}
The least upper bounds for \(P_1,\dots,P_4\) are:
\vspace{-4pt}
\begin{itemize}
\item \(\lambda x.\lambda \nu.\phi_{P_1}'\equiv \lambda x.\lambda \nu.\nu=1 \imply x=1\),
\item \(\lambda x.\lambda \nu.\phi_{P_2}'\equiv \lambda x.\lambda \nu.\nu=0 \imply (x=-1 \lor x=0)\),
\item \(\lambda x.\lambda \nu.\phi_{P_3}'\equiv \lambda x.\lambda \nu.x=1 \imply \nu=1\), and
\item \(\lambda x.\lambda \nu.\phi_{P_4}'\equiv \lambda x.\lambda \nu.x=0 \imply \nu=0\).
\end{itemize}
\vspace{-4pt}

In general, the least upper bound of some predicate variable may not 
exist because the constraint generation algorithm in 
Section~\ref{sec:extcg} may generate a Horn clause of the form 
\(P(x)\land P(y) \imply \phi\).  Note here that \(P\) occurs twice in 
the left hand side of the constraint.  In such a case, the least upper 
bound for \(P\) may not exist.

Let \(X\) be the set of predicate variables in \(C\) which have the 
least upper bound.  We pick an equivalence class \(S_0 \in X / E\) (e.g., 
the largest one), and further classify \(S_0\) by using the bounds so 
that predicate variables which never have the same solution are 
separated.
%
Formally, we find \(S_1\dots,S_n\) such that:
\vspace{-4pt}
\begin{itemize}
\item \(S_0 = S_1 \cup \dots \cup S_n\),
\item for any \(i \in \set{1,\dots,n}\), if \(S_i
=\set{Q_1,\dots,Q_{\ell}}\), then \(\phi_{Q_1} \lor \dots \lor
\phi_{Q_{\ell}}\) implies \(\phi_{Q_1}' \land \dots \land
\phi_{Q_{\ell}}'\), and \item for any \(i,j \in \set{1,\dots,n}\), if
\(i \neq j\) and \(S_i \cup S_j = \set{Q_1,\dots,Q_{\ell}}\), then
\(\phi_{Q_1} \lor \dots \lor \phi_{Q_{\ell}}\) does not imply
\(\phi_{Q_1}' \land \dots \land \phi_{Q_{\ell}}'\).
\end{itemize}
\vspace{-4pt}
We then pick some \(S \in \set{S_1,\dots,S_n}\) (e.g., the largest one).
%
For the running example \(C_{\texttt{cc}}\), we get
\(S=S_0=S_1=\set{P_1,\dots,P_4}\) since \(\phi_{P_1} \lor \dots
\lor\phi_{P_4}\) implies \(\phi_{P_1}' \land \dots \land \phi_{P_4}'\).

\paragraph{Finding a candidate solution \(\lambda \seq{x}.\phi\) for \(S\):}
We then find a candidate solution \(\lambda \seq{x}.\phi\) for \(S = 
\set{Q_1,\dots,Q_{\ell}}\) such that:
\vspace{-4pt}
\begin{itemize}
\item \(\phi_{Q_1} \lor \dots \lor \phi_{Q_{\ell}}\) implies \(\phi\),
\item \(\phi\) implies \(\phi_{Q_1}' \land \dots \land \phi_{Q_{\ell}}'\), and
\item \(\FV{\phi} \subseteq \set{\seq{x}}\).
\end{itemize}
\vspace{-4pt}
%Terauchi2010
We can compute such a formula \(\phi\) as an interpolant
\(\mathcal{I}(\phi_{Q_1} \lor \dots \lor \phi_{Q_{\ell}},\neg
(\phi_{Q_1}' \land \dots \land \phi_{Q_{\ell}}'))\) but the three
conditions of interpolants are not always sufficient for our purpose to
find general predicates.  Actually, we want to obtain as simple
interpolant as possible with respect to the number of disjunctions.
% and conjunctions.
To this end, we propose a new heuristic operator \(\mathcal{J}\) that 
combines the interpolation \(\mathcal{I}\) and convex hull operators.  
Let us write \(\mathcal{H}(\phi)\) to denote the convex hull of \(\phi\). 
 For formulas \(\phi_1\) and \(\phi_2\) (such that \(\phi_1\) and 
\(\phi_2\) are inconsistent), the new operator 
\(\mathcal{J}(\phi_1,\phi_2)\) is defined as follows:
\[
\mathcal{J}(\phi_1,\phi_2) =
\left\{
\begin{array}{ll}
\mathcal{I}(\mathcal{H}(\phi_1),\mathcal{H}(\phi_2)) & (\mbox{if~}\mathcal{H}(\phi_1) \INCON \mathcal{H}(\phi_2)) \\
\mathcal{I}(\mathcal{H}(\phi_1),\phi_2) & (\mbox{if~}\neg (\mathcal{H}(\phi_1) \INCON \mathcal{H}(\phi_2)) \land \\
&\ \quad \mathcal{H}(\phi_1) \INCON \phi_2) \\
\mathcal{I}(\phi_1,\phi_2) & (\mbox{otherwise})
\end{array}
\right.
\]
Here, we write \(\phi_1 \INCON \phi_2\) to denote that \(\phi_1\) and 
\(\phi_2\) are inconsistent.  Note here that the use of the convex hull 
operator enables us to eliminate disjunctions in \(\phi_1\) and 
\(\phi_2\), which are passed to an interpolating theorem prover.
%the interpolation operator \(\mathcal{I}\).
In the experiments reported in Section~\ref{sec:experiments}, this often
reduced the number of disjunctions in the output of the interpolating
prover, and hence makes the output more likely to constitute invariants.
%
Thus, we use the new operator \(\mathcal{J}\) instead of \(\mathcal{I}\) 
to compute \(\lambda\seq{x}.\mathcal{J}(\phi_{Q_1} \lor \dots\lor 
\phi_{Q_{\ell}},\neg (\phi_{Q_1}' \land \dots \land\phi_{Q_{\ell}}'))\) 
as a candidate solution \(\lambda \seq{x}. \phi\) for \(S\).  For the 
running example \(C_{\texttt{cc}}\), an interpolating prover returned 
the following candidate solution \(\lambda x \lambda \nu.\phi\) for 
\(P_1,\dots,P_4\) in the experiments reported in 
Section~\ref{sec:experiments}:
\[
\begin{array}{rcl}
\phi&=& \mathcal{J}(\phi_{P_1} \lor \dots \lor \phi_{P_4},\neg (\phi_{P_1}' \land \dots \land \phi_{P_4}')) \\
%&=& \mathcal{I}(\mathcal{H}(\phi_{P_1} \lor \dots \lor \phi_{P_4}),\neg (\phi_{P_1}' \land \dots \land \phi_{P_4}')) \\
&=& \mathcal{I}(\mathcal{H}(x=\nu=0 \lor x=\nu=1),\neg (\phi_{P_1}' \land \dots \land \phi_{P_4}')) \\
&=& \mathcal{I}(0 \leq x=\nu \leq 1,\neg (\phi_{P_1}' \land \dots \land \phi_{P_4}')) \\
&=& x=\nu
\end{array}
\]

\paragraph{Substituting \(\lambda \seq{x}.\phi\) for \(S\) in \(C\):}
We then substitute the candidate solution \(\lambda \seq{x}.\phi\) for 
\(S\) in \(C\).  Note, however, that we cannot always substitute all the 
predicates \(Q_1,\dots,Q_{\ell} \in S\) with the candidate solution 
\(\lambda\seq{x}.\phi\) because \(Q_i\) may depend on \(Q_j\) for some 
\(i \neq j\).  For example, let us consider the following constraints:
\begin{eqnarray*}
x=0 \imply Q_1(x), \quad
Q_1(x) \imply Q_2(x+1), \quad
Q_2(x) \imply 0 \leq x \leq 2
\end{eqnarray*}
Here, the greatest lower bounds of \(Q_1\) and \(Q_2\) are respectively 
\(\lambda \nu.\nu=0\) and \(\lambda \nu.\nu=1\).  The least upper bounds 
of \(Q_1\) and \(Q_2\) are respectively \(\lambda \nu.-1 \leq \nu\leq 1\) 
and \(\lambda \nu.0 \leq \nu \leq 2\).  Thus, we obtain, for example, 
\(\lambda \nu.\mathcal{J}(\nu=0 \lor \nu=1,\neg (-1 \leq \nu\leq 1 \land 
0 \leq \nu \leq 2)) \equiv \lambda \nu.0 \leq \nu \leq 1\) as a 
candidate solution for \(Q_1\) and \(Q_2\).  However, \([\lambda\nu.0 
\leq \nu \leq 1/Q_1,\lambda \nu.0 \leq \nu \leq 1/Q_2](Q_1(x) \imply 
Q_2(x+1))\) is not valid.  Actually, it is only safe to substitute 
\(\lambda \nu.0 \leq \nu\leq 1\) for either \(Q_1\) or \(Q_2\).

Therefore, we find and substitute only a maximal nonempty subset 
\(\{R_1,\dots,R_m\}\) of \(S\) for which we can safely substitute 
\(\lambda\seq{x}.\phi\) (i.e., 
\([\lambda\seq{x}.\phi/R_1,\dots,\lambda\seq{x}.\phi/R_m]C\) is 
equisatisfiable with \(C\)).
%
%%%such that \(\lambda\seq{x}.\phi\) is in fact a solution for every predicate variable in \(M\).
%
%%%Formally, we find \(M=\set{R_1,\dots,R_m}\) such that:
%%%\begin{itemize}
%%%\item \(M \subseteq S\),
%%%\item \(\theta[\lambda \seq{x}.\phi/R_1,\dots,\lambda
%%%\seq{x}.\phi/R_m]C\) is valid for some substitution \(\theta\) for
%%%predicate variables, and
%%%\item for any \(R_0 \in S \setminus M\) and \(\theta\) the following
%%%formula is not valid: \(\theta[\lambda \seq{x}.\phi/R_0,\lambda
%%%\seq{x}.\phi/R_1,\dots,\lambda \seq{x}.\phi/R_m]C\).
%%%\end{itemize}
For the running example \(C_{\texttt{cc}}\), it is safe to substitute 
the candidate solution \(\lambda x \lambda \nu.x=\nu\) for 
\(P_1,\dots,P_4\) (i.e., \(M=S\)).  As a result of the substitution, 
every predicate variable in \(C_{\texttt{cc}}\) is eliminated.  Thus, we 
obtain the following refinement types for \(D_{\texttt{cc}}\):
\[
\begin{array}{rcl}
\texttt{copy1}&\COL&(x\COL\TFun{\INT}{\set{\nu\COL\INT \mid x=\nu}}) \\
%\texttt{copy2}&\COL&(x\COL\TFun{\INT}{\set{\nu\COL\INT \mid x=\nu}}) \\
%\texttt{copy3}&\COL&(x\COL\TFun{\INT}{\set{\nu\COL\INT \mid x=\nu}}) \\
\vdots\quad &&\qquad\qquad \vdots \\
\texttt{copy4}&\COL&(x\COL\TFun{\INT}{\set{\nu\COL\INT \mid x=\nu}})
\end{array}
\]


%\todo{discuss termination?}

%%%infer a refinement type of each
%%%subexpression \(e\) of an ordinary ML type \(\tau\):
%%%  The method then computes an
%%%interpolant \(\phi\) of \(\phi_{post}\) and \(\phi_{pre}\), and returns
%%%\(\set{\nu:\tau \mid \phi}\) as a refinement type of \(e\).
%
%%%%Thus, the method considers both forward and backward information of \(e\)
%%%%respectively obtained from \(e\) and the context of \(e\).
