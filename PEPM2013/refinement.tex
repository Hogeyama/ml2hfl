\newcommand{\todo}[1]{\textcolor{red}{***ToDo: #1***}}

\newcommand{\INCON}{\ \bot\ }
\newcommand\imply{\Rightarrow}


\newcommand{\aset}[1]{\{{#1}\}}
\newcommand{\aseq}[1]{\langle#1\rangle}
\newcommand{\adseq}[1]{\langle\hspace*{-0.1em}\langle{#1}\rangle\hspace*{-0.1em}\rangle}
\newcommand{\sembrack}[1]{[\hspace*{-0.1em}[#1]\hspace*{-0.1em}]}
\newcommand{\dom}{\mathop\textit{dom}\nolimits}
\newcommand{\ran}{\mathop\textit{ran}\nolimits}
\newcommand{\smallcolon}{\hspace*{-0.2em}:\hspace*{-0.2em}}
\newcommand{\smallless}{\hspace*{-0.2em}<\hspace*{-0.2em}}

\newcommand{\lvec}{\overrightarrow}
\newcommand{\emptyseq}{\varepsilon}
\newcommand{\inttype}{\texttt{int}}
\newcommand{\unittype}{\texttt{unit}}
\newcommand{\booltype}{\texttt{bool}}
\newcommand{\reftype}[2]{\aset{{#1} \mid {#2}}}
\newcommand{\oldreftype}[3]{\aset{{#1} \smallcolon {#2} \mid {#3}}}
\newcommand{\funtype}[3]{{#1}\smallcolon{#2}\rightarrow{#3}}
\newcommand{\sfuntype}[2]{{#1}\rightarrow{#2}}
\newcommand{\univtype}[3]{\forall{#1}\smallcolon{#2}.{#3}}

\newcommand{\univtobind}[1]{\aseq{{#1}}}

\newcommand{\builttype}{{\it type}}
\newcommand{\conjoftype}{{\it conj}}


\newcommand{\ttst}[2]{\texttt{st}\:{#1}\:{\tt in}\:{#2}}
\newcommand{\ttlet}[3]{\texttt{let}\:{#1}\,\texttt{=}\,{#2}\:\texttt{in}\:{#3}}
\newcommand{\ttifndet}[2]{\texttt{if}\:{*}\:{\tt then}\:{#1}\:{\tt else}\:{#2}}
\newcommand{\ttif}[3]{\texttt{if}\:{#1}\:\texttt{then}\:{#2}\:\texttt{else}\:{#3}}
\newcommand{\ttfun}[3]{{#1}\,{#2}\:\texttt{=}\:{#3}}


\newcommand{\ttlam}[2]{\lambda{#1}.{#2}}
\newcommand{\ttfix}[2]{\texttt{fix}\:{#1}.{#2}}
\newcommand{\ttapp}[2]{{#1}\:{#2}}
\newcommand{\ttassert}[1]{\texttt{assert}\:{#1}}
\newcommand{\ttassume}[1]{\texttt{assume}\:{#1}}
\newcommand{\ttfail}{{\sf fail}}
\newcommand{\ttsafe}{{\sf safe}}
\newcommand{\ttmain}{{\tt main}}

\newcommand{\tttrue}{1}
\newcommand{\ttfalse}{0}

\newcommand{\envmapsto}{\mapsto}
\newcommand{\cpstype}{\star}
\newcommand{\weak}{\preceq}

%\newcommand{\wellscoped}[1]{{\it closed}({#1})}
\newcommand{\scopevars}[1]{\textit{sv}({#1})}
\newcommand{\free}[1]{\textit{fv}({#1})}
%\newcommand{\wellscopedp}[2]{{\it closed}_{#1}({#2})}
\newcommand{\latleq}{\preceq}
\newcommand{\cty}[1]{\textit{ty}({#1})}
\newcommand{\sty}[1]{\textit{sty}({#1})}
\newcommand{\tyshape}[1]{\textit{tshape}({#1})}
\newcommand{\utyshape}[1]{\textit{ushape}({#1})}
\newcommand{\abs}[1]{\ddot{#1}}

\newcommand{\reduce}[1]{\rightarrow_{#1}}

\newcommand{\trans}[1]{\textit{tr}({#1})}

\newcommand{\wildcard}{\_}

\newcommand{\wpre}{\textit{wpre}}
\newcommand{\godel}[1]{\lceil#1\rceil}

\newcommand{\basevar}[1]{\underline{#1}}

\newcommand{\evalterm}{\Downarrow_{\it  val}}

\newcommand{\stepfix}{\Psi_{\it wp}}

\newcommand{\wpretype}{{\it wpty}}

\newcommand{\defaultargs}{\lvec{\beta}}

\newcommand{\basedefaultargs}{\lvec{\alpha}}

\newcommand{\args}{{\it args}}

\newcommand{\notargs}{{\it args}^-}

\newcommand{\vdashs}{\vdash^s}

\newcommand{\mochi}{{\sf MoCHi}}
\newcommand{\maxargs}{{\sf maxargs}}

\newcommand{\head}{x_{\it hd}}

\newcommand{\vdashuni}{\vdash_\forall}

\newcommand{\cand}{\textit{cands}}

\newcommand{\cex}{\leadsto}

\newcommand{\pureintexps}{{\it pureExps}}

\newcommand{\CG}[2]{\textit{cgen}\left(#1 \vdash #2\right)}
\newcommand{\CS}[3]{\textit{cgen}\left(#1 \vdash #2 \leq #3\right)}


\section{Predicate Discovery}
\label{sec:refine}

In this section, we propose an extension of our previous predicate
discovery method for higher-order programs used in
MoCHi~\cite{KobayashiPLDI2011}.  First, we briefly overview the previous
method in Section~\ref{sec:prev} and then discuss its limitation in
Section~\ref{sec:limit}. Section~\ref{sec:ext} explains the extension of
the method, which remedies the limitation.

\subsection{Previous Method}
\label{sec:prev}

In MoCHi, predicates for abstracting each term of a given program are
specified as a kind of dependent types called abstraction types.  MoCHi
infers abstraction types automatically in a counterexample-guided manner
(recall Figure~\ref{fig:cegar}): In a CEGAR iteration of MoCHi, if the
predicate abstraction at that point is not precise enough to show the
safety of the original program, an error path of the abstracted program
is returned as a result of higher-order model checking.  If the abstract
error path is infeasible (i.e., not a genuine path of the original
program), MoCHi generates a straightline higher-order program (SHP)
that is safe if and only if the abstract error path is infeasible.
MoCHi then uses an existing method~\cite{Unno2009} to infer refinement
types that witness the safety of the SHP.  Here, to make the inference
context-sensitive and complete as discussed in Section~\ref{sec:intro},
MoCHi ensures the generated SHP to be linear (i.e., each function is
called exactly once) and recursion-free by duplicating and renaming the
functions called multiple times in the infeasible error path (see
\cite{KobayashiPLDI2011} for more details).  Finally, MoCHi extracts
abstraction types from the refinement types, which contain precise
enough predicates to refute the infeasible error path.

The key ingredient of the above predicate discovery procedure is the
refinement type inference method~\cite{Unno2009}, which consists of two
steps: constraint generation and solving.  We review the two steps
respectively in Sections~\ref{sec:cg} and \ref{sec:cs}.

\subsubsection{Constraint Generation}
\label{sec:cg}

\begin{figure}[t]
\begin{alltt}
letrec copy x = if x=0 then 0 else 1 + copy (x-1)
let main n = assert (copy n = n)
\end{alltt}
\caption{A Simplified Version of the Program in Figure~\ref{fig:copy}}
\label{fig:copy2}
\end{figure}

Let us assume that we are given a SHP \(D\) which is typable under a
refinement type system (see, for example, \cite{Unno2009} for the
definition of the system) if and only if the abstract error path is
infeasible.  For example, let us consider the program in
Figure~\ref{fig:copy2}, which is a simplified version of the one in
Figure~\ref{fig:copy}.  In the course of its verification, we may obtain
the following SHP \(D_{\texttt{copy}}\):
\begin{alltt}
 let copy1 x = assume (x<>0); 1 + copy2 (x-1)
 let copy2 x = assume (x=0); 0
 let main n = assume (copy1 n <> n); fail
\end{alltt}
The SHP corresponds to an infeasible error path where the else- and the
then-branches of \(\texttt{copy}\) are respectively taken in the first
and the second function call of \(\texttt{copy}\), and the assertion in
the \(\texttt{main}\) function fails.  Note here that
\(D_{\texttt{copy}}\) is safe (i.e., \texttt{fail} is not reachable),
and hence is typable under the refinement type system.

From a SHP \(D\), we generate Horn-clause-like constraints which are
satisfiable if and only if \(D\) is typable.  To this end, for each
function in \(D\), we prepare a refinement type template with predicate
variables, which act as placeholders of refinement predicates to be
inferred.  We then generate a typing derivation for \(D\) under the type
environment that associates each function with its type template.
Horn-clause-like constraints on the predicate variables are then
extracted from the derivation.  Since the SHP \(D\) is linear and
recursion-free, generated constraints are non-recursive.  This is
desirable since constraint solving of non-recursive Horn clauses over
decidable underlying theories (e.g., linear arithmetic) is decidable.
For the running example \(D_{\texttt{copy}}\), we use the following
templates:\footnote{For the sake of simplicity, we here omit the type
template of \texttt{main} as well as the refinement predicates for the
argument of \texttt{copy1} and \texttt{copy2}.}
\[
\begin{array}{rcl}
%\begin{eqnarray*}
\texttt{copy1}&\COL&(x\COL\TFun{\INT}{\set{\nu\COL\INT \mid P_1(x,\nu)}}) \\
\texttt{copy2}&\COL&(x\COL\TFun{\INT}{\set{\nu\COL\INT \mid P_2(x,\nu)}})
%\end{eqnarray*}
\end{array}
\]
By using them, we obtain the following set \(C_{\texttt{copy}}\) of
constraints:
\[
\begin{array}{rcl}
%\begin{eqnarray*}
x=0 \land y=0 &\imply& P_2(x,y) \\
P_2(x-1,y) \land x\neq0 \land z=1+y &\imply& P_1(x,z) \\
P_1(n,x) &\imply& x=n
%\end{eqnarray*}
\end{array}
\]

%%%We then use the constraint generation algorithm for terms defined in
%%%Figure~\ref{fig:cgen} to obtain constraints.  In the figure,
%%%$\CG{\Gamma}{e}$ returns a pair of a refinement type $\tau$ and a
%%%constraint $\theta$ such that $\Gamma \vdash e : \tau$ is derivable if
%%%and only if $\theta$ is valid. Similarly, $\CS{\Gamma}{\tau_1}{\tau_2}$
%%%returns a constraint $\theta$ such that $\Gamma \vdash \tau_1 \leq
%%%\tau_2$ is derivable if and only if $\theta$ is valid.
%%%%
%%%The algorithm is almost a straightforward modification of the typing and
%%%subtyping rules in Section~\ref{sec:reftypesystem} except that the
%%%application of the subsumption rule is restricted to the acutal argument
%%%of function applications.

%%%\begin{figure*}[tbh]
%%%\begin{eqnarray*}
%%%\CG{\Gamma}{x}
%%%&=&(\reftype{u}{u = x},\top) \quad (\mbox{if}~\sty{x}=\inttype) \\
%%%\CG{\Gamma}{\kappa}
%%%&=&(\Gamma(\kappa),\top) \quad (\mbox{if}~\sty{\kappa} \in \rightarrow) \\
%%%\CG{\Gamma}{c}
%%%&=&(\cty{c},\top) \\
%%%\CG{\Gamma}{\ttlet{x}{e_1}{e_2}}
%%%&=&\mbox{let~}(\sigma,\theta_1)=\CG{\Gamma}{e_1} \\
%%%& &\mbox{let~}(\cpstype,\theta_2)=\CG{\Gamma,x\smallcolon\sigma}{e_2} \\
%%%& &(\cpstype,\theta_1 \land \theta_2) \\
%%%\CG{\Gamma}{\ttapp{e}{x}}
%%%&=&\mbox{let~}(\funtype{y}{\sigma}{\tau},\theta_1)=\CG{\Gamma}{e} \\
%%%& &\mbox{let~}(\sigma',\theta_2)=\CG{\Gamma}{x} \\
%%%& &(\tau[x/y],\theta_1 \land \theta_2 \land \CS{\Gamma}{\sigma'}{\sigma}) \\
%%%\CG{\Gamma}{\ttifndet{e_1}{e_2}}
%%%&=&\mbox{let~}(\cpstype,\theta_1)=\CG{\Gamma}{e_1} \\
%%%& &\mbox{let~}(\cpstype,\theta_2)=\CG{\Gamma}{e_2} \\
%%%& &(\cpstype,\theta_1 \land \theta_2) \\
%%%\CS{\Gamma}{\cpstype}{\cpstype}
%%%&=&\top \\
%%%\CS{\Gamma}{\funtype{x}{\sigma_1}{\tau_1}}{\funtype{x}{\sigma_2}{\tau_2}}
%%%&=&\CS{\Gamma}{\sigma_2}{\sigma_1} \land \CS{\Gamma,x:\sigma_2}{\tau_1}{\tau_2} \\
%%%\CS{\Gamma}{\reftype{u}{\theta_1}}{\reftype{u}{\theta_2}}
%%%&=&\forall u.(\sembrack{\Gamma} \wedge \theta_1) \imply \theta_2  \quad (\mbox{if}~u \notin \free{\sembrack{\Gamma}})
%%%\end{eqnarray*}
%%%\caption{Constraint generation algorithm.}
%%%\label{fig:cgen}
%%%\end{figure*}

\subsubsection{Constraint Solving}
\label{sec:cs}

Given a set \(C\) of non-recursive Horn clauses, our previous constraint
solving algorithm returns a substitution \(\theta\) for predicate
variables in \(C\) such that \(\theta C\) is valid.
%in a backward manner
The algorithm iteratively finds a solution for each predicate variable
\(P\) in \(C\) as follows:  The algorithm first computes equi-satisfiable
constraints \(C_P\) of the following form by eliminating the other
predicate variables in \(C\) than \(P\):
\begin{eqnarray*}
\phi_{P} \imply P(\seq{x}) \qquad
P(\seq{x}) \imply \phi_{P}'
\end{eqnarray*}
Here, \(\FV{\phi_{P}} \cap \FV{\phi_{P}'} \subseteq \set{\seq{x}}\)
always holds.  Intuitively, the predicate \(P(\seq{x})\) represents an
invariant of some subexpression \(e\) in the SHP, where some variable
\(\nu \in\set{\seq{x}}\) represents the value of \(e\) and each variable
in \(\set{\seq{x}} \setminus \set{\nu}\) represents a free variable in
\(e\).  \(\phi_P\) and \(\phi_{P}'\) respectively represent the
strongest condition satisfied by the value \(\nu\) and the weakest
condition on \(\nu\) required by the context of \(e\).
%
The algorithm then computes \(\mathcal{I}(\phi_P,\neg \phi_P')\) as a
solution for \(P(\seq{x})\) with the help of a technique called
interpolation~\cite{McMillan2005,Beyer2008} from automated theorem
proving.  Here, an interpolant \(\mathcal{I}(\phi_1,\phi_2)\) of
\(\phi_1\) and \(\phi_2\) (such that \(\phi_1\) and \(\phi_2\) are
inconsistent) is a formula \(\phi\) that satisfies the following
conditions:\footnote{Note that interpolants of \(\phi_1\) and \(\phi_2\)
are not unique.  Actually, existing theorem
provers~\cite{McMillan2005,Beyer2008} return one of them, which is
denoted by \(\mathcal{I}(\phi_1,\phi_2)\).}
\vspace{-5pt}
\begin{itemize}
\item \(\phi_1\) implies \(\phi\),
\item \(\phi\) and \(\phi_2\) are inconsistent, and
\item \(\FV{\phi} \subseteq \FV{\phi_1} \cap \FV{\phi_2}\).
\end{itemize}
\vspace{-3pt}
%
For the running example \(C_{\texttt{copy}}\), we obtain the following
constraints by eliminating the other predicate variables than \(P_1\):
\[
\begin{array}{rcl}
%\begin{eqnarray*}
x=1 \land y=0 \land x\neq0 \land \nu=1+y &\imply& P_1(x,\nu) \\
P_1(x,\nu) &\imply& \nu=x
%\end{eqnarray*}
\end{array}
\]
We then obtain, for example, the following solution for \(P_1(x,\nu)\):
\begin{eqnarray*}
\mathcal{I}(x=1 \land y=0 \land x\neq0 \land \nu=1+y,\neg \nu=x)
\equiv \nu=x.
\end{eqnarray*}
By substituting this for \(P_1\) in \(C_{\texttt{copy}}\), we get:
\[
\begin{array}{rcl}
%\begin{eqnarray*}
x=0 \land \nu=0 &\imply& P_2(x,\nu) \\
P_2(x,\nu) &\imply& (x+1\neq0 \land z=1+\nu \imply z=x+1)
%\end{eqnarray*}
\end{array}
\]
We then get, for example, the following solution for \(P_2(x,\nu)\):
\[
\begin{array}{rcl}
%\begin{eqnarray*}
&&\mathcal{I}(x=0 \land \nu=0,\neg (x+1\neq0 \land z=1+\nu \imply z=x+1)) \\
&\equiv& \nu=x.
%\end{eqnarray*}
\end{array}
\]
We thus obtain the following refinement types for \(D_{\texttt{copy}}\):
\[
\begin{array}{rcl}
%\begin{eqnarray*}
\texttt{copy1}&\COL&(x\COL\TFun{\INT}{\set{\nu\COL\INT \mid x=\nu}}) \\
\texttt{copy2}&\COL&(x\COL\TFun{\INT}{\set{\nu\COL\INT \mid x=\nu}})
%\end{eqnarray*}
\end{array}
\]

\subsection{Limitation of Previous Method}
\label{sec:limit}

We now explain the limitation of the previous method by using the
program in Figure~\ref{fig:copy}.  Let us consider the following SHP
\(D_{\texttt{cc}}\):
\begin{alltt}
 let copy1 x = assume (x<>0); 1 + copy2 (x-1)
 let copy2 x = assume (x=0); 0
 let copy3 x = assume (x<>0); 1 + copy4 (x-1)
 let copy4 x = assume (x=0); 0
 let main n = assume (copy3 (copy1 n) <> n); fail
\end{alltt}
The SHP corresponds to an infeasible error path where the else-branch of
\(\texttt{copy}\) is taken in the first and the third calls of
\(\texttt{copy}\), the then-branch is taken in the second and the fourth
calls of \(\texttt{copy}\), and the assertion in the \(\texttt{main}\)
function fails.

For the SHP \(D_{\texttt{cc}}\), we use the following type templates:
\[
\begin{array}{rcl}
%\begin{eqnarray*}
\texttt{copy1}&\COL&(x\COL\TFun{\INT}{\set{\nu\COL\INT \mid P_1(x,\nu)}}) \\
%\texttt{copy2}&\COL&(x\COL\TFun{\INT}{\set{\nu\COL\INT \mid P_2(x,\nu)}}) \\
%\texttt{copy3}&\COL&(x\COL\TFun{\INT}{\set{\nu\COL\INT \mid P_3(x,\nu)}}) \\
\vdots\quad &&\qquad\qquad \vdots \\
\texttt{copy4}&\COL&(x\COL\TFun{\INT}{\set{\nu\COL\INT \mid P_4(x,\nu)}})
%\end{eqnarray*}
\end{array}
\]
And, we get the following set \(C_{\texttt{cc}}\) of constraints:
\[
\begin{array}{rcl}
%\begin{eqnarray*}
x=0 \land y=0 &\imply& P_2(x,y) \\
P_2(x-1,y) \land x\neq0 \land z=1+y &\imply& P_1(x,z) \\
x=0 \land y=0 &\imply& P_4(x,y) \\
P_4(x-1,y) \land x\neq0 \land z=1+y &\imply& P_3(x,z) \\
P_1(n,x) \land P_3(x,y) &\imply& y=n
%\end{eqnarray*}
\end{array}
\]

By eliminating the other predicate variables than \(P_3\) (and with some
simplification), we get the following constraints \(C_{P_3}\):
\[
\begin{array}{rcl}
%\begin{eqnarray*}
x=1 \land \nu=1 &\imply& P_3(x,\nu) \\
%P_3(x,\nu) &\imply& (n=1 \land x=1 \imply \nu=n)
P_3(x,\nu) &\imply& (x=1 \imply \nu=1)
%\end{eqnarray*}
\end{array}
\]
Existing interpolating proves such as \cite{Beyer2008} returns the
following solution for \(P_3(x,\nu)\):
\begin{eqnarray*}
%&&\mathcal{I}(x=1 \land \nu=1,\neg (n=1 \land x=1 \imply \nu=n)) \\
\mathcal{I}(x=1 \land \nu=1,\neg (x=1 \imply \nu=1))
\equiv x=1 \land \nu=1.
\end{eqnarray*}
Note here that the solution is specific to the calling context of the
particular function \texttt{copy3}, and cannot be used as a solution for
\(P_2\) and \(P_4\).  We here want to get more general solutions like
\(\lambda (\nu,x).\nu=x\) which are more likely to constitute an
invariant of the function \texttt{copy} in the original program.  For
this purpose, we believe it is desirable to find the same solution (if
possible) for ``related'' predicate variables which represent (possibly
different) refinement predicates for the same argument or return value
of the same function in the original program.  For the running example
\(C_{\texttt{cc}}\), we want to get the same solution for
\(P_1,\dots,P_4\), and \(\lambda (\nu,x).\nu=x\) in fact satisfies this
extra constraint.

\subsection{Extended Method}
\label{sec:ext}

We now explain our extension of the previous method to remedy the
limitation discussed in Section~\ref{sec:limit}.  The extended predicate
discovery method is based on the framework of the previous method
overviewed in Section~\ref{sec:prev}, but the component for refinement
type inference is extended so that it can merge and generalize
information from multiple calling contexts of a function in multiple
infeasible error paths.  This enables MoCHi to infer a general
refinement type of the function that type-checks the multiple calling
contexts, while preserving the path- and context-sensitivity.  In other
words, the extended method generates constraints from multiple
infeasible error paths (see Section~\ref{sec:extcg}), and tries to find
the same solution (if possible) for related predicate variables (see
Section~\ref{sec:extcs}).

\subsubsection{Extensions of Constraint Generation}
\label{sec:extcg}

We extend the previous constraint generation algorithm overviewed in
Section~\ref{sec:cg} as follows.
\vspace{-5pt}
\begin{itemize}
\item For each CEGAR iteration, we generate constraints from multiple
infeasible error paths instead of a single path:  We keep the set
\(\set{\pi_1,\cdots,\pi_n}\) of the infeasible error paths found so far,
generate the set \(C_i\) of Horn clauses for each path \(\pi_i\), and
pass \(C=C_1 \cup \dots \cup C_n\) to the extended constraint solving
algorithm described in Section~\ref{sec:extcs} as an input.
%For example, \todo{}
\item We also construct and pass an equivalence relation \(E\) on the
predicate variables in \(C\) such that \(P\ E\ Q\) if and only if the
predicate variables \(P\) and \(Q\) represent (possibly different)
refinement predicates for the same argument or return value of the same
function in the original program.  For example, we obtain the trivial
equivalence relation \(E_{\texttt{cc}}=\set{P_1,\dots,P_4}
\times\set{P_1,\dots,P_4}\) for \(C_{\texttt{cc}}\).  The constraint
solving algorithm in Section~\ref{sec:extcs} exploits \(E\) to find
general solutions for \(C\).
\end{itemize}
\vspace{-5pt}
Thus, the extended algorithm generates a pair \((C,E)\) of Horn clauses
\(C\) for multiple paths and an equivalence relation \(E\) on the
predicate variables in \(C\) unlike the previous algorithm which
generates only Horn clauses for a single path.  Here, the pair \((C,E)\)
of constraints can be viewed as hierarchical constraints where \(C\)
must be always satisfied and \(E\) should be satisfied if possible.

\subsubsection{Extensions of Constraint Solving}
\label{sec:extcs}

In this section, we extend the previous constraint solving algorithm
overviewed in Section~\ref{sec:cs}.  Given a pair \((C,E)\) of Horn
clauses \(C\) and an equivalence relation \(E\) on the predicate
variables in \(C\), the algorithm returns a substitution \(\theta\) for
the predicate variables in \(C\) such that \(\theta C\) is valid.  A
distinguishing feature of the algorithm is that it tries to find the
same solution for predicate variables related by \(E\) if possible.
This enables the algorithm to obtain general predicates, which are more
likely to constitute invariants. % of the original program.

%%%\begin{figure}
%%%\todo{}
%%%\caption{The Overall Structure of Our Extended Constraint Solving Algorithm}
%%%\label{fig:extcs}
%%%\end{figure}

%%%The overall structure of the algorithm is shown in Figure~\ref{fig:extcs}.
The extended constraint solving algorithm proceeds as follows:
\vspace{-10pt}
\begin{enumerate}
\item Find a set \(S\) of predicate variables which are related by \(E\)
and may have the same solution in \(C\).
\item Find a candidate solution \(\lambda \seq{x}.\phi\) for all
predicate variable \(Q \in S\).
\item Substitute \(\lambda \seq{x}.\phi\) for predicate variables \(S\)
in \(C\) and repeat the entire procedure if the result still contains a
predicate variable.
\end{enumerate}
\vspace{-10pt}

\paragraph{Finding a set \(S\) of predicate variables:}
To find a set of predicate variables that may have the same solution in
\(C\), for each predicate variable \(P\) in \(C\), we compute
constraints \(C_P\) from \(C\) by eliminating the other predicate
variables than \(P\).
%
For the running example \(C_{\texttt{cc}}\), we obtain:
\[
\begin{array}{rcl}
%\begin{eqnarray*}
C_{P_1}&=&\{x=1 \land \nu=1 \imply P_1(x,\nu), \\
&&\ P_1(x,\nu) \imply (\nu=1 \imply x=1)\}, \\
C_{P_2}&=&\{x=0 \land \nu=0 \imply P_2(x,\nu), \\
&&\ P_2(x,\nu) \imply (\nu=0 \imply (x=-1 \lor x=0))\}, \\
C_{P_3}&=&\{x=1 \land \nu=1 \imply P_3(x,\nu), \\
&&\ P_3(x,\nu) \imply (x=1 \imply \nu=1)\}, \\
C_{P_4}&=&\{x=0 \land \nu=0 \imply P_4(x,\nu), \\
&&\ P_4(x,\nu) \imply (x=0 \imply \nu=0)\}.
%\end{eqnarray*}
\end{array}
\]

%%%In general, the least upper bound of some predicate variable may not
%%%exist because the constraint generation algorithm in
%%%Section~\ref{sec:extcg} may generate a Horn clause of the form
%%%\(P(x)\land P(y) \imply \phi\).  Note here that \(P\) occurs twice in
%%%the left hand side of the constraint.  In such a case, the least upper
%%%bound for \(P\) may not exist.

Let \(\set{P_1,\dots,P_m}\) be the set of predicate variables in \(C\).
%which have the least upper bound.
We pick an equivalence class \(S_0 \in \set{P_1,\dots,P_m} / E\) (e.g.,
the largest one), and further classify \(S_0\) by using
\(C_{P_1},\dots,C_{P_m}\) so that predicate variables which never have
the same solution are separated.
%
Formally, we find \(S_1\dots,S_n\) such that:
\begin{itemize}
\item \(S_0 = S_1 \cup \dots \cup S_n\),
\item \(\phi_{P_{i,1}} \lor \dots \lor\phi_{P_{i,\ell_i}}\) implies
\(\phi_{P_{i,1}}' \land \dots \land\phi_{P_{i,\ell_i}}'\) for each \(i
\in \set{1,\dots,n}\), and
\item \(\phi_{P_{i,1}} \lor \dots \lor\phi_{P_{i,\ell_i}} \lor
\phi_{P_{j,1}} \lor \dots \lor\phi_{P_{j,\ell_j}}\) does not imply
\(\phi_{P_{i,1}}' \land \dots \land\phi_{P_{i,\ell_i}}' \land
\phi_{P_{j,1}}' \land \dots \land\phi_{P_{j,\ell_j}}'\) for each \(i,j
\in \set{1,\dots,n}\) such that \(i \neq j\).
\end{itemize}
Here, \(C_{P}=\set{\phi_{P} \imply P(\seq{x}),P(\seq{x}) \imply
\phi_{P}'}\) and \(S_i=\set{P_{i,1},\dots,P_{i,\ell_i}}\).

We then pick some \(S \in \set{S_1,\dots,S_n}\) (e.g., the largest one).
 For the running example \(C_{\texttt{cc}}\), we get
\(S=S_0=S_1=\set{P_1,\dots,P_4}\) since \(\phi_{P_1} \lor \dots
\lor\phi_{P_4}\) implies \(\phi_{P_1}' \land \dots \land \phi_{P_4}'\).

\paragraph{Finding a candidate solution \(\lambda \seq{x}.\phi\) for \(S\):}
We find a single candidate solution \(\lambda \seq{x}.\phi\) for all the
predicate variables \(Q_1,\dots,Q_{\ell} \in S\) by simultaneously
solving \(C_{Q_1},\dots,C_{Q_{\ell}}\) unlike the previous method.
Formally, we find \(\phi\) such that:
\vspace{-5pt}
\begin{itemize}
\item \(\phi_{Q_1} \lor \dots \lor \phi_{Q_{\ell}}\) implies \(\phi\),
\item \(\phi\) implies \(\phi_{Q_1}' \land \dots \land \phi_{Q_{\ell}}'\), and
\item \(\FV{\phi} \subseteq \set{\seq{x}}\).
\end{itemize}
Here, \(C_{Q_i}=\set{\phi_{Q_i} \imply Q_i(\seq{x}),Q_i(\seq{x}) \imply
\phi_{Q_i}'}\).  We can compute such a formula \(\phi\) as an
interpolant \(\mathcal{I}(\phi_{Q_1} \lor \dots \lor
\phi_{Q_{\ell}},\neg (\phi_{Q_1}' \land \dots \land \phi_{Q_{\ell}}'))\)
but the three conditions of interpolants are not always sufficient for
our purpose to find general predicates.  Actually, we want to obtain as
simple interpolant as possible with respect to the number of
disjunctions.
% and conjunctions.
To this end, we propose a new heuristic operator \(\mathcal{J}\) that
combines the interpolation \(\mathcal{I}\) and convex hull operators.
Let us write \(\mathcal{H}(\phi)\) to denote the convex hull of \(\phi\).
 For formulas \(\phi_1\) and \(\phi_2\) (such that \(\phi_1\) and
\(\phi_2\) are inconsistent), the new operator
\(\mathcal{J}(\phi_1,\phi_2)\) is defined as follows:
\[
\mathcal{J}(\phi_1,\phi_2) =
\left\{
\begin{array}{ll}
\mathcal{I}(\mathcal{H}(\phi_1),\mathcal{H}(\phi_2)) & (\mbox{if~}\mathcal{H}(\phi_1) \INCON \mathcal{H}(\phi_2)) \\
\mathcal{I}(\mathcal{H}(\phi_1),\phi_2) & (\mbox{if~}\neg (\mathcal{H}(\phi_1) \INCON \mathcal{H}(\phi_2)) \land \\
&\ \quad \mathcal{H}(\phi_1) \INCON \phi_2) \\
\mathcal{I}(\phi_1,\phi_2) & (\mbox{otherwise})
\end{array}
\right.
\]
Here, we write \(\phi_1 \INCON \phi_2\) to denote that \(\phi_1\) and
\(\phi_2\) are inconsistent.  Note here that the use of the convex hull
operator enables us to eliminate disjunctions in \(\phi_1\) and
\(\phi_2\), which are passed to an interpolating theorem prover.
%the interpolation operator \(\mathcal{I}\).
In the experiments reported in Section~\ref{sec:experiments}, this often
reduced the number of disjunctions in the output of the interpolating
prover, and hence makes the output more likely to constitute invariants.
%
Thus, we use the new operator \(\mathcal{J}\) instead of \(\mathcal{I}\)
to compute \(\lambda \seq{x}. \mathcal{J}(\phi_{Q_1} \lor \dots\lor
\phi_{Q_{\ell}},\neg (\phi_{Q_1}' \land \dots \land\phi_{Q_{\ell}}'))\)
as a candidate solution \(\lambda \seq{x}. \phi\) for all the predicate
variables \(Q_1,\dots,Q_{\ell} \in S\).  For the running example
\(C_{\texttt{cc}}\), we obtain, for example, the following candidate
solution \(\lambda (x, \nu). \phi\) for \(P_1,\dots,P_4\):
%an interpolating prover returned ... in the experiments reported in Section~\ref{sec:experiments}:
\vspace{-5pt}
\[
\begin{array}{rcl}
%\begin{eqnarray*}
\phi&=& \mathcal{J}(\phi_{P_1} \lor \dots \lor \phi_{P_4},\neg (\phi_{P_1}' \land \dots \land \phi_{P_4}')) \\
%&=& \mathcal{I}(\mathcal{H}(\phi_{P_1} \lor \dots \lor \phi_{P_4}),\neg (\phi_{P_1}' \land \dots \land \phi_{P_4}')) \\
&=& \mathcal{I}(\mathcal{H}(x=\nu=0 \lor x=\nu=1),\neg (\phi_{P_1}' \land \dots \land \phi_{P_4}')) \\
&=& \mathcal{I}(0 \leq x=\nu \leq 1,\neg (\phi_{P_1}' \land \dots \land \phi_{P_4}')) \\
&\equiv& x=\nu
%\end{eqnarray*}
\end{array}
\]

\paragraph{Substituting \(\lambda \seq{x}.\phi\) for \(S\) in \(C\):}
We then substitute the candidate solution \(\lambda \seq{x}.\phi\) for
\(S=\set{Q_1,\dots,Q_{\ell}}\) in \(C\).  Note, however, that we cannot
always substitute all the predicate variables in \(S\) with the
candidate solution \(\lambda\seq{x}.\phi\) because \(Q_i\) may depend on
\(Q_j\) for some \(i \neq j\).  For example, let us consider the
following constraints:
\begin{eqnarray*}
x=0 \imply Q_1(x), \quad
Q_1(x) \imply Q_2(x+1), \quad
Q_2(x) \imply 0 \leq x \leq 2
\end{eqnarray*}
From the constraints, we get:
\begin{eqnarray*}
C_{Q_1}&=&\{\nu=0 \imply Q_1(\nu), Q_1(\nu) \imply -1 \leq \nu\leq 1\}, \\
C_{Q_2}&=&\{\nu=1 \imply Q_2(\nu), Q_2(\nu) \imply 0 \leq \nu \leq 2\}.
\end{eqnarray*}
Thus, we obtain, for example, \(\mathcal{J}(\nu=0 \lor \nu=1,\neg (-1
\leq \nu\leq 1 \land 0 \leq \nu \leq 2)) \equiv 0 \leq \nu \leq 1\) as a
candidate solution for \(Q_1(\nu)\) and \(Q_2(\nu)\).  However,
\([\lambda\nu.0 \leq \nu \leq 1/Q_1,\lambda \nu.0 \leq \nu \leq
1/Q_2](Q_1(x) \imply Q_2(x+1))\) is not valid.  Actually, it is only
safe to substitute \(\lambda \nu.0 \leq \nu\leq 1\) for either \(Q_1\)
or \(Q_2\).

Therefore, we find and substitute only a maximal nonempty subset \(M\)
of \(S\) for which we can safely substitute \(\lambda\seq{x}.\phi\) (i.e.,
\(\set{R \mapsto \lambda\seq{x}.\phi \mid R \in M}C\) is equi-satisfiable
with \(C\)).
%
%%%such that \(\lambda\seq{x}.\phi\) is in fact a solution for every predicate variable in \(M\).
%
%%%Formally, we find \(M=\set{R_1,\dots,R_m}\) such that:
%%%\begin{itemize}
%%%\item \(M \subseteq S\),
%%%\item \(\theta[\lambda \seq{x}.\phi/R_1,\dots,\lambda
%%%\seq{x}.\phi/R_m]C\) is valid for some substitution \(\theta\) for
%%%predicate variables, and
%%%\item for any \(R_0 \in S \setminus M\) and \(\theta\) the following
%%%formula is not valid: \(\theta[\lambda \seq{x}.\phi/R_0,\lambda
%%%\seq{x}.\phi/R_1,\dots,\lambda \seq{x}.\phi/R_m]C\).
%%%\end{itemize}
For the running example \(C_{\texttt{cc}}\), it is in fact safe to
substitute the candidate solution \(\lambda (x,\nu). x=\nu\) for
\(P_1,\dots,P_4\) (i.e., \(M=S\)).  As a result of the substitution, all
the predicate variables in \(C_{\texttt{cc}}\) are eliminated.  Thus, we
obtain the following refinement types for \(D_{\texttt{cc}}\):
\[
\begin{array}{rcl}
%\begin{eqnarray*}
\texttt{copy1}&\COL&(x\COL\TFun{\INT}{\set{\nu\COL\INT \mid x=\nu}}) \\
%\texttt{copy2}&\COL&(x\COL\TFun{\INT}{\set{\nu\COL\INT \mid x=\nu}}) \\
%\texttt{copy3}&\COL&(x\COL\TFun{\INT}{\set{\nu\COL\INT \mid x=\nu}}) \\
\vdots\quad &&\qquad\qquad \vdots \\
\texttt{copy4}&\COL&(x\COL\TFun{\INT}{\set{\nu\COL\INT \mid x=\nu}})
%\end{eqnarray*}
\end{array}
\]


%\todo{discuss termination?}

%%%infer a refinement type of each
%%%subexpression \(e\) of an ordinary ML type \(\tau\):
%%%  The method then computes an
%%%interpolant \(\phi\) of \(\phi_{post}\) and \(\phi_{pre}\), and returns
%%%\(\set{\nu:\tau \mid \phi}\) as a refinement type of \(e\).
%
%%%%Thus, the method considers both forward and backward information of \(e\)
%%%%respectively obtained from \(e\) and the context of \(e\).
