\vspace{-5pt}
\section{Source Language}
\label{sec:language}
In this section, we introduce the source language of our verification method.
%We extend the language for recursive data structures later.
%As mentioned in Section~\ref{sec:intro},
%We formalize the verification method for a language with lists.
%and argue about recursive data structures later.
%Here, we formalize a language for lists. We extend the language for
%recursive data structures later.

The source language is a simply-typed call-by-value higher-order
functional language with recursion and integers (a la ``PCF'') with the
syntax defined as follows:
\[
\begin{array}{lrl}
D\text{ (programs)} &::=& \{f_1 = v_1, \dots, f_n = v_n\} \\
t\text{ (terms)}
  &::=& n \mid x \mid \Abs{x}{t} \mid t_1\ t_2 \mid \Op{v_1,\dots,v_n} \\
 &\mid& \If{t_1}{t_2}{t_3} \mid \FAIL \\
v\text{ (values)} &::=& n \mid x \mid \Abs{x}{t} \\
\end{array}
\]
%  A program consists a set of top-level functions.
The meta-variable $\OP$ ranges over the set of operators over integers.
The expressions are standard except that there is a primitive $\FAIL$
that aborts a program.  We use $\TRUE$ and $\FALSE$ as aliases of $0$
and $1$, and we write $\Assert{t}$ for $\If{t}{0}{\FAIL}$,
$\Let{x}{t_1}{t_2}$ for $(\Abs{x}{t_2})\ t_1$, and $(v_1\ \OP\ v_2)$ for $\Op{v_1,v_2}$.  We
assume that (i) a program is well-typed in the standard simple type system,
where the set of types is given as $\tau ::= \INT \mid \TFun{\tau_1}{\tau_2}$, (ii)
every function in a program $D$ has a function type, and (iii) $D$ contains a
distinguished function symbol $\main \in \set{f_1,\dots,f_n}$ whose type
is $\TFun{\INT}{\INT}$.

%\begin{figure}[t]
%\[
%\begin{array}{lrl}
%D\text{ (programs)} &::=& \{f_1 = v_1, \dots, f_n = v_n\} \\
%t\text{ (terms)}
%  &::=& n \mid x \mid \Abs{x}{t} \mid t_1\ t_2 \mid \Op{v_1,\dots,v_n} \\
% &\mid& \If{t_1}{t_2}{t_3} \mid \FAIL \\
%v\text{ (values)} &::=& n \mid x \mid \Abs{x}{t} \\
%\tau\text{ (types)} &::=& \INT \mid \TFun{\tau_1}{\tau_2} \\
%\end{array}
%\]
%\caption{The syntax of source language}
%\label{fig:source-syntax}
%\end{figure}

The goal of our verification is to check whether $\App{\main}{n}
\not\reds_D \FAIL$ for all integer $n$.

%\begin{figure}[t]
%\begin{minipage}{\widthcoef\textwidth}
%\[
%\begin{array}{lrl}
%D\text{ (programs)} &::=& \{f_1 = v_1, \dots, f_n = v_n\} \\
%t\text{ (terms)}
%  &::=& n \mid x \mid \Abs{x}{t} \mid t_1\ t_2 \mid \Op{v_1,\dots,v_n} \mid \If{t_1}{t_2}{t_3} \\
% &\mid& \FAIL \mid \Pair{v_1}{v_2} \mid \Fst{t} \mid \Snd{t} \mid \NIL \mid \Cons{v_1}{v_2} \\
% &\mid& (\MATCH\ {t_1}\ \WITH\ \NIL \ra {t_2} \mid \Cons{x_1}{x_2} \ra {t_3}) \\
%
%v\text{ (values)}
%  &::=& n \mid \Abs{x}{t} \mid \Pair{v_1}{v_2} \mid \NIL \mid \Cons{v_1}{v_2} \\
%
%E\text{ (eval. ctx.)}
%  &::=& [\,] \mid E\ t \mid \App{\Abs{x}{t}}{E} \mid \If{E}{t_2}{t_3} \mid \Fst{E} \mid \Snd{E} \\
% &\mid& (\Match{E}{t_2}{t_3}) \\
%
%\tau\text{ (types)} &::=& \INT \mid \TFun{\tau_1}{\tau_2} \mid \TPair{\tau_1}{\tau_2} \mid \List{\tau} \\
%\end{array}
%\]
%\end{minipage}
%\caption{The syntax of source language}
%\label{fig:source-syntax}
%\end{figure}


%\begin{figure}[t]
%\begin{minipage}{\textwidth}
%  \infrule[E-Var]{(f,t) \in D}{f \Red{D} t}
%
%\medskip
%
%  \infax[E-App]{\App{(\Abs{x}{M})}{v} \Red{D} [x \mapsto v]t}
%
%\medskip
%
%  \infax[E-Op]{\Op{v_1,\dots,v_n} \Red{D} \Denote{\OP}(v_1,\dots,v_n)}
%
%\medskip
%
%  \infax[E-If-Then]{\If{0}{t_1}{t_2} \Red{D} t_1}
%
%\medskip
%
%  \infrule[E-If-Else]{n \neq 0}{\If{n}{t_1}{t_2} \Red{D} t_2}
%
%\medskip
%
%  \infax[E-Fst]{\Fst{\Pair{v_1}{v_2}} \Red{D} v_1}
%
%\medskip
%
%  \infax[E-Snd]{\Snd{\Pair{v_1}{v_2}} \Red{D} v_2}
%
%\medskip
%
%  \infax[E-Match-Nil]{\Match{\NIL}{t_2}{t_3} \Red{D} t_2}
%
%\medskip
%
%  \infax[E-Match-Cons]{\Match{\Cons{v_1}{v_2}}{t_2}{t_3} \Red{D} t_3[\subst{x_1}{v_1}][\subst{x_2}{v_2}]}
%
%\medskip
%
%  \infrule[E-Context]{t \Red{D} t'}{E[t] \Red{D} E[t']}
%
%\medskip
%
%  \infax[E-Fail]{E[\FAIL] \Red{D} \FAIL}
%\end{minipage}
%\caption{The semantics of source language}
%\label{fig:source-semantics}
%\end{figure}
