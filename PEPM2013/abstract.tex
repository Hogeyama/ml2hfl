\begin{abstract}
In our recent paper, we have shown how to construct a fully-automated
program verification tool (so called a ``software model checker'') for a
tiny subset of functional language ML, by combining higher-order model
checking, predicate abstraction, and CEGAR.  This can be viewed as a
higher-order counterpart of previous software model checkers for
imperative languages like BLAST and SLAM.  The na\"{\i}ve application of
the proposed approach, however, suffered from scalability problems, both
in terms of efficiency and supported language features.  To obtain more
scalable software model checkers for full-scale functional languages, we
propose several optimizations and extensions of the previous
approach. Among others, we introduce (i) selective predicate abstraction
and (ii) selective CPS transformation as optimization techniques, and
(iii) functional encoding of recursive data structures and control
operations to support a larger subset of ML.  We have implemented the
proposed methods, and obtained promising results.
%%%Higher-order model checking has been studied recently, and applied to
%%%verification of higher-order functional programs.  In previous work, we
%%%have developed an automated verifier for functional programs. The
%%%verification framework, however, have the following problems: (i) it is
%%%not scalable in a na\"{\i}ve way, and (ii) it cannot directly deal with
%%%recursive data types (e.g., lists and trees) and control operators
%%%(e.g., exceptions and call/ccs), which are necessary to practical
%%%functional languages.
%%%
%%%In this paper, to overcome these problems, we extend and refine the
%%%verification framework in the following way. First, we formalize
%%%functional encoding of recursive data types and control operators.
%%%Second, we define techniques of selective CPS transformation and reduced
%%%abstraction.  We have implemented a prototype verifier for functional
%%%programs based on the framework and tested for several programs.
\end{abstract}



%\section{Introduction}
%[background]
%
%[Show running example of verifying a program with lists]
%
%[discovering general predicates]
%
%[Show running example of expanding non-recursive functions]
%
%[the overview of this framework]
%
%\section{Language}
% [Introduction of the target language (= list constructor/destructor + language of PLDI2011 + pairs.)]
%
%\section{Model Checking for Higher-order Programs with Integers}
% [Introduction of language (of PLDI2011 + pairs.)]
%
% [Description of the framework of PLDI2011.]
%
%\section{Optimization}
%\subsection{Inlining Non-recursive Function}
% [Show an example of the verification with inlining]
%
%\subsection{Selective CPS Transformation}
% [Show an example of the selective CPS transformation]
%
%\section{Functional Encoding of Lists}
% [Show how to encode lists, and how to translate the program with lists]
%\subsection{Extensions for Recursive Data Structures}
% [Show how to encode other data structures]
%
%\section{Extension for Control Operators}
% [Introduce language with control operators (exceptions, call/cc, etc.), and just to say ``Just to do CPS transformation'']
%
%\section{Implementation and Preliminary Experiments}
%
%\section{Related work}
% [Comparison with container abstraction (Dillig et al. POPL11, etc.)]
%
% [Comparison with verification of functional programs with data structures (liquid types, sized types, HMTT, PMRS, etc.)]
%
% [Comparison with others (Soonho Kong et al. APLAS10, etc.)]
%
%\section{Conclusion}
%
%
